\title{\texttt{pst-plot}\\plotting data and mathematical functions\thanks{%
    This document was written with \texttt{Kile: 1.7} (\texttt{Qt: 3.1.1; KDE: 3.3;}
    \url{http://sourceforge.net/projects/kile/}) and the PDF output
    was build with VTeX/Free (\url{http://www.micropress-inc.com/linux})}\\
    \small v.\pstplotFV}
\author{Herbert Vo�}
\date{\today}

\maketitle

\begin{abstract}
This version of \verb+pst-plot+ includes all the macros which were for testing
part of the  \verb+pstricks-add+-package. This documentation shows only these
extensions. For the other macros have a look into the old PSTricks
documentation.

\verb+pst-plot+ uses the extended version of the keyval package. So be sure, that
you have installed \verb+pst-xkey+ which is part of the \verb+xkeyval+-package and that all
packages, that uses the old keyval interface are loaded \textbf{before} the \verb+xkeyval+.
\end{abstract}

\clearpage

\tableofcontents

\clearpage

\lstset{wide=true}

%------------------------------------------------------------------------------------
\section{ New options}
%------------------------------------------------------------------------------------
The option \verb+tickstyle=full|top|bottom+ is already present in the \verb+pst-plot+
package, but it is mentioned here for some completness.



\begin{longtable}{>{\ttfamily}l>{\ttfamily}l>{\ttfamily}l}
\caption{All new parameters for \texttt{pst-plot}}\\[-5pt]
Name         & Type  & Default\\\hline
\endfirsthead
Name         & Type  & Default\\\hline
\endhead
infix                 & none|TeX|PS & none\\ %ok
comma                 & false|true & false\\ %ok
xAxis                 & false|true & true\\%ok
yAxis                 & false|true & true\\%ok
xyAxes                & false|true & true\\%ok
xDecimals             & <number> or empty  & \{\}\\%ok
yDecimals             & <number> or empty  & \{\}\\%ok
xyDecimals            & <number> or empty   & \{\}\\%ok
xLabel                & <anything> & \{\}\\%ok
yLabel                & <anything> & \{\}\\%ok
xyLabel               & <anything> & \{\}\\%ok
tickstyle             & full|top|bottom & full\\%ok
ticks                 & <all|x|y|none>   & all\\%ok
labels                & <all|x|y|none>   & all\\%ok
trigLabels            & false|true       & false\\%ok
subticks              & <number>         & 0\\
xsubticks             & <number>         & 0\\
ysubticks             & <number>         & 0\\
ticksize              & <length [length]>         & -4pt 4pt\\
subticksize           & <number>         & 0.75\\
tickwidth             & <length>         & 0.5\verb+\pslinewidth+\\
subtickwidth          & <length>         & 0.25\verb+\pslinewidth+\\
tickcolor             & <color>          & black\\
xtickcolor            & <color>          & black\\
ytickcolor            & <color>          & black\\
subtickcolor          & <color>          & darkgray\\
xsubtickcolor         & <color>          & darkgray\\
ysubtickcolor         & <color>          & darkgray\\
ticklinestyle         & solid|dashed|dotted|none & solid\\
subticklinestyle      & solid|dashed|dotted|none & solid\\
xlabelFactor          & <anything> & \{\}\\
ylabelFactor          & <anything> & \{\}\\
xlogBase              & <number> or empty   & \{\}\\
ylogBase              & <number> or empty   & \{\}\\
xylogBase             & <number> or empty   & \{\}\\
logLines              & <none|x|y|all>   & none\\
ignoreLines           & <number>          & 0\\
nStep                 & <number>          & 1\\
nStart                & <number>          & 0\\
nEnd                  & <number> or empty          & \{\}\\
xStep                 & <number>          & 0\\
yStep                 & <number>          & 0\\
xStart                & <number> or empty         & \{\}\\
yStart                & <number> or empty         & \{\}\\
xEnd                  & <number> or empty         & \{\}\\
yEnd                  & <number> or empty         & \{\}\\
plotNo                & <number>          & 1\\
plotNoMax             & <number>          & 1\\
xAxisLabel            & <anything> & \{\}\\
yAxisLabel            & <anything> & \{\}\\
xAxisLabelPos         & <(x,y)> or empty & \{\}\\
yAxisLabelPos         & <(x,y)> or empty & \{\}\\
llx                   & <length>  & 0pt\\
lly                   & <length>  & 0pt\\
urx                   & <length>  & 0pt\\
ury                   & <length>  & 0pt\\
polarplot             & false|true    & false\\
\end{longtable}


%--------------------------------------------------------------------------------------
\subsection{\texttt{infix}}
%--------------------------------------------------------------------------------------

By default the function of \verb+\psplot+ has to be described in Reversed Polish Notation, also calles
\texttt{postfix notation}.
The option \verb+infix=none|TeX|PS+ allows to do this in the common algebraic notation, also
called \texttt{infix notation}. E.g.:

\begin{tabular}{l|l}
RPN & infix\\\hline
\verb+x ln+ & \verb+ln(x)+\\
\verb+x cos 2.71 x neg 10 div exp mul+ & \verb+cos(x)*2.71^(-x/10)+\\
\verb+1 x div cos 4 mul+ & \verb+4*cos(1/x)+
\end{tabular}

\verb+pst-plot+ allows two different infix-postfix conversion modes. With \verb+infix=TeX+ the conversion is done
on \TeX{} level (inside \verb+pst-plot+) and with \verb+infix=PS+ it is done on \PS{} level (outside of 
\verb+pst-plot+). This is in general not important for the normal
user, but allows advanced users further developing.


\begin{showexample}[pos=a,wide]
\psgraph(-10,-3.5)(10,3.5){.9\linewidth}{5cm}
  \psset{infix=PS, plotpoints=101}
  \psplot[linecolor=yellow, linewidth=4\pslinewidth]{-10}{10}{3*sin(x)}%
  \psplot[linecolor=red, showpoints=true,VarStep=0.8]{-10}{10}{3*sin(x)}
\endpsgraph
\end{showexample}

\begin{showexample}[pos=a,wide]
\psgraph(0,-4.25)(20,4.25){.9\linewidth}{5cm}
  \psset{infix=PS, plotpoints=501}
  \psplot[linecolor=yellow, linewidth=4\pslinewidth]{0.01}{20}{ln(x)}%
  \psplot[linecolor=red]{0.01}{20}{ln(x)}
  \psplot[linecolor=yellow, linewidth=4\pslinewidth]{0}{20}{cos(x)*2.71^(-x/10)}
  \psplot[linecolor=blue, showpoints=true,VarStep=0.8]{0}{20}{cos(x)*2.71^(-x/10)}
\endpsgraph
\end{showexample}

\begin{showexample}[pos=a,wide]
\begin{psgraph}(0,-4)(0.2,4){.9\linewidth}{5cm}
  \psset{infix=PS, plotpoints=501}
  \psplot[linecolor=yellow, linewidth=4\pslinewidth]{0.02}{.2}{4*cos(1/x)}%
  \psplot[linecolor=red,VarStep=0.95]{.02}{.2}{4*cos(1/x)}%
\end{psgraph}
\end{showexample}


%--------------------------------------------------------------------------------------
\subsection{\texttt{comma}}
%--------------------------------------------------------------------------------------
Syntax:
\begin{verbatim}
comma=false|true
\end{verbatim}
Setting this option to true gives labels with a comma as a decimal separator instead
of the dot. \verb|comma| and \verb|comma=true| is the same.

\resetPSTPlotOptions
\medskip
\begin{showexample}[width=5.5cm]
\begin{pspicture}(-0.5,-0.5)(5,5.5)
\psaxes[Dx=1.5,Dy=0.5,comma]{->}(5,5)
\psplot[linecolor=red,linewidth=3pt]{0}{4.5}%
   {x 180 mul 1.52 div cos 2 mul 2.5 add}
\psline[linestyle=dashed](0,2.5)(4.5,2.5)
\end{pspicture}
\end{showexample}



%--------------------------------------------------------------------------------------
\subsection{\texttt{xyAxes}, \texttt{xAxis} and \texttt{yAxis}}
%--------------------------------------------------------------------------------------
Syntax:
\begin{verbatim}
xyAxes=true|false
xAxis=true|false
yAxis=true|false
\end{verbatim}

Sometimes there is only a need for one axis with ticks. In this case you can set one
of the following options to false. The \verb+xyAxes+ makes only sense, when you want
to set both, x and y to true with only one command again to the default, because with
\verb+xyAxes=false+ you get nothing with the \verb+psaxes+ macro.


\resetPSTPlotOptions%
\begin{showexample}[pos=a]
\begin{pspicture}(8,1)
\psaxes[yAxis=false,linecolor=blue]{->}(0,0.5)(8,0.5)
\end{pspicture}%
\begin{pspicture}(1,8)
\psaxes[xAxis=false,linecolor=red]{->}(0.5,0)(0.5,8)
\end{pspicture}\hspace{2em}
\begin{pspicture}(1,8)
\psaxes[xAxis=false,linecolor=red,labelsep=-20pt]{->}(0.5,0)(0.5,8)
\end{pspicture}
\begin{pspicture}(1,8)
\psaxes[xAxis=false,linecolor=red]{->}(0.5,0)(0.501,8)
\end{pspicture}%
\end{showexample}

As seen in the example, a single y axis gets the labels on the right side. This can be
changed in two ways, first with the option \verb+labelsep+ and second with a very
short and therefore invisible x-axis (right example). 




%--------------------------------------------------------------------------------------
\subsection{\texttt{xyDecimals}, \texttt{xDecimals} and \texttt{yDecimals}}
%--------------------------------------------------------------------------------------
Syntax:
\begin{verbatim}
xyDecimals=<number>
xDecimals=<any>
yDecimals=<any>
\end{verbatim}
By default the labels of the axes get numbers with or without decimals, just depending to the
numbers. With these options \verb|??Decimals| it is possible to determine the decimals,
where the option \verb|xyDecimals| sets this identical for both axes.
The default setting \verb|{}| means, that you'll get the standard behaviour.


\begin{showexample}[width=6cm]
\begin{pspicture}(-1.5,-0.5)(5,4.75)
  \psaxes[xyDecimals=2]{->}(0,0)(4.5,4.5)
\end{pspicture}
\end{showexample}


\begin{showexample}[pos=a]
\psset{xunit=10cm, yunit=0.01cm}
\begin{pspicture}(-0.15,-150)(1.5,550.0)
  \psaxes[Dx=0.25,Dy=100,tickstyle=bottom,xyLabel=\footnotesize,comma=true,%
    xDecimals=3,yDecimals=1]{->}(0,0)(0,-100)(1.4,520)
\end{pspicture}
\end{showexample}



%--------------------------------------------------------------------------------------
\subsection{\texttt{xyLabel}, \texttt{xLabel} and \texttt{yLabel}}
%--------------------------------------------------------------------------------------
Syntax:
\begin{verbatim}
xyLabel=<any>
xLabel=<any>
yLabel=<any>
\end{verbatim}
There are no special keywords to change the \index{labelstyle}labelstyle for the \verb|\psaxes|
macro. With \verb|xyLabel| it is possible to set both axes with the same command sequence. Unlike to the default \verb|pst-plot| package the coordinates are not printed in mathmode. This makes it easier to choose other text styles.

\resetPSTPlotOptions
\begin{showexample}[pos=a]
\psset{yunit=1cm,xunit=3cm}
\begin{pspicture}(-0.3,-0.5)(5,4.75)
\psaxes[xLabel={\scriptsize\itshape},yLabel={\sffamily\footnotesize},%
   Dy=0.5, Dx=0.25]{->}(0,0)(4.5,4.5)
\end{pspicture}
\end{showexample}



%--------------------------------------------------------------------------------------
\subsection{\texttt{tickstyle}}
%--------------------------------------------------------------------------------------
Syntax:
\begin{verbatim}
tickstyle=full|bottom|top
\end{verbatim}

This option is already in the \verb+pst-plot+ package and only mentioned here for
some completness.

\begin{showexample}[width=3.5cm]
\begin{pspicture}(-1,-1)(2,2)
\psaxes[tickstyle=bottom,subticks=5]{->}(0,0)(-1,-1)(2,2)
\end{pspicture}\\[0.5cm]
%
\begin{pspicture}(-1,-1)(2,2)
\psaxes[tickstyle=bottom,subticks=5]{->}(0,0)(2,2)(-1,-1)
\end{pspicture}
\end{showexample}

\begin{showexample}[width=3.5cm]
\begin{pspicture}(-1,-1)(2,2)
\psaxes[tickstyle=top,subticks=5]{->}(0,0)(-1,-1)(2,2)
\end{pspicture}\\[0.5cm]
%
\begin{pspicture}(-1,-1)(2,2)
\psaxes[tickstyle=top,subticks=5]{->}(0,0)(2,2)(-1,-1)
\end{pspicture}
\end{showexample}


The \verb+tickstyle+ option changes the position of the labels by default.
If you want the labels on the other side of an axis, then use the options
\verb+labelsep+ or set the ticks with \verb+ticksize+.

%--------------------------------------------------------------------------------------
\subsection{\texttt{ticks}}
%--------------------------------------------------------------------------------------
Syntax:
\begin{verbatim}
ticks=all|x|y|none
\end{verbatim}

This option is also already in the \verb+pst-plot+ package and only mentioned here for
some completness.

\begin{showexample}[width=3.5cm]
\psset{ticksize=6pt}
\begin{pspicture}(-1,-1)(2,2)
\psaxes[ticks=all,subticks=5]{->}(0,0)(-1,-1)(2,2)
\end{pspicture}
\end{showexample}

\begin{showexample}[width=3.5cm]
\begin{pspicture}(-1,-1)(2,2)
\psaxes[ticks=y,subticks=5]{->}(0,0)(-1,-1)(2,2)
\end{pspicture}
\end{showexample}

\begin{showexample}[width=3.5cm]
\begin{pspicture}(-1,-1)(2,2)
\psaxes[ticks=x,subticks=5]{->}(0,0)(2,2)(-1,-1)
\end{pspicture}
\end{showexample}

\begin{showexample}[width=3.5cm]
\begin{pspicture}(-1,-1)(2,2)
\psaxes[ticks=none,subticks=5]{->}(0,0)(2,2)(-1,-1)
\end{pspicture}
\end{showexample}


%--------------------------------------------------------------------------------------
\subsection{\texttt{labels}}
%--------------------------------------------------------------------------------------
Syntax:
\begin{verbatim}
labels=all|x|y|none
\end{verbatim}

This option is also already in the \verb+pst-plot+ package and only mentioned here for
some completness.

\begin{showexample}[width=3.5cm]
\psset{ticksize=6pt}
\begin{pspicture}(-1,-1)(2,2)
\psaxes[labels=all,subticks=5]{->}(0,0)(-1,-1)(2,2)
\end{pspicture}
\end{showexample}

\begin{showexample}[width=3.5cm]
\begin{pspicture}(-1,-1)(2,2)
\psaxes[labels=y,subticks=5]{->}(0,0)(-1,-1)(2,2)
\end{pspicture}
\end{showexample}

\begin{showexample}[width=3.5cm]
\begin{pspicture}(-1,-1)(2,2)
\psaxes[labels=x,subticks=5]{->}(0,0)(2,2)(-1,-1)
\end{pspicture}
\end{showexample}

\begin{showexample}[width=3.5cm]
\begin{pspicture}(-1,-1)(2,2)
\psaxes[labels=none,subticks=5]{->}(0,0)(2,2)(-1,-1)
\end{pspicture}
\end{showexample}


%--------------------------------------------------------------------------------------
\subsection{Axis with trigonmetrical units}
%--------------------------------------------------------------------------------------
With the option \verb+trigLabels=true+ the labels on the x axis are trigonometrical ones:

\medskip
\begin{showexample}[pos=a]
\begin{pspicture}(-0.5,-1.25)(10,1.25)
  \psplot[linecolor=red,linewidth=1.5pt]%
     {0}{9.424777961}{x 180 mul 3.141592654 div sin}
  \psaxes[xunit=1.570796327,showorigin=false,trigLabels]{->}(0,0)(-0.5,-1.25)(6.4,1.25)
\end{pspicture}
\end{showexample}

With the value of \verb+xunit+ one can change the labels.

\medskip
\begin{showexample}[pos=a]
\begin{pspicture}(-0.5,-1.25)(10,1.25)
  \psplot[linecolor=red,linewidth=1.5pt]%
     {0}{9.424777961}{x 180 mul 3.141592654 div sin}
  \psaxes[xunit=0.7853981635,showorigin=false,trigLabels]{->}(0,0)(-1,-1.25)(12.8,1.25)
\end{pspicture}
\end{showexample}

\medskip
\begin{showexample}[pos=a]
\begin{pspicture}(-0.5,-1.25)(10,1.25)
  \psplot[linecolor=red,linewidth=1.5pt]%
     {0}{9.424777961}{x 180 mul 3.141592654 div sin}
  \psaxes[xunit=0.7853981635,showorigin=false,trigLabels,Dx=2]{->}(0,0)(-1,-1.25)(12.8,1.25)
\end{pspicture}
\end{showexample}

%--------------------------------------------------------------------------------------
\subsection{\texttt{ticksize}, \texttt{xticksize}, \texttt{yticksize}}
%--------------------------------------------------------------------------------------

Syntax:
\begin{verbatim}
ticksize=value[unit]
ticksize=value[unit] value[unit]
xticksize=value[unit]
xticksize=value[unit] value[unit]
yticksize=value[unit]
yticksize=value[unit] value[unit]
\end{verbatim}

\verb+ticksize+ sets both values.

\begin{showexample}[width=6cm]
\psset{arrowscale=3}
\begin{pspicture}(-1.5,-1.5)(4,3.5)
  \psaxes[ticksize=0.5cm]{->}(0,0)(-1.5,-1.5)(4,3.5)
\end{pspicture}
\end{showexample}

\begin{showexample}[width=6cm]
\psset{arrowscale=3}
\begin{pspicture}(-1.5,-1.5)(4,3.5)
  \psaxes[xticksize=-10pt 0,yticksize=0 10pt]{->}(0,0)(-1.5,-1.5)(4,3.5)
\end{pspicture}
\end{showexample}

A grid is also possible by setting the values to the max/min coordinates.

\begin{showexample}[width=6cm]
\psset{arrowscale=3}
\begin{pspicture}(-.5,-.5)(5,4.5)
  \psaxes[ticklinestyle=dashed,ticksize=0 4cm]{->}(0,0)(-.5,-.5)(5,4.5)
\end{pspicture}
\end{showexample}

%--------------------------------------------------------------------------------------
\subsection{\texttt{subticks}}
%--------------------------------------------------------------------------------------
Syntax:
\begin{verbatim}
subticks=<number>
\end{verbatim}

By default subticks cannot have labels.

\begin{showexample}[width=3.5cm]
\psset{ticksize=6pt}
\begin{pspicture}(-1,-1)(2,2)
\psaxes[ticks=all,subticks=5]{->}(0,0)(-1,-1)(2,2)
\end{pspicture}
\end{showexample}

\begin{showexample}[width=3.5cm]
\begin{pspicture}(-1,-1)(2,2)
\psaxes[ticks=y,subticks=5]{->}(0,0)(-1,-1)(2,2)
\end{pspicture}
\end{showexample}

\begin{showexample}[width=3.5cm]
\begin{pspicture}(-1,-1)(2,2)
\psaxes[ticks=x,subticks=5]{->}(0,0)(2,2)(-1,-1)
\end{pspicture}
\end{showexample}

\begin{showexample}[width=3.5cm]
\begin{pspicture}(-1,-1)(2,2)
\psaxes[ticks=none,subticks=5]{->}(0,0)(2,2)(-1,-1)
\end{pspicture}
\end{showexample}


%--------------------------------------------------------------------------------------
\subsection{\texttt{subticksize}, \texttt{xsubticksize}, \texttt{ysubticksize}}
%--------------------------------------------------------------------------------------

Syntax:
\begin{verbatim}
subticksize=value
xsubticksize=value
ysubticksize=value
\end{verbatim}

\verb+subticksize+ sets both values, which are relative to the ticksize length and 
can have any number. 1 sets it to the same length as the main ticks.

\begin{showexample}[pos=a]
\psset{yunit=1.5cm,xunit=3cm}
\begin{pspicture}(-1.25,-4.5)(3.25,.75)
  \psaxes[xticksize=-4 0.5,ticklinestyle=dashed,subticks=5,xsubticksize=1,%
     ysubticksize=0.75,xsubticklinestyle=dotted,xsubtickwidth=1pt,
     subtickcolor=gray]{->}(0,0)(-1,-4)(3.25,0.5)
\end{pspicture}
\end{showexample}

%--------------------------------------------------------------------------------------
\subsection{\texttt{tickcolor}, \texttt{subtickcolor}}
%--------------------------------------------------------------------------------------
Syntax:
\begin{verbatim}
tickcolor=<color>
xtickcolor=<color>
ytickcolor=<color>
subtickcolor=<color>
xsubtickcolor=<color>
ysubtickcolor=<color>
\end{verbatim}

\verb+tickcolor+ and \verb+subtickcolor+ set both for the x- and the y-Axis.

\begin{showexample}[pos=a]
\begin{pspicture}(0,-0.75)(10,1)
\psaxes[xLabel=\footnotesize,labelsep=2pt,yAxis=false,%
  labelsep=-10pt,ticksize=0 10mm,subticks=10,subticksize=0.75,%
  tickcolor=red,subtickcolor=blue,tickwidth=1pt,%
  subtickwidth=0.5pt](10.01,0)
\end{pspicture}
\end{showexample}

\begin{showexample}[width=5cm]
\begin{pspicture}(5,-0.75)(10,1)
\psaxes[xLabel=\footnotesize,labelsep=2pt,yAxis=false,%
  labelsep=5pt,ticksize=0 -10mm,subticks=10,subticksize=0.75,%
  tickcolor=red,subtickcolor=blue,tickwidth=1pt,%
  subtickwidth=0.5pt,Ox=5](5,0)(5,0)(10.01,0)
\end{pspicture}
\end{showexample}

%--------------------------------------------------------------------------------------
\subsection{\texttt{ticklinestyle} and \texttt{subticklinestyle}}
%--------------------------------------------------------------------------------------
Syntax:
\begin{verbatim}
ticklinestyle=solid|dashed|dotted|none
xticklinestyle=solid|dashed|dotted|none
yticklinestyle=solid|dashed|dotted|none
subticklinestyle=solid|dashed|dotted|none
xsubticklinestyle=solid|dashed|dotted|none
ysubticklinestyle=solid|dashed|dotted|none
\end{verbatim}

\verb+ticklinestyle+ and \verb+subticklinestyle+ set both values for the x and y axis. The
value \verb+none+ doesn't really makes sense, because it is the same to 
\verb+[sub]ticklines=0+

\begin{showexample}[pos=a]
\psset{unit=4cm}
\pspicture(-0.15,-0.15)(2.5,1)
  \psaxes[axesstyle=frame,logLines=y,xticksize=0 1,xsubticksize=1,%
    ylogBase=10,tickcolor=red,subtickcolor=blue,tickwidth=1pt,%
    subticks=20,xsubticks=10,xticklinestyle=dashed,%
    xsubticklinestyle=dashed](2.5,1)
\endpspicture
\end{showexample}


%--------------------------------------------------------------------------------------
\subsection{\texttt{loglines}}
%--------------------------------------------------------------------------------------
Syntax:
\begin{verbatim}
loglines=all|x|y
\end{verbatim}

\begin{showexample}[width=5.5cm]
\pspicture(0,-1)(5,5)
   \psaxes[subticks=5,axesstyle=frame,xylogBase=10,logLines=all](5,5)
\endpspicture
\end{showexample}

\begin{showexample}[pos=a]
\psset{unit=4cm}
\pspicture(-0.15,-0.15)(2.5,3)
  \psaxes[axesstyle=frame,logLines=y,xticksize=0 3,xsubticksize=1,%
    ylogBase=10,tickcolor=red,subtickcolor=blue,tickwidth=1pt,%
    subticks=20,xsubticks=10](2.5,3)
\endpspicture
\end{showexample}

\begin{showexample}[pos=a]
\psset{unit=4}
\pspicture(0,-0.3)(3,1.2)
   \psaxes[axesstyle=frame,logLines=x,xlogBase=10,Dy=0.5,%
     tickcolor=red,subtickcolor=blue,tickwidth=1pt,ysubticks=5,xsubticks=10](3,1)
\endpspicture
\end{showexample}

%--------------------------------------------------------------------------------------
\subsection{\texttt{xylogBase}, \texttt{xlogBase} and \texttt{ylogBase}}
%--------------------------------------------------------------------------------------
There are additional options \verb|xylogBase | xlogBase | ylogBase| to get one or both axes with logarithm labels.
For an intervall of [$10^{-3} ... 10^2$] choose a \verb|pstricks| intervall of [-3,2]. \verb|pstricks| takes $0$ as the origin of this axes, which is wrong
if we want to have a logarithm axes. With the options \verb|Oy| and \verb|Ox| we can set
the origin to $-3$, so that the first label gets $10^{-3}$. If this is not done by the
user then \verb|pstricks-add| does it by default. An alternative is to set these
parameters to empty values \verb|Ox={},Oy={}|, in this case \verb|pstricks-add| does nothing.


%------------------------------------------------------------------------------------
\subsubsection{\texttt{xylogBase}}
%------------------------------------------------------------------------------------
This mode is in math also called double logarithm. It is a combination of the two forgoing modes and the function is now $y=\log x$ and is shown in the following example.

\medskip
\begin{showexample}[width=7cm]
\begin{pspicture}(-3.5,-3.5)(3.5,3.5)
  \psplot[linewidth=2pt,linecolor=red]{0.001}{3}{x log}
  \psaxes[xylogBase=10,Oy=-3]{->}(-3,-3)(3.5,3.5)
  \uput[-90](3.5,-3){x}
  \uput[180](-3,3.5){y}
  \rput(2.5,1){$y=\log x$}
\end{pspicture}
\end{showexample}



%--------------------------------------------------------------------------------------------
\subsubsection{\texttt{ylogBase}}
%--------------------------------------------------------------------------------------------
The values for the \texttt{psaxes} y-coordinate are now the exponents to the base $10$ and for the right function to the base $e$: $10^{-3} \ldots 10^1$ which corresponds to the given y-intervall $-3\ldots 1.5$, where only integers as exponents are possible. These logarithm labels have no effect to the internal used units. To draw the logarithm function we have to use the math function
\[y=\log\{\log x\}\]
\[y=\ln\{\ln x\}\]
with an drawing intervall of $1.001\ldots 6$.

\medskip
\begin{showexample}[width=7cm]
\begin{pspicture}(-0.5,-3.5)(6.5,1.5)
  \psaxes[ylogBase=10]{->}(0,-3)(6.5,1.5)
  \uput[-90](6.5,-3){x}
  \uput[0](0,1.4){y}
  \rput(5,1){$y=\log x$}
  \psplot[linewidth=2pt,%
  plotpoints=100,linecolor=red]{1.001}{6}{x log log} % log(x)
\end{pspicture}
\end{showexample}

\medskip
\begin{showexample}[pos=a]
\begin{pspicture}(-0.5,-3.5)(6.5,1.5)
  \psplot[linewidth=2pt,plotpoints=100,linecolor=red]%
    {1.04}{6}{/ln {log 0.4343 div} def x ln ln} % log(x)
  \psaxes[ylogBase=e]{->}(0,-3)(6.5,1.5)
  \uput[-90](6.5,-3){x}
  \uput[0](0,1.5){y}
  \rput(5,1){$y=\ln x$}
\end{pspicture}
\end{showexample}



\medskip
\begin{showexample}[pos=a]
  \begin{pspicture}(-0.5,1.5)(6.5,4.5)
    \psaxes[ylogBase=10,Oy=2]{->}(0,2)(0,2)(6.5,4.5)
  \end{pspicture}
\end{showexample}



\medskip
\begin{showexample}[pos=a]
  \begin{pspicture}(-0.5,-0.5)(6.5,4.5)
    \psplot{0}{6}{x x cos add log}                       % x   + cox(x)
    \psplot[linecolor=red]{0}{6}{x 3 exp x cos add log}  % x^3 + cos(x)
    \psplot[linecolor=cyan]{0}{6}{x 5 exp x cos add log} % x^5 + cos(x)
    \psaxes[ylogBase=10]{->}(6.5,4.5)
  \end{pspicture}
\end{showexample}



\medskip
\begin{showexample}[pos=a]
\begin{pspicture}(-0.5,-1.5)(6.5,4.5)
  \psplot{0}{6}{x x cos add log}                       % x   + cox(x)
  \psplot[linecolor=red]{0}{6}{x 3 exp x cos add log}  % x^3 + cos(x)
  \psplot[linecolor=cyan]{0}{6}{x 5 exp x cos add log} % x^5 + cos(x)
  \psaxes[ylogBase=10]{->}(0,-1)(0,-1)(6.5,4.5)
\end{pspicture}
\end{showexample}



\medskip
\begin{showexample}[width=4cm]
\begin{pspicture}(2.5,1.75)(6.5,4.5)
  \psplot[linecolor=cyan]{3}{6}{x 5 exp x cos add log} % x^5 + cos(x)
  \psaxes[ylogBase=10,Ox=3,Oy=2]{->}(3,2)(3,2)(6.5,4.5)
\end{pspicture}
\end{showexample}




%--------------------------------------------------------------------------------------
\subsubsection{\texttt{xlogBase}}
%--------------------------------------------------------------------------------------
Now we have to use the easy math function $y=x$ because the x axis is still $\log x$.

\medskip
\begin{showexample}[width=7cm]
\begin{pspicture}(-3.5,-3.5)(3.5,3.5)
  \psplot[linewidth=2pt,linecolor=red]{-3}{3}{x} % log(x)
  \psplot[linewidth=2pt,linecolor=blue]{-1.3}{1.5}{x 0.4343 div} % ln(x)
  \psaxes[xlogBase=10,Oy=-3]{->}(-3,-3)(3.5,3.5)
  \uput[-90](3.5,-3){x}
  \uput[180](-3,3.5){y}
  \rput(2.5,1){$y=\log x$}
  \rput[lb](0,-1){$y=\ln x$}
\end{pspicture}
\end{showexample}


\begin{showexample}[pos=a]
\psset{yunit=3cm,xunit=2cm}
\begin{pspicture}(-1.25,-1.25)(4.25,1.5)
  \uput[-90](4.25,-1){x}
  \uput[0](-1,1){y}
  \rput(0,1){$y=\sin x$}
  \psplot[linewidth=2pt,plotpoints=5000,linecolor=red]{-1}{3.5}{10 x exp sin }
  \psaxes[xlogBase=10,Oy=-1]{->}(-1,-1)(4.25,1.25)
\end{pspicture}
\end{showexample}


\begin{showexample}[pos=a]
\begin{pspicture}(-3.5,-2.75)(3.5,2.5)
  \psaxes[xlogBase=10]{->}(0,0)(-3.5,-2.5)(3.5,2.5)
  \psplot{-2.5}{2.5}{10 x exp log}
\end{pspicture}
\end{showexample}



\medskip
\begin{showexample}[width=7cm]
\begin{pspicture}(-3.5,-2.75)(3.5,2.5)
  \psaxes[xlogBase=10,Ox={},Oy={}]{->}(0,0)(-3.5,-2.5)(3.5,2.5)
  \psplot{-2.5}{2.5}{10 x exp log}
\end{pspicture}
\end{showexample}


%------------------------------------------------------------------------------------
\subsubsection{No logstyle (\texttt{xylogBase=\{\}})}
%------------------------------------------------------------------------------------
This is only a demonstration that the default option \verb|logBase={}| still works ... :-)

\medskip
\begin{showexample}[width=7cm]
\begin{pspicture}(-3.5,-0.5)(3.5,2.5)
  \psplot[linewidth=2pt,linecolor=red,xylogBase={}]{0.5}{3}{x log} % log(x)
  \psaxes{->}(0,0)(-3.5,0)(3.5,2.5)
  \uput[-90](3.5,0){x}
  \uput[180](0,2.5){y}
  \rput(2.5,1){$y=\log x$}
\end{pspicture}
\end{showexample}



%--------------------------------------------------------------------------------------
\subsection{\texttt{subticks}, \texttt{tickwidth} and \texttt{subtickwidth}}
%--------------------------------------------------------------------------------------


\begin{center}
  \psset{arrowscale=3}
   \psaxes[xLabel=\footnotesize,labelsep=2pt,yAxis=false,subticks=8]{->}(0,0)(-5,-1)(5,1)\\[1cm]
   \psaxes[yAxis=false,subticks=4,tickstyle=bottom]{->}(0,0)(5,1)(-5,-1)\\
   \psaxes[yAxis=false,subticks=4,ticksize=-10pt 0]{->}(0,0)(-5,-5)(5,5)\\[1cm]
   \psaxes[yAxis=false,subticks=10,ticksize=0 -10pt,labelsep=15pt]{->}(0,0)(-5,-5)(5,5)\\[1cm]
   \psaxes[yAxis=false,subticks=4,ticksize=0 10pt,labelsep=-15pt]{->}(0,0)(5,5)(-5,-5)\\[1cm]
   \psaxes[yAxis=false,subticks=4,ticksize=0 -10pt]{->}(0,0)(5,5)(-5,-5)\\[0.25cm]
   \psaxes[yAxis=false,subticks=0]{->}(0,0)(-5,-5)(5,5)\\[1cm]
   \psaxes[yAxis=false,subticks=0,tickcolor=red,linecolor=blue]{->}(0,0)(5,5)(-5,-5)\\
   \psaxes[yAxis=false,subticks=5,tickwidth=2pt,subtickwidth=1pt]{->}(0,0)(-5,-5)(5,5)\\[1cm]
   \psaxes[yAxis=false,subticks=0,tickcolor=red]{->}(0,0)(5,5)(-5,-5)
\end{center}

\begin{lstlisting}
  \psset{arrowscale=3}
   \psaxes[xLabel=\footnotesize,labelsep=2pt,yAxis=false,subticks=8]{->}(0,0)(-5,-1)(5,1)\\[1cm]
   \psaxes[yAxis=false,subticks=4,tickstyle=bottom]{->}(0,0)(5,1)(-5,-1)\\
   \psaxes[yAxis=false,subticks=4,ticksize=-10pt 0]{->}(0,0)(-5,-5)(5,5)\\[1cm]
   \psaxes[yAxis=false,subticks=10,ticksize=0 -10pt,labelsep=15pt]{->}(0,0)(-5,-5)(5,5)\\[1cm]
   \psaxes[yAxis=false,subticks=4,ticksize=0 10pt,labelsep=-15pt]{->}(0,0)(5,5)(-5,-5)\\[1cm]
   \psaxes[yAxis=false,subticks=4,ticksize=0 -10pt]{->}(0,0)(5,5)(-5,-5)\\[0.25cm]
   \psaxes[yAxis=false,subticks=0]{->}(0,0)(-5,-5)(5,5)\\[1cm]
   \psaxes[yAxis=false,subticks=0,tickcolor=red,linecolor=blue]{->}(0,0)(5,5)(-5,-5)\\
   \psaxes[yAxis=false,subticks=5,tickwidth=2pt,subtickwidth=1pt]{->}(0,0)(-5,-5)(5,5)\\[1cm]
   \psaxes[yAxis=false,subticks=0,tickcolor=red]{->}(0,0)(5,5)(-5,-5)
\end{lstlisting}
   
\vspace*{4.5cm}
\begin{center}
  \psset{arrowscale=3}
  \psaxes[xAxis=false,subticks=8]{->}(0,0)(-5,-5)(5,5)\hspace{2em}
  \psaxes[xAxis=false,subticks=4]{->}(0,0)(5,5)(-5,-5)\hspace{4em}
  \psaxes[xAxis=false,subticks=4,tickstyle=top]{->}(0,0)(-5,-5)(5,5)\hspace{3em}
  \psaxes[xAxis=false,subticks=4,tickstyle=bottom]{->}(0,0)(-5,-5)(5,5)\hspace{1em}
  \psaxes[xAxis=false,subticks=4,tickstyle=top]{->}(0,0)(5,5)(-5,-5)\hspace{2em}
  \psaxes[xAxis=false,subticks=4,tickstyle=bottom,linecolor=red]{->}(0,0)(5,5)(-5,-5)\hspace{4em}
  \psaxes[xAxis=false,subticks=0]{->}(0,0)(-5,-5)(5,5)\hspace{1em}
  \psaxes[xAxis=false,subticks=0,tickcolor=red,linecolor=blue]{->}(0,0)(5,5)(-5,-5)\hspace{4em}
  \psaxes[xAxis=false,subticks=5,tickwidth=2pt,subtickwidth=1pt]{->}(0,0)(-5,-5)(5,5)\hspace{2em}
  \psaxes[xAxis=false,subticks=5,tickcolor=red,tickwidth=2pt,%
      ticksize=10pt,subtickcolor=blue,subticksize=0.75]{->}(0,0)(5,5)(-5,-5)
\end{center}

\vspace*{5cm}
   \begin{lstlisting}
  \psset{arrowscale=3}
  \psaxes[xAxis=false,subticks=8]{->}(0,0)(-5,-5)(5,5)\hspace{2em}
  \psaxes[xAxis=false,subticks=4]{->}(0,0)(5,5)(-5,-5)\hspace{4em}
  \psaxes[xAxis=false,subticks=4,tickstyle=top]{->}(0,0)(-5,-5)(5,5)\hspace{3em}
  \psaxes[xAxis=false,subticks=4,tickstyle=bottom]{->}(0,0)(-5,-5)(5,5)\hspace{1em}
  \psaxes[xAxis=false,subticks=4,tickstyle=top]{->}(0,0)(5,5)(-5,-5)\hspace{2em}
  \psaxes[xAxis=false,subticks=4,tickstyle=bottom,linecolor=red]{->}(0,0)(5,5)(-5,-5)\hspace{4em}
  \psaxes[xAxis=false,subticks=0]{->}(0,0)(-5,-5)(5,5)\hspace{1em}
  \psaxes[xAxis=false,subticks=0,tickcolor=red,linecolor=blue]{->}(0,0)(5,5)(-5,-5)\hspace{4em}
  \psaxes[xAxis=false,subticks=5,tickwidth=2pt,subtickwidth=1pt]{->}(0,0)(-5,-5)(5,5)\hspace{2em}
  \psaxes[xAxis=false,subticks=5,tickcolor=red,tickwidth=2pt,%
     ticksize=10pt,subtickcolor=blue,subticksize=0.75]{->}(0,0)(5,5)(-5,-5)
\end{lstlisting}
   
\begin{showexample}[width=5.5cm]
\pspicture(5,5.5)
\psaxes[subticks=4,ticksize=6pt,subticksize=0.5,%
   tickcolor=red,subtickcolor=blue]{->}(5.4,5)
\endpspicture
\end{showexample}

\begin{showexample}[width=5.5cm]
\pspicture(5,5.5)
   \psaxes[subticks=5,ticksize=6pt,subticksize=0.5,tickstyle=top]{->}(5.4,5)
\endpspicture
\end{showexample}

\begin{showexample}[width=5.5cm]
\pspicture(5,5.5)
   \psaxes[subticks=5,ticksize=6pt,subticksize=0.5,tickstyle=bottom]{->}(5.4,5)
\endpspicture
\end{showexample}

\begin{showexample}[width=6.5cm]
\pspicture(-3,-3)(3,3.5)
   \psaxes[subticks=5,ticksize=6pt,subticksize=0.5,tickstyle=top]{->}(0,0)(3,3)(-3,-3)
\endpspicture
\end{showexample}

\begin{showexample}[width=6.5cm]
\pspicture(0,0.5)(-3,-3)
   \psaxes[subticks=5,ticksize=6pt,subticksize=0.5,%
      tickstyle=bottom,linecolor=red]{->}(-3,-3)
\endpspicture
\end{showexample}



\begin{showexample}[width=5.5cm]
\psset{axesstyle=frame}
\pspicture(5,5.5)
   \psaxes[subticks=4,tickcolor=red,subtickcolor=blue](5,5)
\endpspicture
\end{showexample}

\vspace{1cm}
\begin{showexample}[width=5.5cm]
\pspicture(5,5.5)
   \psaxes[subticks=5,subticksize=1,subtickcolor=lightgray](5,5)
\endpspicture
\end{showexample}

\begin{showexample}[width=5.5cm]
\pspicture(5,5.5)
   \psaxes[subticks=2,subticksize=1,subtickcolor=lightgray](5,5)
\endpspicture
\end{showexample}

\begin{showexample}[width=3.5cm]
\pspicture(3,4.5)
   \psaxes[subticks=5,ticksize=-7pt 0](3,4)
\endpspicture
\end{showexample}


\begin{showexample}[width=3.75cm]
\pspicture(0,1)(-3,-4)
   \psaxes[subticks=5](-3,-4)
\endpspicture
\end{showexample}

\begin{showexample}[width=3.5cm]
\pspicture(3,4.5)
   \psaxes[axesstyle=axes,subticks=5](3,4)
\endpspicture
\end{showexample}

\begin{showexample}[width=4cm]
\pspicture(0,1)(-3,-4)
  \psaxes[axesstyle=axes,subticks=5,%
    ticksize=0 10pt,labelsep=13pt](-3,-4)
\endpspicture
\end{showexample}


%--------------------------------------------------------------------------------------
\subsection{\texttt{xlabelFactor} and \texttt{ylabelFactor}}
%--------------------------------------------------------------------------------------
When having big numbers as data records then it makes sense to write the values
as ${<number>\cdot 10^{<exp>}}$. These new options allow to define the additional part
of the value.

\resetPSTPlotOptions
\begin{showexample}[pos=a]
\readdata{\data}{demo1.dat}
\pstScalePoints(1,0.000001){}{}% (x,y){additional x operator}{y op}
\psset{llx=-1cm,lly=-1cm}
\psgraph[ylabelFactor={\cdot 10^6},Dx=5,Dy=100](0,0)(25,750){8cm}{5cm} 
   \listplot[linecolor=red, linewidth=2pt, showpoints=true]{\data}
\endpsgraph
\pstScalePoints(1,1){}{}% reset
\end{showexample}


%--------------------------------------------------------------------------------------
\subsection{Plot style \texttt{bar} and option \texttt{barwidth}}
%--------------------------------------------------------------------------------------
This option allows to draw bars for the data records. The width of the bars
is controlled by the option \verb+barwidth+, which is set by default to
value of \verb+0.25cm+, which is the total width.

\def\barData{
0 0.03
1 0.11
2 0.28
3 0.84
4 6.70
5 8.55
6 8.77
7 11.09
8 7.18
9 6.20
10 5.78
11 4.19
12 2.37
13 2.26
14 1.68
15 1.03
16 1.37
17 1.34
18 0.92
19 0.67
20 0.87
21 1.20
22 1.98
23 3.99
24 5.08
25 5.17
26 5.78
27 4.44
28 0.11 
}

\begin{showexample}[pos=a]
\psset{xunit=.44cm,yunit=.3cm}
\begin{pspicture}(-2,-1.75)(29,13)
  \psaxes[axesstyle=axes,Ox=1466,Oy=0,Dx=4,Dy=2,%
     ylabelFactor={\,\%}]{-}(29,12)
  \listplot[shadow=true,linecolor=blue,plotstyle=bar,barwidth=0.3cm,
     fillcolor=red,fillstyle=solid]{\barData}
  \rput{90}(-3,6.25){Amount}
\end{pspicture}
\end{showexample}

\begin{showexample}[pos=a]
\psset{xunit=.44cm,yunit=.3cm}
\begin{pspicture}(-2,-1.75)(29,13)
  \psaxes[axesstyle=axes,Ox=1466,Oy=0,Dx=4,Dy=2,%
     ylabelFactor={\,\%}]{-}(29,12)
  \listplot[linecolor=blue,plotstyle=bar,barwidth=0.3cm,
     fillcolor=red,fillstyle=crosshatch]{\barData}
  \rput{90}(-3,6.25){Amount}
\end{pspicture}
\end{showexample}

\begin{showexample}[pos=a]
\psset{xunit=.44cm,yunit=.3cm}
\begin{pspicture}(-2,-1.75)(29,13)
  \psaxes[axesstyle=axes,Ox=1466,Oy=0,Dx=4,Dy=2,%
     ylabelFactor={\,\%}]{-}(29,12)
  \listplot[linecolor=blue,plotstyle=bar,barwidth=0.3cm,
     fillcolor=red,fillstyle=vlines]{\barData}
  \listplot[showpoints=true]{\barData}
  \rput{90}(-3,6.25){Amount}
\end{pspicture}
\end{showexample}



%------------------------------------------------------------------------------------
\subsection{New options for \texttt{\textbackslash readdata}}
%------------------------------------------------------------------------------------
By default the macros \verb|\readdata| reads every
data record, which could be annoying when there are more than 10000 records to read.
The package \verb|pst-plot-add| defines an additional key \verb|nStep|, which allows
to read only a selected part of the data records,
e.g. \verb|nStep=10|, only every 10\textsuperscript{th} records is saved.

\begin{lstlisting}
\readdata[nStep=10]{\dataA}{stressrawdata.dat}
\end{lstlisting}

The default value for \verb+nStep+ is $1$.

%--------------------------------------------------------------------------------------
\section{New options for \texttt{\textbackslash listplot}}
%--------------------------------------------------------------------------------------
By default the plot macros \verb|\dataplot|, \verb|\fileplot| and \verb|\listplot| plot every
data record. The package \verb|pst-plot-add| defines additional keys \verb|nStep, nStart, nEnd| and \verb|xStep, xStart, xEnd|, which allows
to plot only a selected part of the data records, e.g. \verb|nStep=10|. These "n" options mark the number of the record to be plot ($0,1,2,...$) and the "x" ones the x-values of the data records.


\begin{center}
\begin{tabular}{>{\small\ttfamily}l|l}
Name & Default setting\\\hline
ignoreLines          &  $0$\\
nStart &  $1$\\
nEnd   &  \verb|{}|\\
nStep  &  $1$\\
xStart &  \verb|{}|\\
xEnd   &  \verb|{}|\\
yStart &  \verb|{}|\\
yEnd   &  \verb|{}|\\
xStep  &  $0$\\
plotNo  &  $1$\\
plotNoMax|  &  $1$
\end{tabular}
\end{center}

Except the \verb+ignoreLines+ options all new ones are only available
for the \verb|\listplot| macro, which is not a real limitation, because all data records can be read
from a file with the \verb|\readdata| macro (see example files or \cite{dtk02.2:jackson.voss:plot-funktionen}):
\begin{verbatim}
\readdata[ignoreLines=2]{\data}{/home/voss/data/data1.dat}
\readdata[nStep=10]{\data}{/home/voss/data/data1.dat}
\end{verbatim}


%--------------------------------------------------------------------------------------
\subsection{Example for \texttt{ignoreLines}}
%--------------------------------------------------------------------------------------
By default the macro \verb|\readdata| reads every
data record line, which could be annoying when you have some text lines at top of your
data files or when there are more than 10000 records to read. 

The default value for  \verb+ignoreLines+ is $0$.
The following data file has two text lines which shall be ignored by the \verb+\readdata+ macro.
Without the \verb+ignoreLine+ option there will be an error when reading this data file.

\begin{showexample}[width=4cm]
\begin{filecontents*}{pstricks-add-data9.dat}
some nonsense in this line ������
time forcex forcey
0 0.2
1 1
2 4
\end{filecontents*}
\readdata[ignoreLines=2]{\data}{pstricks-add-data9.dat}
\pspicture(2,4)
  \listplot[showpoints=true]{\data}
  \psaxes{->}(2,4)
\endpspicture
\end{showexample}



%--------------------------------------------------------------------------------------
\subsection{Example for \texttt{nStep/xStep}}
%--------------------------------------------------------------------------------------

The use \verb|nStep| and \verb|xStep| options make only real sense when also using the
option \verb|plotstyle=dots|. Otherwise the coordinates are connected by a line as usual. Also the \verb|xStep| option needs increasing x values.
Pay attention that \verb+nStep+ can be used for \verb+\readdata+ and for \verb+\listplot+. If used
in both macros than the effect is multiplied, e.g. \verb+\readdata+ with \verb+nStep=5+ and
\verb+\listplot+ with \verb+nStep=10+ means, that only every 50\textsuperscript{th} data records
is read and plotted.

When both, \verb|x/yStart/End| are defined then the values are also compared with
both values.


The datafile \verb|data.dat| contains $1000$ data records. The thin blue line is the plot
of all records with the plotstyle option \verb|curve|.
\begin{showexample}[pos=a]
\readdata{\data}{examples/data.dat}
\psset{xunit=0.125mm,yunit=0.00015mm}
\begin{pspicture}(-80,-30000)(1000,310000)
\psaxes[axesstyle=frame,Dx=100,dx=100,Dy=50000,dy=50000](1000,300000)
\listplot[nStep=50,linewidth=3pt,linecolor=red,plotstyle=dots]{\data}
\listplot[linewidth=1pt,linecolor=blue]{\data}
\end{pspicture}
\end{showexample}



%--------------------------------------------------------------------------------------
\subsection{Example for \texttt{nStart/xStart}}
%--------------------------------------------------------------------------------------
\begin{showexample}[pos=a]
\readdata{\data}{examples/data.dat}
\psset{xunit=0.125mm,yunit=0.00015mm}
\begin{pspicture}(-80,-30000)(1000,310000)
\psaxes[axesstyle=frame,Dx=100,dx=100,Dy=50000,dy=50000](1000,300000)
\listplot[nStart=200,linewidth=3pt,linecolor=blue]{\data}
\end{pspicture}
\end{showexample}

%--------------------------------------------------------------------------------------
\subsection{Example for \texttt{nEnd/xEnd}}
%--------------------------------------------------------------------------------------
\begin{showexample}[pos=a]
\readdata{\data}{examples/data.dat}
\psset{xunit=0.125mm,yunit=0.00015mm}
\begin{pspicture}(-80,-30000)(1000,310000)
\psaxes[axesstyle=frame,Dx=100,dx=100,Dy=50000,dy=50000](1000,300000)
\listplot[nEnd=800,linewidth=3pt,linecolor=blue]{\data}
\end{pspicture}
\end{showexample}


%--------------------------------------------------------------------------------------
\subsection{Example for all new options}
%--------------------------------------------------------------------------------------
\begin{showexample}[pos=a]
\readdata{\data}{examples/data.dat}
\psset{xunit=0.125mm,yunit=0.00015mm}
\begin{pspicture}(-80,-30000)(1000,310000)
\psaxes[axesstyle=frame,Dx=100,dx=100,Dy=50000,dy=50000](1000,300000)
\listplot[nStart=200, nEnd=800, nStep=50,linewidth=3pt,linecolor=blue,%
	plotstyle=dots]{\data}
\end{pspicture}
\end{showexample}


%--------------------------------------------------------------------------------------
\subsection{Example for \texttt{xStart}}
%--------------------------------------------------------------------------------------

This example shows the use of the same plot with different units and different
\verb|xStart| value. The blue curve is the original plot of the data records.
To show the important part of the curve there is another one plotted with a
greater \verb|yunit| and a start value of \verb|xStart=0.35|. This makes it
possible to have a kind of a zoom to the original graphic.

\begin{showexample}[pos=a]
\psset{xunit=10cm, yunit=0.01cm,xLabel={\scriptsize\sffamily},yLabel={\scriptsize\sffamily}}
\readdata{\data}{examples/data3.dat}
\begin{pspicture}(-0.1,-100)(1.5,700.0)
  \psaxes[Dx=0.25,Dy=100,dy=100\psyunit,tickstyle=bottom]{->}(0,0)(0,-100)(1.4,520)
  \uput[0](1.4,0){\textsf{t [s]}}
  \rput(-0.125,200){\rotateleft{\small\sffamily flow [ml/s]}}
  \listplot[linewidth=2pt, linecolor=blue]{\data}
  \rput(0.4,300){
    \pscustom[yunit=0.04cm, linewidth=1pt]{%
      \listplot[xStart=0.355]{\data}
      \psline(1,-2.57)(1,0)(0.355,0)
      \fill[fillstyle=hlines,fillcolor=gray,hatchwidth=0.4pt,hatchsep=1.5pt,hatchcolor=red]%
      \psline[linewidth=0.5pt]{->}(0.7,0)(1.05,0)
    }%
  }
  \psline[linewidth=.01]{->}(0.75,300)(0.4,20)
  \psline[linewidth=.01]{->}(1,290)(1.1,440)
  \rput(1.1,470){\footnotesize\sffamily leak volume}
  \psline[linewidth=.01]{->}(0.78,200)(1,100)
  \rput[l](1.02,100){\footnotesize\sffamily closing volume}
\end{pspicture}
\end{showexample}



%--------------------------------------------------------------------------------------
\subsection{Example for \texttt{yStart}/\texttt{yEnd}}
%--------------------------------------------------------------------------------------

\begin{showexample}[pos=a]
\readdata{\data}{examples/data.dat}
\psset{xunit=0.125mm,yunit=0.00015mm}
\begin{pspicture}(-80,-30000)(1000,310000)
  \psaxes[axesstyle=frame,Dx=100,dx=100,Dy=50000,dy=50000](1000,300000)
  \psset{linewidth=0.1pt, linestyle=dashed,linecolor=red}
  \psline(0,40000)(1000,40000)
  \psline(0,175000)(1000,175000)
  \listplot[yStart=40000, yEnd=175000,linewidth=3pt,linecolor=blue,plotstyle=dots]{\data}
\end{pspicture}
\end{showexample}



%--------------------------------------------------------------------------------------
\subsection{Example for \texttt{plotNo/plotNoMax}}
%--------------------------------------------------------------------------------------
By default the plot macros expect \verb+x|y+ data records, but
when having data files with multiple values for y, like:
\begin{verbatim}
x y1 y2 y3 y4 ... yMax
x y1 y2 y3 y4 ... yMax
...
\end{verbatim}

you can select the y value which should be plotted. The option \verb+plotNo+ marks the plotted
value (default $1$) and the option \verb+plotNoMax+ tells \verb+pst-plot+ how many $y$ values are
present. There are no real restrictions in the maximum number for \verb+plotNoMax+.

We have the following data file:
\begin{verbatim}
[% file examples/data.dat
0    0    3.375    0.0625
10    5.375    7.1875    4.5
20    7.1875    8.375    6.25
30    5.75    7.75    6.6875
40    2.1875    5.75    5.9375
50    -1.9375    2.1875    4.3125
60    -5.125    -1.8125    0.875
70    -6.4375    -5.3125    -2.6875
80    -4.875    -7.1875    -4.875
90    0    -7.625    -5.625
100    5.5    -6.3125    -5.8125
110    6.8125    -2.75    -4.75
120    5.25    2.875    -0.75
]%
\end{verbatim}

\noindent which holds data records for multiple plots (\verb+x y1 y2 y3+). This can be plotted
without any modification to the data file:

\begin{showexample}[pos=a]
\readdata\Data{examples/dataMul.dat}
\psset{xunit=0.1cm, yunit=0.3cm}
\begin{pspicture}(0,-7.5)(150,10)
\psaxes[Dx=10,Dy=2.5]{->}(0,0)(0,-7.5)(150,7.5)
\psset{linewidth=2pt,plotstyle=line}
\listplot[linecolor=green,plotNo=1,plotNoMax=3]{\Data}
\listplot[linecolor=red,plotNo=2,plotNoMax=3]{\Data}
\listplot[linecolor=blue,plotNo=3,plotNoMax=3]{\Data}
\end{pspicture}
\end{showexample}



%--------------------------------------------------------------------------------------
\section{Polar plots}
%--------------------------------------------------------------------------------------

With the option \verb+polarplot=false|true+ it is possible to use \verb+\psplot+
in polar mode:
\begin{verbatim}
\psplot[polarplot=true,...]{<start angle>}{<end angle>}{<r(alpha)>}
\end{verbatim}

The equation in PostScript code is interpreted as a function $r=f(\alpha)$, e.g. for the
circle with radius 1 as $r=\sqrt{\sin^2x+\cos^2x}$:

\begin{verbatim}
x sin dup mul x cos dup mul add sqrt 
\end{verbatim}


\resetPSTPlotOptions

\medskip
\begin{showexample}[pos=a]
\psset{plotpoints=200,unit=0.75}
\begin{pspicture}*(-5,-5)(3,3)
  \psaxes[labelsep=.75mm,xyLabel=\footnotesize,
    arrowlength=1.75,ticksize=2pt,%
    linewidth=0.17mm]{->}(0,0)(-4.99,-4.99)(3,3)
  \rput[Br](3,-.35){$x$}
  \rput[tr](-.15,3){$y$}
  \rput[Br](-.15,-.35){$0$}
  \psset{linewidth=.35mm,polarplot=true}
  \psplot[linecolor=red]{140}{310}{3 neg x sin mul x cos mul x sin 3 exp x cos 3 exp add div}
  \psplot[linecolor=cyan]{140}{310}{6 neg x sin mul x cos mul x sin 3 exp x cos 3 exp add div}
  \psplot[linecolor=blue]{140}{310}{9 neg x sin mul x cos mul x sin 3 exp x cos 3 exp add div}
\end{pspicture}
\end{showexample}


\medskip
\resetPSTPlotOptions
\begin{showexample}[pos=a]
\psset{plotpoints=200,unit=1}
\begin{pspicture}(-2.5,-2.5)(2.5,2.5)% Ulrich Dirr
 \psaxes[labelsep=.75mm,xyLabel=\footnotesize,%
  arrowlength=1.75, ticksize=2pt,linewidth=0.17mm]{->}(0,0)(-2.5,-2.5)(2.5,2.5)
  \rput[Br](2.5,-.35){$x$}
  \rput[tr](-.15,2.5){$y$}
  \rput[Br](-.15,-.35){$0$}
  \psset{linewidth=.35mm,plotstyle=curve,polarplot=true}
  \psplot[linecolor=red]{0}{360}{x cos 2 mul x sin mul}
  \psplot[linecolor=green]{0}{360}{x cos 3 mul x sin mul}
  \psplot[linecolor=blue]{0}{360}{x cos 4 mul x sin mul}
\end{pspicture}
\end{showexample}



\medskip
\begin{showexample}[pos=a]
\psset{plotpoints=200,unit=0.5}
\begin{pspicture}(-8.5,-8.5)(9,9)% Ulrich Dirr 
\psaxes[Dx=2,dx=2,Dy=2,dy=2,labelsep=.75mm,xyLabel=\footnotesize,%
  arrowlength=1.75,ticksize=2pt,linewidth=0.17mm]{->}(0,0)(-8.5,-8.5)(9,9)
\rput[Br](9,-.7){$x$}
\rput[tr](-.3,9){$y$}
\rput[Br](-.3,-.7){$0$}
%
\psset{linewidth=.35mm,plotstyle=curve,polarplot=true}
\psplot[linecolor=blue]{0}{720}{8 2.5 x mul sin mul}
\end{pspicture}
\end{showexample}






%--------------------------------------------------------------------------------------
\section{New commands and environments}
%--------------------------------------------------------------------------------------
%--------------------------------------------------------------------------------------
\subsection{\CMD{pstScalePoints}}
%--------------------------------------------------------------------------------------
The syntax is
\begin{verbatim}
\pstScalePoints(xScale,xScale){xPS}{yPS}
\end{verbatim}

\verb+xScale,yScale+ are decimal values as scaling factors, the \verb+xPs+ and \verb+yPS+
are additional PostScript code to the x- and y-values of the data records. This macro
is only valid for the \CMD{listplot} macro! 

\resetPSTPlotOptions
\begin{showexample}[width=6cm]
\def\data{0 0
1 3
2 4
3 1
4 2 
5 3 
6 6}
\begin{pspicture}(6,6)
\psaxes{->}(6,6)
\listplot[showpoints=true,linecolor=red]{\data}
\pstScalePoints(1,0.5){}{3 add}
\listplot[showpoints=true,linecolor=blue]{\data}
\end{pspicture}
\end{showexample}

\bigskip

Changes with \verb+\pstScalePoints+ are always global to all following \verb+\listplot+
macros. This is the reason why it is a good idea to reset the values at the end of the
\verb+pspicture+ environment.

\begin{verbatim}
\pstScalePoints(1,1){}{}
\end{verbatim}

%--------------------------------------------------------------------------------------
\subsection{\texttt{psgraph} environment}
%--------------------------------------------------------------------------------------
This new environment does the scaling, it expects as parameter the values (without units!) for the
internal coordinate system and the values of the physical width and height (with units!). The syntax is:

\begin{verbatim}
\psgraph[<options>](xMin,yMin)(xMax,yMax){xLength}{yLength}
...
\endpsgraph

\begin{psgraph}[<options>](xMin,yMin)(xMax,yMax){xLength}{yLength}
...
\end{psgraph}
\end{verbatim}

where the options are valid only for the the \verb+\psaxes+ macro. \TeX{} has problems with the division of very big
and very small values. This may cause some problems when using such values. The following example shows how
the data values can be scaled (by the macro \verb+\pstSclaePoints+) to get values for the \verb+\psgraph+ environment
which causes no division error.


\begin{showexample}[pos=a]
\readdata{\data}{demo1.dat}
\pstScalePoints(1,0.000001){}{}% (x,y){additional x operator}{y op}
\psset{llx=-0.5cm,lly=-1cm}
\psgraph[axesstyle=frame,xticksize=0 759,yticksize=0 25,%
    subticks=0,ylabelFactor={\cdot 10^6},%
    Dx=5,dy=100\psyunit,Dy=100](0,0)(25,750){12cm}{9cm}         % parameters
   \listplot[linecolor=red, linewidth=2pt, showpoints=true]{\data}
\endpsgraph
\end{showexample}


\begin{showexample}[width=7cm]
\readdata{\data}{demo1.dat}
\psset{xAxisLabel=x-Axes,yAxisLabel=y-Axes,llx=-1cm,%
   xAxisLabelPos={3cm,-1cm},yAxisLabelPos={-1.5cm,2.5cm}}
\pstScalePoints(1,0.00000001){}{}
\begin{psgraph}[axesstyle=frame,xticksize=0 7.5,yticksize=0 25,subticksize=1,%
     ylabelFactor={\cdot 10^8},Dx=5,Dy=1,xsubticks=2](0,0)(25,7.5){5.5cm}{5cm}
  \listplot[linecolor=red, linewidth=2pt, showpoints=true]{\data}
\end{psgraph}
\end{showexample}


\begin{showexample}[pos=a]
\readdata{\data}{demo1.dat}
\psset{llx=-0.5cm,lly=-1cm}
\pstScalePoints(1,0.000001){}{}
\psgraph[arrows=->,Dx=5,dy=200\psyunit,Dy=200,%
    subticks=5,ticksize=-10pt 0,tickwidth=0.5pt,%
    subtickwidth=0.1pt](0,0)(25,750){5.5cm}{5cm}
\listplot[linecolor=red,linewidth=2pt,showpoints=true,]{\data}
\endpsgraph
\end{showexample}


\begin{showexample}[pos=a]
\pstScalePoints(1,0.2){}{log}
\psset{lly=-0.75cm}
\psgraph[ylogBase=10,Dx=5,Dy=1,subticks=5](0,0)(25,2){12cm}{4cm}
  \listplot[linecolor=red, linewidth=2pt, showpoints=true]{\data}
\endpsgraph
\end{showexample}


\begin{showexample}[pos=a]
\readdata{\data}{demo0.dat}
\pstScalePoints(1,1){}{log}
\begin{psgraph}[arrows=->,Dx=0.5,ylogBase=10,Oy=-1,xsubticks=10,%
     ysubticks=2](0,-3)(3,1){12cm}{4cm}
  \listplot[linecolor=red, linewidth=2pt, showpoints=true]{\data}
\end{psgraph}
\end{showexample}


\begin{showexample}[width=6.5cm]
\readdata{\data}{demo0.dat}
\pstScalePoints(1,1){}{log}
\psgraph[arrows=->,Dx=0.5,ylogBase=10,Oy=-1,subticks=4](0,-3)(3,1){6cm}{3cm}
  \listplot[linecolor=red, linewidth=2pt, showpoints=true]{\data}
\endpsgraph
\end{showexample}


\begin{showexample}[pos=a]
\readdata{\data}{demo2.dat}%
\readdata{\dataII}{demo3.dat}%
\pstScalePoints(1,1){1989 sub}{}
\psset{llx=-0.5cm,lly=-1cm, xAxisLabel=Year,yAxisLabel=Whatever,%
     xAxisLabelPos={2in,-0.4in},yAxisLabelPos={-0.4in,1in}}
\psgraph[axesstyle=frame,Dx=2,Ox=1989,subticks=2](0,0)(12,6){4in}{2in}%
  \listplot[linecolor=red,linewidth=2pt]{\data}
  \listplot[linecolor=blue,linewidth=2pt]{\dataII}
  \listplot[linecolor=cyan,linewidth=2pt,yunit=0.5]{\dataII}
\endpsgraph
\end{showexample}

\begin{showexample}[pos=a]
\psset{llx=-0.5cm,lly=-0.75cm}
\pstScalePoints(1,1){1989 sub}{2 sub}
\begin{psgraph}[axesstyle=frame,Dx=2,Ox=1989,Oy=2,subticks=2](0,0)(12,4){6in}{3in}%
  \listplot[linecolor=red,linewidth=2pt]{\data}
  \listplot[linecolor=blue,linewidth=2pt]{\dataII}
  \listplot[linecolor=cyan,linewidth=2pt,yunit=0.5]{\dataII}
\end{psgraph}
\end{showexample}

%\newpage
An example with ticks on every side of the frame:

\begin{showexample}[pos=a]
\def\data{0 0 1 1 2 4 3 9}
\begin{psgraph}[axesstyle=frame,tickstyle=top](0,-0.5)(3.0,9.0){12cm}{5cm}
  \psaxes[axesstyle=frame,labels=none,tickstyle=bottom](3,9)(0,0)(3,9)
  \listplot[linecolor=red,linewidth=2pt]{\data}
\end{psgraph}
\end{showexample}


%-------------------------------------------------------------------------------------------
\subsubsection{The new options}
%-------------------------------------------------------------------------------------------

\begin{center}
\begin{tabular}{>{\tt}l>{\tt}ll}
\textrm{name} & \textrm{default} & meaning\\\hline
xAxisLabel & x & label for the x-axis\\
yAxisLabel & y & label for the y-axis\\
xAxisLabelPos & \{\} & where to put the x-label\\
yAxisLabelPos & \{\} & where to put the y-label\\
llx & 0pt & trim for the lower left x\\
lly & 0pt & trim for the lower left y\\
urx & 0pt & trim for the upper right x\\
ury & 0pt & trim for the upper right y
\end{tabular}
\end{center}

There is one restriction in using the trim parameters, they must been set
\textbf{before} \verb+psgraph+ is called. They are senseless, when using
as parameters of \verb+psgraph+ itself.

\medskip
\resetPSTPlotOptions

\begin{showexample}[pos=a,wide=false]
\psset{llx=-1cm,lly=-1.25cm,urx=0.5cm,ury=0.1in,xAxisLabel=Year,%
   yAxisLabel=Whatever,xAxisLabelPos={.4\linewidth,-0.4in},%
   yAxisLabelPos={-0.4in,2in}}
\pstScalePoints(1,1){1989 sub}{}
\psframebox[linestyle=dashed,boxsep=0pt]{%
\begin{psgraph}[axesstyle=frame,Ox=1989,subticks=2](0,0)(12,6){0.8\linewidth}{4in}%
  \listplot[linecolor=red,linewidth=2pt]{\data}%
  \listplot[linecolor=blue,linewidth=2pt]{\dataII}%
  \listplot[linecolor=cyan,linewidth=2pt,yunit=0.5]{\dataII}%
\end{psgraph}%
}
\end{showexample}
\pstScalePoints(1,1){}{}% reset




\vspace{0.5cm}
%--------------------------------------------------------------------------------------
\subsection{\CMD{resetPSTPlotOptions}}
%--------------------------------------------------------------------------------------

Sometimes it is difficult to know what options which are changed inside a long document
are different to the default one. With this
macro all options depending to \verb+pst-plot+ can be reset. This depends to all
options of the package \verb+pst-plot+.

\begin{lstlisting}
\def\resetPSTPlotOptions{%
  \@zero=0%
  \pstScalePoints(1,1){}{}% reset
  \psset{%
       infix=none,%
       xyAxes=true,
       xyDecimals={},%
       xyLabel={},
       xylogBase={},
       logLines=none,
       xlabelFactor=\relax,ylabelFactor=\relax,
       nStep=1,nStart=0,nEnd={},%
       xStep=0,yStep=0,xStart={},xEnd={},yStart={},yEnd={},comma=false,%
       plotNo=1,plotNoMax=1,
       xAxisLabel=x,yAxisLabel=y,
       yAxisLabelPos=\@empty,xAxisLabelPos=\@empty,
       plotstyle=line,
       plotpoints=50,
       polarplot=false,
       method=default, whichabs=default, whichord=default,%
       plotfuncx=default, plotfuncy=default, buildvector=false, 
       VarStep=default,
       dimen=middle,% remark of ML
       ticks=all,
       labels=all,
       Ox=0,Dx=1,dx=0,Oy=0,Dy=1,dy=0,
       showorigin=true,
       xticksize=-4pt 4pt,yticksize=-4pt 4pt,
       tickstyle=full,
       subticksize=0.75,subticks=1,tickcolor=black,ticklinestyle=solid,%
       subticklinestyle=solid,%
       subtickcolor=gray,%
       tickwidth=0.5\pslinewidth,%
       subtickwidth=0.25\pslinewidth,
       axesstyle=axes,
       barwidth=0.25cm,
       xAxisLabel=x,yAxisLabel=y,
       yAxisLabelPos=\@empty,xAxisLabelPos=\@empty,
       llx=\z@, lly=\z@, urx=\z@, ury=\z@}% prevents rounding errors 
}
\end{lstlisting}

%--------------------------------------------------------------------------------------
\section{Credits}
%--------------------------------------------------------------------------------------
%{Hendri Adriaans | } 
%{Ulrich Dirr | } 
%{Hubert G��lein |}
{Denis Girou | } 
%{Peter Hutnick | } 
%{Christophe Jorssen | } 
%{Manuel Luque | } 
%{Jens-Uwe Morawski |}
%{Tobias N�hring |}
%{Rolf Niepraschk |}
%{Dominique Rodriguez |}
{Timothy Van Zandt}



\nocite{*}
\bibliographystyle{plain}
\bibliography{pstricks}


