\title{The ''new`` \texttt{pstricks} 2005\thanks{%
    This document was written with \texttt{Kile: 1.7 (Qt: 3.1.1; KDE: 3.3;}
    \url{http://sourceforge.net/projects/kile/}) and the PDF output
    was build with VTeX/Free (\url{http://www.micropress-inc.com/linux})}\\
    \small v.\pstFV}
\author{Herbert Vo�}
\date{\today}

\maketitle

\begin{abstract}
This version of \verb+pstricks+ is the 2005-edition, which collects some new
macros and mostly all of the \verb+pstricks-add+ package.
 
\begin{itemize}
\item It is important to load \verb+pstricks+ as \textbf{first} PSTricks related package, otherwise
a lot of the macros won't work in the expected way. 
\item \verb+pstricks+ now uses the extended version of the keyval package. So be sure, that
you have installed \verb+pst-xkey+ which is part of the \verb+xkeyval+-package and that all
packages, that uses the old keyval interface are loaded \textbf{before} the \verb+xkeyval+.
\end{itemize}

\end{abstract}

\tableofcontents

\clearpage

%--------------------------------------------------------------------------------------
\section{\texttt{pspicture} environment}
%--------------------------------------------------------------------------------------
The old \PST version has an optional argument for the \texttt{pspicture}-environment which was
different to other ones, which always uses the xkeyval interface. To get the same behaviour
there is now the new option \texttt{shift} which has the same meaning as the old option.
This makes it possible to use more options for this environment. On the other hand, this is
not compatible to the old versions of \PST!



\begin{example}[pos=a]
\begin{center}
\textcolor{red}{\rule{5mm}{1pt}}%
\begin{pspicture}[shift=-0.5](-0.5,-0.5)(0.5,0.5)
  \psframe[linecolor=blue](-0.5,-0.5)(0.5,0.5)\rput(0,0){-0.5}
\end{pspicture}%
\textcolor{red}{\rule{5mm}{1pt}}
\hspace{1cm}%
\textcolor{red}{\rule{5mm}{1pt}}%
\begin{pspicture}(-0.5,-0.5)(0.5,0.5)
  \psframe[linecolor=blue](-0.5,-0.5)(0.5,0.5)\rput(0,0){0}
\end{pspicture}\textcolor{red}{\rule{5mm}{1pt}}
\hspace{1cm}%
\textcolor{red}{\rule{5mm}{1pt}}%
\begin{pspicture}[shift=0.5](-0.5,-0.5)(0.5,0.5)
  \psframe[linecolor=blue](-0.5,-0.5)(0.5,0.5)\rput(0,0){0.5}
\end{pspicture}%
\textcolor{red}{\rule{5mm}{1pt}}
\end{center}
\end{example}


\begin{example}[pos=a]
\textcolor{red}{\rule{5mm}{1pt}}
\begin{pspicture}[frame=true](-1,-0.5)(1.5,1)
  \psaxes[labels=none]{->}(0,0)(-1,-0.5)(1.5,1)
\end{pspicture}
\textcolor{red}{\rule{5mm}{1pt}}
\begin{pspicture}[shift=-0.75,frame=true](-1,-0.5)(1.5,1)
  \psaxes[labels=none]{->}(0,0)(-1,-0.5)(1.5,1)
\end{pspicture}
\textcolor{red}{\rule{5mm}{1pt}}
\newpsstyle{psframestyle}{linestyle=dotted,linecolor=blue}
\begin{pspicture}[shift=0.75,frame=true](-1,-0.5)(1.5,1)
  \psaxes[labels=none]{->}(0,0)(-1,-0.5)(1.5,1)
\end{pspicture}
\textcolor{red}{\rule{5mm}{1pt}}
\end{example}

%--------------------------------------------------------------------------------------
\section{Numeric functions}
%--------------------------------------------------------------------------------------

All macronames contain a \textat{} in their name, because they are only for internal use,
but it is no problem to use it as the other macros. One can define another name without 
a \textat{}:
\begin{verbatim}
\makeatletter
\let\pstdivide\pst@divide
\makeatother
\end{verbatim}

or put the macro inside of the \verb+\makeatletter+ -- \verb+\makeatother+ sequence. Nevertheless,
all these new numeric macros are primary for use in combination with the \PST related packages.

%--------------------------------------------------------------------------------------
\subsection{\CMD{pst\textat{}divide}}
%--------------------------------------------------------------------------------------

\verb+pstricks+ itself has its own divide macro, called \verb+\pst@divide+ which can divide two lengthes and saves the quotient as a floating point number:

\begin{verbatim}
\pst@divide{<dividend>}{<divisor>}{<result as a macro>}
\end{verbatim}

\begin{example}[width=2cm]
\makeatletter
\pst@divide{34pt}{6pt}\quotient \quotient\\
\pst@divide{-6pt}{34pt}\quotient \quotient
\makeatother
\end{example}

\noindent this gives the output $5.66666$. The result is not a length!

%--------------------------------------------------------------------------------------
\subsection{\CMD{pst\textat{}mod}}
%--------------------------------------------------------------------------------------
\verb+pstricks-add+ defines an additional numeric function for the modulus:

\begin{verbatim}
\pst@mod{<integer>}{<integer>}{<result as a macro>}
\end{verbatim}

\begin{example}[width=2cm]
\makeatletter
\pst@mod{34}{6}\modulo \modulo\\
\pst@mod{25}{-6}\modulo \modulo
\makeatother
\end{example}

\noindent this gives the output $4$. Using this internal numeric functions in documents
requires a setting inside the \verb+makeatletter+ and \verb+makeatother+ environment.
It makes some sense to define a new macroname in the preamble to use it throughou, e.g.
\verb+\let\modulo\pst@mod+.

%--------------------------------------------------------------------------------------
\subsection{\CMD{pst\textat{}max}}
%--------------------------------------------------------------------------------------

\begin{verbatim}
\pst@max{<integer>}{<integer>}{<result as count register>}
\end{verbatim}

\begin{example}[width=2cm]
\newcount\maxNo
\makeatletter
\pst@max{-34}{-6}\maxNo \the\maxNo\\
\pst@max{0}{11}\maxNo \the\maxNo
\makeatother
\end{example}


%--------------------------------------------------------------------------------------
\subsection{\CMD{pst\textat{}maxdim}}
%--------------------------------------------------------------------------------------

\begin{verbatim}
\pst@maxdim{<dimension>}{<dimension>}{<result as dimension register>}
\end{verbatim}

\begin{example}[width=2cm]
\newdimen\maxDim
\makeatletter
\pst@maxdim{34cm}{1234pt}\maxDim \the\maxDim\\
\pst@maxdim{34cm}{123pt}\maxDim \the\maxDim
\makeatother
\end{example}

%--------------------------------------------------------------------------------------
\subsection{\CMD{pst\textat{}abs}}
%--------------------------------------------------------------------------------------

\begin{verbatim}
\pst@abs{<integer>}{<result as a count register>}
\end{verbatim}

\begin{example}[width=2cm]
\newcount\absNo
\makeatletter
\pst@abs{-34}\absNo \the\absNo\\
\pst@abs{4}\absNo \the\absNo
\makeatother
\end{example}

%--------------------------------------------------------------------------------------
\subsection{\CMD{pst\textat{}absdim}}
%--------------------------------------------------------------------------------------

\begin{verbatim}
\pst@absdim{<dimension>}{<result as a dimension register>}
\end{verbatim}

\begin{example}[width=2cm]
\newdimen\absDim
\makeatletter
\pst@absdim{-34cm}\absDim \the\absDim\\
\pst@absdim{4sp}\absDim \the\absDim
\makeatother
\end{example}



%--------------------------------------------------------------------------------------
\section{Dashed Lines}
%--------------------------------------------------------------------------------------
The number of arguments to the dash option is now limited to a maximum number of 11. The old
\PST-version allowed only 2.

\begin{verbatim}
dash=value1[unit] value2[unit] ...
\end{verbatim}


\lstset{wide=true}
\begin{example}[pos=a]
\begin{pspicture}(-5,-4)(5,4)
 \psset{linewidth=2.5pt}
 \psgrid[subgriddiv=0,griddots=10,gridlabels=0pt]
  \psset{linestyle=dashed}
  \pscurve[dash=5mm 1mm 1mm 1mm,linewidth=0.1](-5,4)(-4,3)(-3,4)(-2,3)
  \psline[dash=5mm 1mm 1mm 1mm 1mm 1mm 1mm 1mm 1mm 1mm](-5,0.9)(5,0.9)
  \psccurve[linestyle=solid](0,0)(1,0)(1,1)(0,1)
  \psccurve[linestyle=dashed,dash=5mm 2mm 0.1 0.2,linetype=0](0,0)(-2.5,0)(-2.5,-2.5)(0,-2.5)
  \pscurve[dash=3mm 3mm 1mm 1mm,linecolor=red,linewidth=2pt](5,-4)(5,2)(4.5,3.5)(3,4)(-5,4)
\end{pspicture}
\end{example}

%--------------------------------------------------------------------------------------
\section{\texttt{\textbackslash rmultiput}, a multiple \texttt{\textbackslash rput}}
%--------------------------------------------------------------------------------------
\verb+PSTricks+ already knows a \verb+multirput+, which puts a box n times with
a difference of $dx$ and $dy$ relativ to each other. It is not possible to put
it with a different distance from one point to the next one. This is possible
with \verb+rmultiput+:
\begin{verbatim}
\rmultiput[<options>]{<any material>}(x1,y1)(x2,y2) ... (xn,yn)
\rmultiput*[<options>]{<any material>}(x1,y1)(x2,y2) ... (xn,yn)
\end{verbatim}

\begin{example}[width=6.2cm]
\psset{unit=0.75}
\begin{pspicture}(-4,-4)(4,4)
\rmultiput[rot=45]{\red\psscalebox{3}{\ding{250}}}%
    (-2,-4)(-2,-3)(-3,-3)(-2,-1)(0,0)(1,2)(1.5,3)(3,3)
\rmultiput[rot=90,ref=lC]{\blue\psscalebox{2}{\ding{253}}}%
    (-2,2.5)(-2,2.5)(-3,2.5)(-2,1)(1,-2)(1.5,-3)(3,-3)
\psgrid[subgriddiv=0,gridcolor=lightgray]
\end{pspicture}
\end{example}




%--------------------------------------------------------------------------------------
\section{Arrows, Arrows, \ldots}
%--------------------------------------------------------------------------------------
\subsection{Definition}
%--------------------------------------------------------------------------------------
\verb|pstricks| defines the following ''arrows`` which are saved in the \verb+pst@arrowtable+:

\begin{center}
  \bgroup
  \def\myline#1{\psline[linecolor=red,linewidth=1pt]{#1}(0,1ex)(1.3,1ex)}%
  \psset{arrowscale=1.5}
  \begin{tabular}{cp{1.8cm}l}%
    Value & Example & Name \\[2pt]\hline
    \verb/-/      & \myline{-}      & None\\
    \verb/<->/    & \myline{<->}    & Arrowheads.\\
    \verb/>-</    & \myline{>-<}    & Reverse arrowheads.\\
    \verb/<<->>/  & \myline{<<->>}  & Double arrowheads.\\
    \verb/>>-<</  & \myline{>>-<<}  & Double reverse arrowheads.\\
    \verb/|-|/    & \myline{|-|}    & T-bars, flush to endpoints.\\
    \verb/|*-|*/  & \myline{|*-|*}  & T-bars, centered on endpoints.\\
    \verb/[-]/    & \myline{[-]}    & Square brackets.\\
    \verb/]-[/    & \myline{]-[}    & Reversed square brackets.\\
    \verb/(-)/    & \myline{(-)}    & Rounded brackets.\\
    \verb/)-(/    & \myline{)-(}    & Reversed rounded brackets.\\
    \verb/o-o/    & \myline{o-o}    & Circles, centered on endpoints.\\
    \verb/*-*/    & \myline{*-*}    & Disks, centered on endpoints.\\
    \verb/oo-oo/  & \myline{oo-oo}  & Circles, flush to endpoints.\\
    \verb/**-**/  & \myline{**-**}  & Disks, flush to endpoints.\\
    \verb/|<->|/  & \myline{|<->|}  & T-bars and arrows.\\
    \verb/|>-<|/  & \myline{|>-<|}  & T-bars and reverse arrows.\\
    \verb/H-H/   & \myline{H-H}    & left/right Hook arrows.\\
    \verb/h-h/   & \myline{h-h}    & left/right hook arrows.\\
  \end{tabular}
  \egroup
\end{center}

You can also mix and match, e.g., \verb/->/, \verb/*-)/ and \verb/[->/ are all valid values
of the \verb|arrows| parameter. The parameter can be set with
\begin{verbatim}
\psset{arrows=<type>}
\end{verbatim}

\noindent or for some macros with a special option, like\\[5pt]
\noindent\verb|\psline[<general options>]{<arrow type>}(A)(B)|\\
\noindent\verb/\psline[linecolor=red,linewidth=2pt]{|->}(0,0)(0,2)/ \ \psline[linecolor=red,linewidth=2pt]{|->}(0,0)(0,2)

\subsection{Multiple arrows}
There are two new options which are only valid for the arrow type \verb+<<+ or \verb+>>+.
\verb+nArrow+ sets both, the \verb+nArrowA+ and the  \verb+nArrowB+ parameter. The meaning 
is declared in the following tables. Without setting one of these parameters the behaviour
is like the one described in the old PSTricks manual.

\begin{center}
  \begin{tabular}{lc}%
    Value & Meaning \\[2pt]\hline
    \verb+->>+   & \ -A \\
    \verb+<<->>+ & A-A\\
    \verb+<<-+   & A-\ \\
    \verb+>>-+   & B-\ \\
    \verb+-<<+   & \ -B\\
    \verb+>>-<<+ & B-B\\
    \verb+>>->>+ & B-A\\
    \verb+<<-<<+ & A-B
  \end{tabular}
\end{center}

\begin{center}
  \bgroup
  \psset{linecolor=red,linewidth=1pt,arrowscale=2}%
  \begin{tabular}{lp{2.8cm}}%
    Value & Example \\[2pt]\hline
    \verb+\psline{->>}(0,1ex)(2.3,1ex)+  & \psline{->>}(0,1ex)(2.3,1ex) \\
    \verb+\psline[nArrowsA=3]{->>}(0,1ex)(2.3,1ex)+  & \psline[nArrowsA=3]{->>}(0,1ex)(2.3,1ex)\\
    \verb+\psline[nArrowsA=5]{->>}(0,1ex)(2.3,1ex)+  & \psline[nArrowsA=5]{->>}(0,1ex)(2.3,1ex)\\
    \verb+\psline{<<-}(0,1ex)(2.3,1ex)+  & \psline{<<-}(0,1ex)(2.3,1ex)\\
    \verb+\psline[nArrowsA=3]{<<-}(0,1ex)(2.3,1ex)+  & \psline[nArrowsA=3]{<<-}(0,1ex)(2.3,1ex)\\
    \verb+\psline[nArrowsA=5]{<<-}(0,1ex)(2.3,1ex)+  & \psline[nArrowsA=5]{<<-}(0,1ex)(2.3,1ex)\\
    \verb+\psline{<<->>}(0,1ex)(2.3,1ex)+  & \psline{<<->>}(0,1ex)(2.3,1ex)\\
    \verb+\psline[nArrowsA=3]{<<->>}(0,1ex)(2.3,1ex)+  & \psline[nArrowsA=3]{<<->>}(0,1ex)(2.3,1ex)\\
    \verb+\psline[nArrowsA=5]{<<->>}(0,1ex)(2.3,1ex)+  & \psline[nArrowsA=5]{<<->>}(0,1ex)(2.3,1ex)\\
    \verb+\psline{<<-|}(0,1ex)(2.3,1ex)+  & \psline{<<-|}(0,1ex)(2.3,1ex)\\
    \verb+\psline[nArrowsA=3]{<<-<<}(0,1ex)(2.3,1ex)+  & \psline[nArrowsA=3]{<<-<<}(0,1ex)(2.3,1ex)\\
    \verb+\psline[nArrowsA=5]{<<-o}(0,1ex)(2.3,1ex)+  & \psline[nArrowsA=5]{<<-o}(0,1ex)(2.3,1ex)\\
    \verb+\psline[nArrowsA=3,nArrowsB=4]{<<-<<}(0,1ex)(2.3,1ex)+  & \psline[nArrowsA=3,nArrowsB=4]{<<-<<}(0,1ex)(2.3,1ex)\\
    \verb+\psline[nArrowsA=3,nArrowsB=4]{>>->>}(0,1ex)(2.3,1ex)+  & \psline[nArrowsA=3,nArrowsB=4]{>>->>}(0,1ex)(2.3,1ex)\\
    \verb+\psline[nArrowsA=1,nArrowsB=4]{>>->>}(0,1ex)(2.3,1ex)+  & \psline[nArrowsA=1,nArrowsB=4]{>>->>}(0,1ex)(2.3,1ex)\\
  \end{tabular}
  \egroup
\end{center}


\subsection{\texttt{hookrightarrow} and \texttt{hookleftarrow}}
This is another type of an arrow and abbreviated with \verb+H+. The length and width of the hook
is set by the new options \verb+hooklength+ and \verb+hookwidth+, which are by default set to
%
\begin{verbatim}
\psset{hooklength=3mm,hookwidth=1mm}
\end{verbatim}
%
If the line begins with a right hook then the line ends with a left hook and vice versa:

\begin{example}[width=3cm]
\begin{pspicture}(3,4)
\psline[linewidth=5pt,linecolor=blue,hooklength=5mm,hookwidth=-3mm]{H->}(0,3.5)(3,3.5)
\psline[linewidth=5pt,linecolor=red,hooklength=5mm,hookwidth=3mm]{H->}(0,2.5)(3,2.5)
\psline[linewidth=5pt,hooklength=5mm,hookwidth=3mm]{H-H}(0,1.5)(3,1.5)
\psline[linewidth=1pt]{H-H}(0,0.5)(3,0.5)
\end{pspicture}
\end{example}



\begin{example}[width=7.25cm]
$\begin{psmatrix}
E&W_i(X)&&Y\\
&&W_j(X)
\psset{arrows=->,nodesep=3pt,linewidth=2pt}
\everypsbox{\scriptstyle}
\ncline[linecolor=red,arrows=H->,%
  hooklength=4mm,hookwidth=2mm]{1,1}{1,2}       
\ncline{1,2}{1,4}^{\tilde{t}}
\ncline{1,2}{2,3}<{W_{ij}}
\ncline{2,3}{1,4}>{\tilde{s}}
\end{psmatrix}$
\end{example}


%--------------------------------------------------------------------------------------
\subsection{\texttt{ArrowInside} Option}
%--------------------------------------------------------------------------------------

It is now possible to have arrows inside the lines and not only at the beginning or
the end. The new defined options

\psset{arrowscale=2,linecolor=red,unit=1cm,linewidth=1.5pt}
\begin{longtable}{l|p{7.5cm}|p{2.2cm}}
Name & Example & Output\\\hline
\endfirsthead
Name & Example & Output\\\hline
\endhead
\texttt{ArrowInside} &
	\texttt{\textbackslash psline[ArrowInside=->](0,0)(2,0)} &
	\psline[ArrowInside=->](0,0.1)(2,0.1) \\
\texttt{ArrowInsidePos} & \texttt{\textbackslash psline[ArrowInside=->,\%}\newline
	\hspace*{10pt}\texttt{ArrowInsidePos=0.25](0,0)(2,0)}
& \psline[ArrowInside=->, ArrowInsidePos=0.25](0,0.1)(2,0.1) \\
\texttt{ArrowInsidePos} & \texttt{\textbackslash psline[ArrowInside=->,\%}\newline
	\hspace*{10pt}\texttt{ArrowInsidePos=10](0,0)(2,0)}
& \psline[ArrowInside=->, ArrowInsidePos=10](0,0.1)(2,0.1) \\
\texttt{ArrowInsideNo} & \texttt{\textbackslash psline[ArrowInside=->,\%}\newline
	\hspace*{10pt}\texttt{ArrowInsideNo=2](0,0)(2,0)}
& \psline[ArrowInside=->, ArrowInsideNo=2](0,0.1)(2,0.1) \\
\texttt{ArrowInsideOffset} & \texttt{\textbackslash psline[ArrowInside=->,\%}\newline
	\hspace*{10pt}\texttt{ArrowInsideNo=2,\%}\newline
	\hspace*{10pt}\texttt{ArrowInsideOffset=0.1](0,0)(2,0)}
& \psline[ArrowInside=->, ArrowInsideNo=2,ArrowInsideOffset=0.1](0,0.1)(2,0.1) \\
%
\texttt{ArrowInside} & \texttt{\textbackslash psline[ArrowInside=->]\{->\}(0,0)(2,0)} &
	\psline[ArrowInside=->]{->}(0,0)(2,0)\\
\texttt{ArrowInsidePos} & \texttt{\textbackslash psline[ArrowInside=->,\%}\newline
	\hspace*{10pt}\texttt{ArrowInsidePos=0.25]\{->\}(0,0)(2,0)}
	& \psline[ArrowInside=->, ArrowInsidePos=0.25]{->}(0,0)(2,0) \\
\texttt{ArrowInsidePos} & \texttt{\textbackslash psline[ArrowInside=->,\%}\newline
	\hspace*{10pt}\texttt{ArrowInsidePos=10]\{->\}(0,0)(2,0)}
	& \psline[ArrowInside=->, ArrowInsidePos=10]{->}(0,0)(2,0) \\
\texttt{ArrowInsideNo} & \texttt{\textbackslash psline[ArrowInside=->,\%}\newline
	\hspace*{10pt}\texttt{ArrowInsideNo=2]\{->\}(0,0)(2,0)}
	& \psline[ArrowInside=->, ArrowInsideNo=2]{->}(0,0)(2,0) \\
\texttt{ArrowInsideOffset} & \texttt{\textbackslash psline[ArrowInside=->,\%}\newline
	\hspace*{10pt}\texttt{ArrowInsideNo=2,\%}\newline
	\hspace*{10pt}\texttt{ArrowInsideOffset=0.1]\{->\}(0,0)(2,0)}
	& \psline[ArrowInside=->, ArrowInsideNo=2,ArrowInsideOffset=0.1]{->}(0,0)(2,0) \\
%
\texttt{ArrowFill} & \texttt{\textbackslash psline[ArrowFill=false,\%}\newline
	\hspace*{10pt}\texttt{arrowinset=0]\{->\}(0,0)(2,0)} &
	\psline[ArrowFill=false,arrowinset=0]{->}(0,0)(2,0)\\
\texttt{ArrowFill} & \texttt{\textbackslash psline[ArrowFill=false,\%}\newline
	\hspace*{10pt}\texttt{arrowinset=0]\{<<->>\}(0,0)(2,0)} &
	\psline[ArrowFill=false,arrowinset=0]{<<->>}(0,0)(2,0)\\
\texttt{ArrowFill} & \texttt{\textbackslash psline[ArrowInside=->,\%}\newline
	\hspace*{10pt}\texttt{arrowinset=0,\%}\newline
	\hspace*{10pt}\texttt{ArrowFill=false,\%}\newline
	\hspace*{10pt}\texttt{ArrowInsideNo=2,\%}\newline
	\hspace*{10pt}\texttt{ArrowInsideOffset=0.1]\{->\}(0,0)(2,0)}
	& \psline[ArrowInside=->, ArrowFill=false, ArrowInsideNo=2, ArrowInsideOffset=0.1]{->}(0,0)(2,0) \\
\end{longtable}

\medskip
Without the default arrow definition there is only the one inside the line, defined
by the type and
the position. The position is relative to the length of the whole line. $0.25$ means
at $25\%$ of the
line length. The peak of the arrow gets the coordinates which are calculated by the
macro. If you want arrows with an abolute position difference, then choose a
value greater than \verb|1|, e.g. \verb|10| which places an arrow every 10 pt. The
default unit \verb|pt| cannot be changed.


%--------------------------------------------------------------------------------------
\subsection{\texttt{ArrowFill} Option}
%--------------------------------------------------------------------------------------

By default all arrows are filled polygons. With the option \verb|ArrowFill=false| there
are ''white``  arrows. Only for the beginning/end arrows they are empty, the inside arrows
are overpainted with the line.

\resetPSTOptions

\begin{example}[width=3.5cm]
\psset{arrowscale=3}
\psline[linecolor=red,arrowinset=0]{<->}(0,0)(3,0)
\end{example}

\begin{example}[width=3.5cm]
\psset{arrowscale=3}
\psline[linecolor=red,arrowinset=0,ArrowFill=false]{<->}(0,0)(3,0)
\end{example}

\begin{example}[width=3.5cm]
\psline[linecolor=red,arrowinset=0,arrowsize=0.5,ArrowFill=false]{<->}(0,0)(3,0)
\end{example}

\begin{example}[width=3.5cm]
\psline[linecolor=blue,arrowscale=3,ArrowFill=true]{>>->>}(0,0)(3,0)
\end{example}

\begin{example}[width=3.5cm]
\psline[linecolor=blue,arrowscale=3,ArrowFill=false]{>>->>}(0,0)(3,0)
\rule{3cm}{0pt}\\[30pt]
\end{example}

\begin{example}[width=3.5cm]
\psline[linecolor=blue,arrowscale=3,ArrowFill=true]{>|->|}(0,0)(3,0)
\end{example}

\begin{example}[width=3.5cm]
\psline[linecolor=blue,arrowscale=3,ArrowFill=false]{>|->|}(0,0)(3,0)%
\end{example}


%--------------------------------------------------------------------------------------
\subsection{Examples}
%--------------------------------------------------------------------------------------

All examples are printed with \verb|\psset{arrowscale=2,linecolor=red}|.
\subsubsection{\texttt{\textbackslash psline}}

\bigskip
\begin{example}[width=2.5cm]
\begin{pspicture}(2,2)
\psset{arrowscale=2,ArrowFill=true}
\psline[ArrowInside=->]{|<->|}(2,1)
\end{pspicture}
\end{example}

\begin{example}[width=2.5cm]
\begin{pspicture}(2,2)
\psset{arrowscale=2,ArrowFill=true}
\psline[ArrowInside=-|]{|-|}(2,1)
\end{pspicture}
\end{example}

\begin{example}[width=2.5cm]
\begin{pspicture}(2,2)
\psset{arrowscale=2,ArrowFill=true}
\psline[ArrowInside=->,ArrowInsideNo=2]{->}(2,1)
\end{pspicture}
\end{example}

\begin{example}[width=2.5cm]
\begin{pspicture}(2,2)
\psset{arrowscale=2,ArrowFill=true}
\psline[ArrowInside=->,ArrowInsideNo=2,ArrowInsideOffset=0.1]{->}(2,1)
\end{pspicture}
\end{example}

\begin{example}[width=6.5cm]
\begin{pspicture}(6,2)
\psset{arrowscale=2,ArrowFill=true}
\psline[ArrowInside=-*]{->}(0,0)(2,1)(3,0)(4,0)(6,2)
\end{pspicture}
\end{example}

\begin{example}[width=6.5cm]
\begin{pspicture}(6,2)
\psset{arrowscale=2,ArrowFill=true}
\psline[ArrowInside=-*,ArrowInsidePos=0.25]{->}(0,0)(2,1)(3,0)(4,0)(6,2)
\end{pspicture}
\end{example}

\begin{example}[width=6.5cm]
\begin{pspicture}(6,2)
\psset{arrowscale=2,ArrowFill=true}
\psline[ArrowInside=-*,ArrowInsidePos=0.25,ArrowInsideNo=2]{->}%
   (0,0)(2,1)(3,0)(4,0)(6,2)
\end{pspicture}
\end{example}

\begin{example}[width=6.5cm]
\begin{pspicture}(6,2)
\psset{arrowscale=2,ArrowFill=true}
\psline[ArrowInside=->, ArrowInsidePos=0.25]{->}%
        (0,0)(2,1)(3,0)(4,0)(6,2)
\end{pspicture}
\end{example}

\begin{example}[width=6.5cm]
\begin{pspicture}(6,2)
\psset{arrowscale=2,ArrowFill=true}
\psline[linestyle=none,ArrowInside=->,ArrowInsidePos=0.25]{->}%
        (0,0)(2,1)(3,0)(4,0)(6,2)
\end{pspicture}
\end{example}

\begin{example}[width=6.5cm]
\begin{pspicture}(6,2)
\psset{arrowscale=2,ArrowFill=true}
\psline[ArrowInside=-<, ArrowInsidePos=0.75]{->}%
     (0,0)(2,1)(3,0)(4,0)(6,2)
\end{pspicture}
\end{example}

\begin{example}[width=6.5cm]
\begin{pspicture}(6,2)
\psset{arrowscale=2,ArrowFill=true,ArrowInside=-*}
\psline(0,0)(2,1)(3,0)(4,0)(6,2)
\psset{linestyle=none}
\psline[ArrowInsidePos=0](0,0)(2,1)(3,0)(4,0)(6,2)
\psline[ArrowInsidePos=1](0,0)(2,1)(3,0)(4,0)(6,2)
\end{pspicture}
\end{example}

\begin{example}[width=6.5cm]
\begin{pspicture}(6,5)
\psset{arrowscale=2,ArrowFill=true}
\psline[ArrowInside=->,ArrowInsidePos=20](0,0)(3,0)%
       (3,3)(1,3)(1,5)(5,5)(5,0)(7,0)(6,3)
\end{pspicture}
\end{example}

\begin{example}[width=6.5cm]
\begin{pspicture}(6,2)
\psset{arrowscale=2,ArrowFill=true}
\psline[linearc=0.5,ArrowInside=-|]{<->}(0,2)(2,0)(3,2)(4,0)(6,2)
\end{pspicture}
\end{example}


%--------------------------------------------------------------------------------------
\subsubsection{\texttt{\textbackslash pspolygon}}
%--------------------------------------------------------------------------------------
% Polygons (\pspolygon macro)

\begin{example}[width=6.5cm]
\begin{pspicture}(6,3)
\psset{arrowscale=2}
\pspolygon[ArrowInside=-|](0,0)(3,3)(6,3)(6,1)
\end{pspicture}
\end{example}

\begin{example}[width=6.5cm]
\begin{pspicture}(6,3)
\psset{arrowscale=2}
\pspolygon[ArrowInside=->,ArrowInsidePos=0.25]%
     (0,0)(3,3)(6,3)(6,1)
\end{pspicture}
\end{example}

\begin{example}[width=6.5cm]
\begin{pspicture}(6,3)
\psset{arrowscale=2}
\pspolygon[ArrowInside=->,ArrowInsideNo=4]%
       (0,0)(3,3)(6,3)(6,1)
\end{pspicture}
\end{example}

\begin{example}[width=6.5cm]
\begin{pspicture}(6,3)
\psset{arrowscale=2}
\pspolygon[ArrowInside=->,ArrowInsideNo=4,%
   ArrowInsideOffset=0.1](0,0)(3,3)(6,3)(6,1)
\end{pspicture}
\end{example}

\begin{example}[width=6.5cm]
\begin{pspicture}(6,3)
\psset{arrowscale=2}
 \pspolygon[ArrowInside=-|](0,0)(3,3)(6,3)(6,1)
 \psset{linestyle=none,ArrowInside=-*}
 \pspolygon[ArrowInsidePos=0](0,0)(3,3)(6,3)(6,1)
 \pspolygon[ArrowInsidePos=1](0,0)(3,3)(6,3)(6,1)
 \psset{ArrowInside=-o}
 \pspolygon[ArrowInsidePos=0.25](0,0)(3,3)(6,3)(6,1)
 \pspolygon[ArrowInsidePos=0.75](0,0)(3,3)(6,3)(6,1)
\end{pspicture}
\end{example}

\begin{example}[width=6.5cm]
\begin{pspicture}(6,5)
\psset{arrowscale=2}
  \pspolygon[ArrowInside=->,ArrowInsidePos=20]%
    (0,0)(3,0)(3,3)(1,3)(1,5)(5,5)(5,0)(7,0)(6,3)
\end{pspicture}
\end{example}


%--------------------------------------------------------------------------------------
\subsubsection{\texttt{\textbackslash psbezier}}
%--------------------------------------------------------------------------------------
% Bezier curves (\psbezier macro)

\begin{example}[width=3.5cm]
\begin{pspicture}(3,3)
\psset{arrowscale=2}
  \psbezier[ArrowInside=-|](1,1)(2,2)(3,3)
  \psset{linestyle=none,ArrowInside=-o}
  \psbezier[ArrowInsidePos=0.25](1,1)(2,2)(3,3)
  \psbezier[ArrowInsidePos=0.75](1,1)(2,2)(3,3)
  \psset{linestyle=none,ArrowInside=-*}
  \psbezier[ArrowInsidePos=0](1,1)(2,2)(3,3)
  \psbezier[ArrowInsidePos=1](1,1)(2,2)(3,3)
\end{pspicture}
\end{example}

\begin{example}[width=4.5cm]
\begin{pspicture}(4,3)
\psset{arrowscale=2}
  \psbezier[ArrowInside=->,showpoints=true]%
     {*-*}(2,3)(3,0)(4,2)
\end{pspicture}
\end{example}


\begin{example}[width=4.5cm]
\begin{pspicture}(4,3)
\psset{arrowscale=2}
  \psbezier[ArrowInside=->,showpoints=true,%
      ArrowInsideNo=2](2,3)(3,0)(4,2)
\end{pspicture}
\end{example}


\begin{example}[width=4.5cm]
\begin{pspicture}(4,3)
\psset{arrowscale=2}
  \psbezier[ArrowInside=->,showpoints=true,%
      ArrowInsideNo=2,ArrowInsideOffset=-0.2]{->}(2,3)(3,0)(4,2)
\end{pspicture}
\end{example}


\begin{example}[width=5.5cm]
\begin{pspicture}(5,3)
\psset{arrowscale=2}
  \psbezier[ArrowInsideNo=9,ArrowInside=-|,%
    showpoints=true]{*-*}(1,3)(3,0)(5,3)
\end{pspicture}
\end{example}

\begin{example}[width=4.5cm]
\begin{pspicture}(4,3)
\psset{arrowscale=2}
  \psset{ArrowInside=-|}
  \psbezier[ArrowInsidePos=0.25,showpoints=true]{*-*}(2,3)(3,0)(4,2)
  \psset{linestyle=none}
  \psbezier[ArrowInsidePos=0.75](2,3)(3,0)(4,2)
\end{pspicture}
\end{example}

\begin{example}[width=5.5cm]
\begin{pspicture}(5,6)
\psset{arrowscale=2}
  \pnode(3,4){A}\pnode(5,6){B}\pnode(5,0){C}
  \psbezier[ArrowInside=->,%
     showpoints=true](A)(B)(C)
  \psset{linestyle=none,ArrowInside=-<}
  \psbezier[ArrowInsideNo=4](A)(B)(C)
  \psset{ArrowInside=-o}
  \psbezier[ArrowInsidePos=0.1](A)(B)(C)
  \psbezier[ArrowInsidePos=0.9](A)(B)(C)
  \psset{ArrowInside=-*}
  \psbezier[ArrowInsidePos=0.3](A)(B)(C)
  \psbezier[ArrowInsidePos=0.7](A)(B)(C)
\end{pspicture}
\end{example}


\begin{example}[pos=a]
\makebox[\linewidth]{
\begin{pspicture}(-3,-5)(0.75\linewidth,3)
  \psbezier[ArrowInsideNo=19,%
      ArrowInside=->,ArrowFill=false,%
      showpoints=true,arrowscale=3]{->}(-3,0)(5,-5)(8,3)(0.75\linewidth,-5)
\end{pspicture}}
\end{example}


%--------------------------------------------------------------------------------------
\section{Random dots}
%--------------------------------------------------------------------------------------
The syntax of the new macro \verb+\psRandom+ is:

\resetPSTOptions

\begin{verbatim}
\psRandom[<option>]{}
\psRandom[<option>]{<clip path>}
\psRandom[<option>](<xMax,yMax>){<clip path>}
\psRandom[<option>](<xMin,yMin>)(<xMax,yMax>){<clip path>}
\end{verbatim}

If there is no area for the dots defined, then \verb+(0,0)(1,1)+ in the actual
scale is used for placing the dots. This area should be greater than the clipping
path to be sure that the dots are placed over the full area. The clipping path can
be everything. If no clipping path is given, then the frame \verb+(0,0)(1,1)+
in user coordinates is used.  The new options are:
 
\begin{center}
\begin{tabular}{l|l|l}
name & default\\\hline
\verb|randomPoints| &   \verb|1000| & number of random dots\tabularnewline
\verb|color| & \verb+false+ & random color\tabularnewline
\end{tabular}
\end{center}


\begin{example}[pos=a,wide=false]
\psset{unit=5cm}
\begin{pspicture}(1,1)
  \psRandom[dotsize=1pt,fillstyle=solid](1,1){\pscircle(0.5,0.5){0.5}}
\end{pspicture}
\begin{pspicture}(1,1)
  \psRandom[dotsize=2pt,randomPoints=5000,color,%
      fillstyle=solid](1,1){\pscircle(0.5,0.5){0.5}}
\end{pspicture}
\end{example}

\begin{example}[pos=a,wide=false]
\psset{unit=5cm}
\begin{pspicture}(1,1)
  \psRandom[randomPoints=200,dotsize=8pt,dotstyle=+]{}
\end{pspicture}
\begin{pspicture}(1.5,1)
  \psRandom[dotsize=5pt,color](0,0)(1.5,0.8){\psellipse(0.75,0.4)(0.75,0.4)}
\end{pspicture}
\end{example}

\begin{example}[pos=a,wide=false]
\psset{unit=3cm}
\begin{pspicture}(0,-1)(3,1)
  \psRandom[dotsize=4pt,dotstyle=o,linecolor=blue,fillcolor=red,%
     fillstyle=solid,randomPoints=1000]%
      (0,-1)(3,1){\psplot{0}{3.14}{ x 114 mul sin }}
\end{pspicture}
\end{example}


\section{''Transparent``colors}
% ---------------------------------------------------------------------------------------
\index{transparent colour}\index{colour!transparent} By default, \PS is transparent 
because the normally white base area is overwritten;
however,  with the filling of a region the base colour is not visible anymore. With
the help of the fill styles, ``transparent''{} colours can be created. This is 
achieved by putting  the fill lines so tightly together that they are no longer  recognized
as a line pattern,  but the underlying colour remains visible. Therefore,
one best defines a small macro \verb+\defineTColor+, which
defines a corresponding new style.

\begin{lstlisting}[caption={Definition of a ``transparent'' colour}]
\def\defineTColor#1#2{% transparent "colours"
  \newpsstyle{#1}{
    fillstyle=vlines,hatchcolor=#2,
    hatchwidth=0.1\pslinewidth,hatchsep=1\pslinewidth%
}}
\end{lstlisting}

On loading the package \verb+pstricks-add+ this macro is
already available, and so a crossing of definitions is possible. Parameters are
options, name and base colour, which needs to be defined.

\begin{example}[width=5cm]
\defineTColor{tRot}{red}
\defineTColor{tCyan}{cyan}
\begin{pspicture}(0,-1)(5,6)
\rput(2.5,2.5){\psframebox[doubleline=true,framearc=0.3]{\Huge\textsf{ \PS}}}
\rput{-30}(1,1){\psframe[style=tRot](2.5,4)}
\rput{30}(2.5,1){\psframe[style=tCyan](2.5,4)}
\end{pspicture}
\end{example}

Remember  that when printing, moire effects\index{moire effect} may
occur, since the lines may overlap unpropitiously. Alternatives are other angles
or choosing the fill style \verb+crosshatch+.



%--------------------------------------------------------------------------------------
\section{\CMD{resetPSTOptions}}
%--------------------------------------------------------------------------------------

Sometimes it is difficult to know what options are changed inside a long document. With this
macro all options depending to \verb+pstricks+ can be reset.

\begin{lstlisting}
\def\resetPSTOptions{%
\psset{shift=0,%
       PstDebug=0,
       swapaxes=false,showpoints=false,border=0pt,bordercolor=white,%
       doubleline=false,doublesep=1.25\pslinewidth,doublecolor=white,%
       shadow=false,shadowsize=3pt,shadowangle=-45,shadowcolor=darkgray,%
       linewidth=.8pt,linecolor=black,
       maxdashes=11,dash=5pt 3pt 0pt 0pt,dashadjust=true,% black white black white 
       hatchangle=45,hatchcolor=black,hatchsep=4pt,hatchwidth=.8pt,%
       fillcolor=white,linestyle=solid,dotsep=3pt,%
       arrowinset=.4,arrowlength=1.4,arrowsize=1.5pt 2,%
       arrowscale=1,fillstyle=none,%
       ArrowFill=true,
       rbracketlength=0.15,bracketlength=0.15,tbarsize=2pt 5,
       hooklength=3mm,hookwidth=1mm,
       nArrows=2,
       ArrowInside={},
       ArrowInsidePos=0.5,
       ArrowInsideNo=1,ArrowInsideOffset=0,
       arrows=-,
       liftpen=0,
       linetype=2,% otherwise there is a problem when using e.g.
       gangle=0,
       curvature=1 .1 0,
       dotsize=2pt 2,
       dotangle=0,
       dotscale=1,
       dotstyle=*,
       dimen=outer,cornersize=relative,framearc=0,linearc=0pt,
       gridlabelcolor=black,gridlabels=10pt,subgriddiv=5,subgriddots=0,%
       subgridcolor=gray,subgridwidth=.4pt,gridcolor=black,griddots=0,%
       gridwidth=.8pt,%
       boxsep=true,framesep=3pt,
       trimode=U,
       arcsep=0,
       radius=.25cm,
       rot=0,ref=c,
       labelsep=5pt,
       refangle=0}
}
\end{lstlisting}


%--------------------------------------------------------------------------------------
\section{Credits}
%--------------------------------------------------------------------------------------
%{Hendri Adriaans | } 
%{Ulrich Dirr | } 
%{Hubert G��lein |}
{Denis Girou | } 
%{Peter Hutnick | } 
%{Christophe Jorssen | } 
%{Manuel Luque | } 
%{Jens-Uwe Morawski |}
{Tobias N�hring |}
%{Rolf Niepraschk |}
{Dominique Rodriguez |}
{Timothy Van Zandt}



\nocite{*}
\bibliographystyle{plain}
\bibliography{pstricks}


%--------------------------------------------------------------------------------------
\section{Change log}
%--------------------------------------------------------------------------------------

See file Changes

%\clearpage
%\section{The code}

%\lstinputlisting{pstricks-add.tex}
