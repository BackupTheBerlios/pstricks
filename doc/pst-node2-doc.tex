\title{\texttt{pst-node}\\plotting data and mathematical functions\thanks{%
    This document was written with \texttt{Kile: 1.7} (\texttt{Qt: 3.1.1; KDE: 3.3;}
    \url{http://sourceforge.net/projects/kile/}) and the PDF output
    was build with VTeX/Free (\url{http://www.micropress-inc.com/linux})}\\
    \small v.\pstnodeFV}
\author{Herbert Vo�}
\date{\today}

\maketitle

\begin{abstract}
This version of \verb+pst-node+ includes all the macros which were for testing
part of the  \verb+pstricks-add+-package. This documentation shows only these
extensions. For the other macros have a look into the old PSTricks
documentation.

\verb+pst-node+ uses the extended version of the keyval package. So be sure, that
you have installed \verb+pst-xkey+ which is part of the \verb+xkeyval+-package and that all
packages, that uses the old keyval interface are loaded \textbf{before} the \verb+xkeyval+.
\end{abstract}

\tableofcontents

\clearpage



\lstset{wide=true}


%--------------------------------------------------------------------------------------
\section{\texttt{\textbackslash ncdiag} and \texttt{\textbackslash pcdiag}}
%--------------------------------------------------------------------------------------
With the new option \verb|lineAngle| the lines drawn by the \verb|ncdiag| macro
can now have a specified gradient. Without this option one has to define the two
arms (which maybe zero) and PSTricks draws the connection between them. Now there
is only a static \verb|armA|, the second one \verb|armB| is calculated when an angle
\verb|lineAngle| is defined. This angle is the gradient of the intermediate line
between the two arms. The syntax of \verb|ncdiag| is

\begin{verbatim}
\ncdiag[<options>]{<Node A>}{<Node B>}
\pcdiag[<options>](<Node A>)(<Node B>)
\end{verbatim}


\begin{tabularx}{\linewidth}{l|X}
name & meaning\\\hline
\verb|lineAngle| & angle of the intermediate line segment. Default is 0, which is the same than using \verb|ncdiag| without the \verb|lineAngle| option.\tabularnewline
\end{tabularx}


\resetPSTNodeOptions
\begin{example}[width=5.5cm]
\begin{pspicture}(5,6)
  \circlenode{A}{A}\quad\circlenode{C}{C}%
    \quad\circlenode{E}{E}
  \rput(0,4){\circlenode{B}{B}}
  \rput(1,5){\circlenode{D}{D}}
  \rput(2,6){\circlenode{F}{F}}
  \psset{arrowscale=2,linearc=0.2,%
    linecolor=red,armA=0.5, angleA=90,angleB=-90}
  \ncdiag[lineAngle=20]{->}{A}{B}
  \ncput*[nrot=:U]{line I}
  \ncdiag[lineAngle=20]{->}{C}{D}
  \ncput*[nrot=:U]{line II}
  \ncdiag[lineAngle=20]{->}{E}{F}
  \ncput*[nrot=:U]{line III}
\end{pspicture}
\end{example}


The \verb|ncdiag| macro sets the \verb|armB| dynamically to the calculated value. Any
user setting of \verb|armB| is overwritten by the macro. The \verb|armA| could be set to
a zero length:



\begin{example}[width=4.5cm]
\begin{pspicture}(4,3)
  \rput(0.5,0.5){\circlenode{A}{A}}
  \rput(3.5,3){\circlenode{B}{B}}
  {\psset{linecolor=red,arrows=<-,arrowscale=2}
  \ncdiag[lineAngle=60,%
      armA=0,angleA=0,angleB=180]{A}{B}
  \ncdiag[lineAngle=60,%
      armA=0,angleA=90,angleB=180]{A}{B}}
\end{pspicture}
\end{example}


\begin{example}[width=4.5cm]
\begin{pspicture}(4,3)
  \rput(1,0.5){\circlenode{A}{A}}
  \rput(4,3){\circlenode{B}{B}}
  {\psset{linecolor=red,arrows=<-,arrowscale=2}
  \ncdiag[lineAngle=60,%
      armA=0.5,angleA=0,angleB=180]{A}{B}
  \ncdiag[lineAngle=60,%
      armA=0,angleA=70,angleB=180]{A}{B}
  \ncdiag[lineAngle=60,%
      armA=0.5,angleA=180,angleB=180]{A}{B}}
\end{pspicture}
\end{example}

\begin{example}[width=4.5cm]
\begin{pspicture}(4,5.5)
  \cnode*(0,0){2pt}{A}%
  \cnode*(0.25,0){2pt}{C}%
  \cnode*(0.5,0){2pt}{E}%
  \cnode*(0.75,0){2pt}{G}%
  \cnode*(2,4){2pt}{B}%
  \cnode*(2.5,4.5){2pt}{D}%
  \cnode*(3,5){2pt}{F}%
  \cnode*(3.5,5.5){2pt}{H}%
  {\psset{arrowscale=2,linearc=0.2,%
    linecolor=red,armA=0.5, angleA=90,angleB=-90}
  \pcdiag[lineAngle=20]{->}(A)(B)
  \pcdiag[lineAngle=20]{->}(C)(D)
  \pcdiag[lineAngle=20]{->}(E)(F)
  \pcdiag[lineAngle=20]{->}(G)(H)}
\end{pspicture}
\end{example}





%--------------------------------------------------------------------------------------
\section{\CMD{ncdiagg} and \CMD{pcdiagg}}
%--------------------------------------------------------------------------------------
This is nearly the same than \verb+\ncdiag+ except that \verb+armB=0+ and the \verb+angleB+
value is computed by the macro, so that the line ends at the node with an angle
like a \verb+\pcdiagg+ line. The syntax of \verb|ncdiagg|/\verb+pcdiagg+ is

\begin{verbatim}
\ncdiag[<options>]{<Node A>}{<Node B>}
\pcdiag[<options>](<Node A>)(<Node B>)
\end{verbatim}

\begin{example}[width=5cm]
\begin{pspicture}(4,6)
  \psset{linecolor=black}
  \circlenode{A}{A}%
  \quad\circlenode{C}{C}%
  \quad\circlenode{E}{E}
  \rput(0,4){\circlenode{B}{B}}
  \rput(1,5){\circlenode{D}{D}}
  \rput(2,6){\circlenode{F}{F}}
  {\psset{arrowscale=2,linearc=0.2,linecolor=red,armA=0.5, angleA=90}
  \ncdiagg[lineAngle=-160]{->}{A}{B}
  \ncput*[nrot=:U]{line I}
  \ncdiagg[lineAngle=-160]{->}{C}{D}
  \ncput*[nrot=:U]{line II}
  \ncdiagg[lineAngle=-160]{->}{E}{F}
  \ncput*[nrot=:U]{line III}}
\end{pspicture}
\end{example}

\begin{example}[width=5cm]
\begin{pspicture}(4,6)
  \psset{linecolor=black}
  \cnode*(0,0){2pt}{A}%
  \cnode*(0.25,0){2pt}{C}%
  \cnode*(0.5,0){2pt}{E}%
  \cnode*(0.75,0){2pt}{G}%
  \cnode*(2,4){2pt}{B}%
  \cnode*(2.5,4.5){2pt}{D}%
  \cnode*(3,5){2pt}{F}%
  \cnode*(3.5,5.5){2pt}{H}%
  {\psset{arrowscale=2,linearc=0.2,linecolor=red,armA=0.5, angleA=90}
  \pcdiagg[lineAngle=20]{->}(A)(B)
  \pcdiagg[lineAngle=20]{->}(C)(D)
  \pcdiagg[lineAngle=20]{->}(E)(F)
  \pcdiagg[lineAngle=20]{->}(G)(H)}
\end{pspicture}
\end{example}

The only catch for \verb+\ncdiagg+ is, that you need the right value for \verb+lineAngle+.
If the node connection is on the wrong side
of the second node, then choose the corresponding angle, e.g.: if $20$ is wrong then take
$-160$, the corresponding to $180$.


\begin{example}[width=4cm]
\begin{pspicture}(4,1.5)
  \circlenode{a}{A}
  \rput[l](3,1){\rnode{b}{H}}
  \ncdiagg[lineAngle=60,angleA=180,armA=.5,nodesepA=3pt,linecolor=blue]{b}{a}
\end{pspicture}
\end{example}

\begin{example}[width=4cm]
\begin{pspicture}(4,1.5)
  \circlenode{a}{A}
  \rput[l](3,1){\rnode{b}{H}}
  \ncdiagg[lineAngle=60,armA=.5,nodesepB=3pt,linecolor=blue]{a}{b}
\end{pspicture}
\end{example}

\begin{example}[width=4cm]
\begin{pspicture}(4,1.5)
  \circlenode{a}{A}
  \rput[l](3,1){\rnode{b}{H}}
  \ncdiagg[lineAngle=-120,armA=.5,nodesepB=3pt,linecolor=blue]{a}{b}
\end{pspicture}
\end{example}


%--------------------------------------------------------------------------------------
\section{\CMD{ncbarr} und \CMD{pcbarr}}
%--------------------------------------------------------------------------------------
This has the same behaviour as \verb+ncbar+, but has 5 segments and all are
horizontal ones. This is the reason why \verb+angleA+ should be $0�$ or alternative $180�$.
Other values are possible but don't make really sense. \verb+AngleB+ is in geenral
\verb+AngleA+180+ and cannot be set to another value. The intermediate horizontal line is
symmetrical to the distance of the two nodes and can be set with the option \verb+tpos+ ($0\le tpos\le 1$).


\begin{example}[width=3.5cm]
\psset{arrowscale=2}%
\circlenode{X}{X}\\[1cm]
\circlenode{Y}{Y}
\ncbarr[angleA=0,arrows=->,arrowscale=2]{X}{Y}
\end{example}

\begin{example}[width=3.5cm]
\psset{arrowscale=2}%
\ovalnode{X}{Xxxxx}\\[1cm]
\circlenode{Y}{Yyyy}
\ncbarr[angleA=180,arrows=->,arrowscale=2,linecolor=red,armB=0.75]{X}{Y}
\end{example}

\begin{example}[width=3.5cm]
\psset{arrowscale=2}%
\ovalnode{X}{Xxxxx}\\[1cm]
\circlenode{Y}{Yyyy}
\ncbarr[angleA=180,arrows=->,arrowscale=2,linecolor=red,armB=0.75,tpos=0.65]{X}{Y}
\end{example}

\begin{example}[width=3.5cm]
\psset{arrowscale=2}%
\ovalnode{X}{Xxxxx}\\[1cm]
\circlenode{Y}{Yyyy}
\ncbarr[angleA=20,arm=1cm,arrows=->,arrowscale=2]{X}{Y}
\end{example}

\begin{example}[width=3.5cm]
\psset{arrowscale=2}%
\ovalnode{X}{Xxxxx}\\[1cm]
\circlenode{Y}{Yyyy}
\pcbarr[arm=1cm,arrows=->,arrowscale=2,tpos=0.3](Y)(X)
\end{example}


%--------------------------------------------------------------------------------------
\section{\CMD{psRelLine}}
%--------------------------------------------------------------------------------------
With this macro it is possible to plot lines relative to a given one. Parameter are
the angle and the length factor:
\begin{verbatim}
\psRelLine(<P0>)(<P1>){<length factor>}{<end node name>}
\psRelLine{<arrows>}(<P0>)(<P1>){<length factor>}{<end node name>}
\psRelLine[<options>](<P0>)(<P1>){<length factor>}{<end node name>}
\psRelLine[<options>]{<arrows>}(<P0>)(<P1>){<length factor>}{<end node name>}
\end{verbatim}

The length factor depends to the distance of $\overline{P_0P_1}$ and the end node name must
be a valid nodename and shouldn't contain any of the special PostScript characters. There are
two valid options:

\begin{tabularx}{\linewidth}{l|l|X}
name & default & meaning\\\hline
\verb|angle| & $0$ & angle between the given line $\overline{P_0P_1}$ and the new one
$\overline{P_0P_endNode}$\tabularnewline
\verb+trueAngle+ & false & defines whether the angle depends to the seen line or to the
mathematical one, which respect the scaling factors \verb+xunit+ and \verb+yunit+.
\end{tabularx}

The following two figures show the same, the first one with a scaling different to $1:1$,
this is the reason why the end points are on an ellipse and not on a circle like in the
second figure.

\begin{example}[width=5.5cm]
\psset{yunit=2,xunit=1}
\begin{pspicture}(-2,-2)(3,2)
\psgrid[subgriddiv=2,subgriddots=10,gridcolor=lightgray]
\pnode(-1,0){A}\pnode(3,2){B}
\psline[linecolor=red](A)(B)
\psRelLine[linecolor=blue,angle=30](-1,0)(B){0.5}{EndNode}
\qdisk(EndNode){2pt}
\psRelLine[linecolor=blue,angle=-30](A)(B){0.5}{EndNode}
\qdisk(EndNode){2pt}
\psRelLine[linecolor=magenta,angle=90](-1,0)(3,2){0.5}{EndNode}
\qdisk(EndNode){2pt}
\psRelLine[linecolor=magenta,angle=-90](A)(B){0.5}{EndNode}
\qdisk(EndNode){2pt}
\end{pspicture}
\end{example}

\begin{example}[width=5.5cm]
\begin{pspicture}(-2,-2)(3,2)
\psgrid[subgriddiv=2,subgriddots=10,gridcolor=lightgray]
\pnode(-1,0){A}\pnode(3,2){B}
\psline[linecolor=red](A)(B)
\psarc[linestyle=dashed](A){2.23}{-90}{135}
\psRelLine[linecolor=blue,angle=30](-1,0)(B){0.5}{EndNode}
\qdisk(EndNode){2pt}
\psRelLine[linecolor=blue,angle=-30](A)(B){0.5}{EndNode}
\qdisk(EndNode){2pt}
\psRelLine[linecolor=magenta,angle=90](-1,0)(3,2){0.5}{EndNode}
\qdisk(EndNode){2pt}
\psRelLine[linecolor=magenta,angle=-90](A)(B){0.5}{EndNode}
\qdisk(EndNode){2pt}
\end{pspicture}
\end{example}



The following figure has also a different scaling, but has set the option \verb+trueAngle+,
all angles depends to what "you see".

\begin{example}[width=6.5cm]
\psset{yunit=2,xunit=1}
\begin{pspicture}(-3,-1)(3,2)\psgrid[subgridcolor=lightgray]
\pnode(-1,0){A}\pnode(3,2){B}
\psline[linecolor=red](A)(B)
\psarc(A){2.83}{-45}{135}
\psRelLine[linecolor=blue,angle=30,trueAngle](A)(B){0.5}{EndNode}
\qdisk(EndNode){2pt}
\psRelLine[linecolor=blue,angle=-30,trueAngle](A)(B){0.5}{EndNode}
\qdisk(EndNode){2pt}
\psRelLine[linecolor=magenta,angle=90,trueAngle](A)(B){0.5}{EndNode}
\qdisk(EndNode){2pt}
\psRelLine[linecolor=magenta,angle=-90,trueAngle](A)(B){0.5}{EndNode}
\qdisk(EndNode){2pt}
\end{pspicture}
\end{example}

Two examples with using \verb+\multido+ to show the behaviour of the options \verb+trueAngle+
and \verb+angle+.

\medskip
\begin{example}[width=8cm]
\psset{yunit=4,xunit=2}
\begin{pspicture}(-1,0)(3,2)\psgrid[subgridcolor=lightgray]
\pnode(-1,0){A}\pnode(1,1){B}
\psline[linecolor=red](A)(3,2)
\multido{\iA=0+10}{36}{%
  \psRelLine[linecolor=blue,angle=\iA](B)(A){-0.5}{EndNode}
  \qdisk(EndNode){2pt}
}
\end{pspicture}
\end{example}

\begin{example}[width=8cm]
\psset{yunit=4,xunit=2}
\begin{pspicture}(-1,0)(3,2)\psgrid[subgridcolor=lightgray]
\pnode(-1,0){A}\pnode(1,1){B}
\psline[linecolor=red](A)(3,2)
\multido{\iA=0+10}{36}{%
  \psRelLine[linecolor=magenta,angle=\iA,trueAngle]{->}(B)(A){-0.5}{EndNode}
}
\end{pspicture}
\end{example}


\begin{example}[pos=a]
\psset{xunit=\linewidth,yunit=\linewidth,trueAngle}%
\begin{pspicture}(1,0.6)%\psgrid
  \pnode(.3,.35){Vk} \pnode(.375,.35){D}
  \pnode(0,.4){DST1} \pnode(1,.18){DST2}
  \pnode(0,.1){A1}   \pnode(1,.31){A1}
  { \psset{linewidth=.02,linestyle=dashed,linecolor=gray}%
    \pcline(DST1)(DST2) % <- Druckseitentangente
    \pcline(A2)(A1) % <- Anstr�mmrichtung
    \lput*{:U}{\small Anstr�mrichtung $v_{\infty}$}
  }%
  \psIntersectionPoint(A1)(A2)(DST1)(DST2){Hk}
  \pscurve(Hk)(.4,.38)(Vk)(.36,.33)(.5,.32)(Hk)
  \psParallelLine[linecolor=red!75!green,arrows=->,arrowscale=2](Vk)(Hk)(D){.1}{FtE}
  \psRelLine[linecolor=red!75!green,arrows=->,%
      arrowscale=2,angle=90](D)(FtE){4}{Fn}% why "4"?
  \psParallelLine[linestyle=dashed](D)(FtE)(Fn){.1}{Fnr1}
  \psRelLine[linestyle=dashed,angle=90](FtE)(D){-4}{Fnr2} % why "-4"?
  \psline[linewidth=1.5pt,arrows=->,arrowscale=2](D)(Fnr2)
  \psIntersectionPoint(D)([nodesep=2]D)(Fnr1)([offset=-4]Fnr1){Fh}
  \psIntersectionPoint(D)([offset=2]D)(Fnr1)([nodesep=4]Fnr1){Fv}
  \psline[linecolor=blue,arrows=->,arrowscale=2](D)(Fh)
  \psline[linecolor=blue,arrows=->,arrowscale=2](D)(Fv)
  \psline[linestyle=dotted](Fh)(Fnr1)
  \psline[linestyle=dotted](Fv)(Fnr1)
  \uput{.1}[0](Fh){\blue $F_{H}$}
  \uput{.1}[180](Fv){\blue $F_{V}$}
  \uput{.1}[-45](Fnr1){$F_{R}$}
  \uput{.1}[90](Fn){\color{red!75!green}$F_{N}$}
  \uput{.25}[-90](FtE){\color{red!75!green}$F_{T}$}
\end{pspicture}
\end{example}


%--------------------------------------------------------------------------------------
\section{\CMD{psParallelLine}}
%--------------------------------------------------------------------------------------
With this macro it is possible to plot lines relative to a given one, which is parallel. There is no special parameter here.
\begin{verbatim}
\psParallelLine(<P0>)(<P1>)(<P2>){<length>}{<end node name>}
\psParallelLine{<arrows>}(<P0>)(<P1>)(<P2>){<length>}{<end node name>}
\psParallelLine[<options>](<P0>)(<P1>)(<P2>){<length>}{<end node name>}
\psParallelLine[<options>]{<arrows>}(<P0>)(<P1>)(<P2>){<length>}{<end node name>}
\end{verbatim}

The line starts at $P_2$, is parallel to $\overline{P_0P_1}$ and the length of this
parallel line depends to the length factor. The end node name must
be a valid nodename and shouldn't contain any of the special PostScript characters.

\begin{example}
\begin{pspicture*}(-5,-4)(5,3.5)
  \psgrid[subgriddiv=0,griddots=5]
  \pnode(2,-2){FF}\qdisk(FF){1.5pt}
  \pnode(-5,5){A}\pnode(0,0){O}
  \multido{\nCountA=-2.4+0.4}{9}{%
    \psParallelLine[linecolor=red](O)(A)(0,\nCountA){9}{P1}
    \psline[linecolor=red](0,\nCountA)(FF)
    \psRelLine[linecolor=red](0,\nCountA)(FF){9}{P2}
  }
  \psline[linecolor=blue](A)(FF)
  \psRelLine[linecolor=blue](A)(FF){5}{END1}
  \rput(0,0){%
     \psline[linewidth=1pt](xLeft)(xRight)
     \psline[linewidth=2pt,arrows=->](F')(FF)
  }%
\end{pspicture*}
\end{example}


%--------------------------------------------------------------------------------------
\section{\CMD{psIntersectionPoint}}
%--------------------------------------------------------------------------------------
This macro calculates the intersection point of two lines, given by the four coordinates.
There is no special parameter here.
\begin{verbatim}
\psIntersectionPoint(<P0>)(<P1>)(<P2>)(<P3>){<node name>}
\end{verbatim}

\begin{example}[width=5.5cm]
\psset{unit=0.5cm}
\begin{pspicture}(-5,-4)(5,5)
  \psaxes{->}(0,0)(-5,-4)(5,5)
  \psline[linecolor=red,linewidth=2pt](-5,-1)(5,5)
  \psline[linecolor=blue,linewidth=2pt](-5,3)(5,-4)
  \qdisk(-5,-1){3pt}\uput[-90](-5,-1){A}
  \qdisk(5,5){3pt}\uput[-90](5,5){B}
  \qdisk(-5,3){3pt}\uput[-90](-5,3){C}
  \qdisk(5,-4){3pt}\uput[-90](5,-4){D}
  \psIntersectionPoint(-5,-1)(5,5)(-5,3)(5,-4){IP}
  \qdisk(IP){5pt}\uput{0.3}[90](IP){IP}
  \psline[linestyle=dashed](IP|0,0)(IP)(0,0|IP)
\end{pspicture}
\end{example}

%--------------------------------------------------------------------------------------
\section{\CMD{psLNode} and \CMD{psLCNode}}
%--------------------------------------------------------------------------------------
\CMD{psLNode} interpolates the Line $\overline{AB}$ by the given value and sets a node at this
point. The syntax is
%
\begin{verbatim}
\setLNode(P1)(P2){value}{Node name}
\end{verbatim}

\begin{example}[width=5cm]
\begin{pspicture}(5,5)
\psgrid[subgriddiv=0,griddots=10]
\psset{linecolor=red}
\psline{o-o}(1,1)(5,5)
\setLNode(1,1)(5,5){0.75}{PI}
\qdisk(PI){4pt}
\psset{linecolor=blue}
\psline{o-o}(4,3)(2,5)
\setLNode(4,3)(2,5){-0.5}{PII}
\qdisk(PII){4pt}
\end{pspicture}
\end{example}


\bigskip
The \CMD{psLCNode} macro builds the linear combination of the two given vectors and stores the end of
the new vector as a node. All vectors start at $(0,0)$, so a \verb+\rput+ maybe appropriate.
 The syntax is
%
\begin{verbatim}
\setLCNode(P1){value 1}(P2){value 2}{Node name}
\end{verbatim}


\begin{example}[width=5cm]
\begin{pspicture}(5,5)
\psgrid[subgriddiv=0,griddots=10]
\psset{linecolor=black}
\psline[linestyle=dashed]{->}(3,1.5)
\psline[linestyle=dashed]{->}(0.375,1.5)
\psset{linecolor=red}
\psline{->}(2,1)\psline{->}(0.5,2)
\setLCNode(2,1){1.5}(0.5,2){0.75}{PI}
\psline[linewidth=2pt]{->}(PI)
\psset{linecolor=black}
\psline[linestyle=dashed](3,1.5)(PI)
\psline[linestyle=dashed](0.375,1.5)(PI)
\end{pspicture}
\end{example}


%--------------------------------------------------------------------------------------
\section{\CMD{ncline}, \CMD{pcline} and inside errors}
%--------------------------------------------------------------------------------------

% Lines (\pcline macro)
\begin{example}[width=2.5cm]
\begin{pspicture}(2,1)
\psset{arrowscale=2}
\Cnode(0,0){A}\Cnode(2,1){B}
\ncline[ArrowInside=->]{A}{B}
\end{pspicture}
\end{example}


\begin{example}[width=2.5cm]
\begin{pspicture}(2,1)
\psset{arrowscale=2}
\Cnode(0,0){A}\Cnode(2,1){B}
\pcline[ArrowInside=->]{<->}(A)(B)
\end{pspicture}
\end{example}


\begin{example}[width=2.5cm]
\begin{pspicture}(2,1)
\psset{arrowscale=2}
\Cnode(0,0){A}\Cnode(2,1){B}
\ncline[ArrowInside=-|,ArrowInsidePos=0.75]{|-|}{A}{B}
\end{pspicture}
\end{example}


\begin{example}[width=2.5cm]
\psset{arrowscale=2}
\Cnode(0,0){A}\Cnode(2,1){B}
\pcline[ArrowInside=->,ArrowInsidePos=0.65]{*-*}(A)(B)
\naput[labelsep=0.3]{\large$g$}
\end{example}


\begin{example}[width=2.5cm]
\psset{arrowscale=2}
\Cnode(0,0){A}\Cnode(2,0){B}
\ncline[ArrowInside=->,ArrowInsidePos=10]{|-|}{A}{B}
\naput[labelsep=0.3]{\large$l$}
\end{example}


%--------------------------------------------------------------------------------------
\section{\CMD{nccurve}, \CMD{pccurve} and inside errors}
%--------------------------------------------------------------------------------------

% Lines (\pcline macro)
\begin{example}[width=2.5cm]
\begin{pspicture}(2,1)
\psset{arrowscale=2}
\Cnode(0,0){A}\Cnode(2,1){B}
\nccurve[ArrowInside=->,angleA=-90,angleB=90]{A}{B}
\end{pspicture}
\end{example}


\begin{example}[width=2.5cm]
\begin{pspicture}(2,1)
\psset{arrowscale=2}
\Cnode(0,0){A}\Cnode(2,1){B}
\pccurve[ArrowInside=->,angleA=-90,angleB=90]{<->}(A)(B)
\end{pspicture}
\end{example}


\begin{example}[width=2.5cm]
\begin{pspicture}(2,1)
\psset{arrowscale=2}
\Cnode(0,0){A}\Cnode(2,1){B}
\nccurve[ArrowInside=-|,ArrowInsidePos=0.75,angleA=-90,angleB=90]{|-|}{A}{B}
\end{pspicture}
\end{example}


\begin{example}[width=2.5cm]
\psset{arrowscale=2}
\Cnode(0,0){A}\Cnode(2,1){B}
\pccurve[ArrowInside=->,ArrowInsidePos=0.65,angleA=-90,angleB=90]{*-*}(A)(B)
\naput[labelsep=0.3]{\large$g$}
\end{example}


\begin{example}[width=2.5cm]
\psset{arrowscale=2}
\Cnode(0,0){A}\Cnode(2,0){B}
\nccurve[ArrowInside=->,ArrowInsidePos=10,angleA=-90,angleB=90]{|-|}{A}{B}
\naput[labelsep=0.3]{\large$l$}
\end{example}



\vspace{0.5cm}
%--------------------------------------------------------------------------------------
\section{\CMD{resetPSTNodeOptions}}
%--------------------------------------------------------------------------------------

Sometimes it is difficult to know what options which are changed inside a long document
are different to the default one. With this
macro all options depending to \verb+pst-plot+ can be reset. This depends to all
options of the package \verb+pst-plot+.

\begin{lstlisting}
\def\resetPSTNodeOptions{%
\psset{%
  nodealign=false,%
  href=0,%
  vref=.7ex,%
  framesize=10pt,%
  nodesep=0pt,%
  arm=10pt,%
  offset=0pt,%
  angle=0,%
  arcangle=8,%
  ncurv=.67,%
  loopsize=1cm,%
  boxsize=.4cm,%
  nrot=0,%
  npos=,%
  tpos=0.5,%
  shortput=none,%
  colsep=1.5cm,%
  rowsep=1.5cm,%
%  shortput=tablr,%%
  mcol=c,%
  mnode=R,%
  emnode=none}
}
\end{lstlisting}

%--------------------------------------------------------------------------------------
\section{Credits}
%--------------------------------------------------------------------------------------
%{Hendri Adriaans | } 
%{Ulrich Dirr | } 
%{Hubert G��lein |}
%{Denis Girou | } 
%{Peter Hutnick | } 
%{Christophe Jorssen | } 
%{Manuel Luque | } 
%{Jens-Uwe Morawski |}
%{Tobias N�hring |}
%{Rolf Niepraschk |}
%{Dominique Rodriguez |}
%{Timothy Van Zandt}



\nocite{*}
\bibliographystyle{plain}
\bibliography{pstricks}


