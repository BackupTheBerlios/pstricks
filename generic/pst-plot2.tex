%% BEGIN pst-plot.tex
%%
%% Plots and axes with PSTricks
%% See the PSTricks User's Guide for documentation.
%%
%%
%% COPYRIGHT 1993, 1994, 1999 by Timothy Van Zandt, <tvz@nwu.edu>.
%% COPYRIGHT 2004. 2005 by Herbert Voss, <voss _at_ PSTricks.de>.
%%
%% This program can be redistributed and/or modified under the terms
%% of the LaTeX Project Public License Distributed from CTAN
%% archives in directory macros/latex/base/lppl.txt.
%%

\csname PSTplotLoaded\endcsname
\let\PSTplotLoaded\endinput
\ifx\PSTricksLoaded\endinput\else\input pstricks \fi
\ifx\MultidoLoaded\endinput\else\input multido \fi
\ifx\PSTXKeyLoaded\endinput\else\input pst-xkey \fi
%
\def\fileversion{2.00}
\def\filedate{2005/03/14}
\message{ v\fileversion, \filedate}
\message{`pst-plot2' v\fileversion, \filedate\space (tvz,hv)}
%
\edef\PstAtCode{\the\catcode`\@} \catcode`\@=11\relax
\SpecialCoor
\pst@addfams{pst-plot}
%
%% prologue for postcript
\pstheader{pst-eqdf.pro}%
%
\newcount\@zero\@zero=0\relax
%
\newif\ifPst@plot@comma
\define@key[psset]{pst-plot}{comma}[true]{\@nameuse{Pst@plot@comma#1}}
%
\newif\ifPst@plot@xAxis
\newif\ifPst@plot@yAxis
\newif\ifPst@plot@xyAxes
\define@key[psset]{pst-plot}{xAxis}[true]{\@nameuse{Pst@plot@xAxis#1}}
\define@key[psset]{pst-plot}{yAxis}[true]{\@nameuse{Pst@plot@yAxis#1}}
\define@key[psset]{pst-plot}{xyAxes}[true]{\psset{xAxis=true,yAxis=true}}%
%
\define@key[psset]{pst-plot}{xyDecimals}{\def\psk@xDecimals{#1}\def\psk@yDecimals{#1}}
\define@key[psset]{pst-plot}{xDecimals}{\def\psk@xDecimals{#1}}
\define@key[psset]{pst-plot}{yDecimals}{\def\psk@yDecimals{#1}}
%
\define@key[psset]{pst-plot}{xyLabel}{\def\psk@xLabel{#1}\def\psk@yLabel{#1}}
\define@key[psset]{pst-plot}{xLabel}{\def\psk@xLabel{#1}}
\define@key[psset]{pst-plot}{yLabel}{\def\psk@yLabel{#1}}
%
\define@key[psset]{pst-plot}{xlogBase}{\def\psk@xlogBase{#1}}
\define@key[psset]{pst-plot}{ylogBase}{\def\psk@ylogBase{#1}}
\define@key[psset]{pst-plot}{xylogBase}{\def\psk@xlogBase{#1}\def\psk@ylogBase{#1}}%
%
%logLines=all|x|y|none  (0,1,2,3)
\define@key[psset]{pst-plot}{logLines}{\pst@expandafter\logLines@ii{#1}\@nil\psk@logLines}
\def\logLines@ii#1#2\@nil#3{%
  \ifx#1a\let#3\z@\else
    \ifx#1x\let#3\@ne\else
      \ifx#1y\let#3\tw@\else
        \ifx#1n\let#3\thr@@\else
          \@pstrickserr{Bad argument: `#1#2'}\@ehpa
  \fi\fi\fi\fi}
%
\define@key[psset]{pst-plot}{ylabelFactor}{\def\psk@ylabelFactor{#1}}
\define@key[psset]{pst-plot}{xlabelFactor}{\def\psk@xlabelFactor{#1}}
%
\newif\ifPst@trigLabels
\define@key[psset]{pst-plot}{trigLabels}[true]{%
  \@nameuse{Pst@trigLabels#1}%
  \ifPst@trigLabels
    \def\pst@@@hlabel##1{\pshlabel{##1}}
    \def\pshlabel##1{$%
      \pst@mod{##1}{2}\pst@cnta%                   0 oder 1
      \ifnum1>\pst@cnta\relax%                    ungerade?
        \pst@cntb=##1\divide\pst@cntb by 2       % #1/2
          \ifnum\pst@cntb>1 \the\pst@cntb\fi\pi % n*pi
      \else\frac{\ifnum##1>1 ##1\fi\pi}{2}\fi$}%                  n/2*pi
  \else
    \def\pst@@@hlabel##1{%
      \edef\@xyDecimals{\psk@xDecimals}%
      \ifnum\psk@ticks<\tw@ % ticks=all|x
        \ifx\psk@xlogBase\@empty
          \pshlabel{\expandafter\@LabelComma##1..\@nil\psk@xlabelFactor}%
        \else%
          \pshlabel{\psk@xlogBase\textsuperscript{\expandafter\@stripDecimals##1..\@nil }}
        \fi%
      \fi%
    }%
    \def\pshlabel##1{$##1$}%
  \fi%
}
%
% Using lists of data is optimized for \dataplot and \fileplot
% Here is the tricky part. As each line is read from file,
% we want to ignore trailing delimiters, and convert arbitrary
% strings of non-trailing delimiters to _D_.
% We end up with
%   D x1 D y1 D x2 D y2 ... D xn D yn
%
\begingroup
\catcode`\{=13
\catcode`\}=13
\catcode`\(=13
\catcode`\)=13
\catcode`\,=13
\catcode`\!=1
\catcode`\*=2
\catcode`\ =13
\catcode`\_=13
\catcode`\^^M=13
\gdef\pst@datadelimiters!% Begin def
\catcode`\{=13%
\catcode`\}=13%
\catcode`\(=13%
\catcode`\)=13%
\catcode`\,=13%
\catcode`\ =13%
\catcode`\^^M=13%
\def,##1!%
\ifcat\noexpand,\noexpand##1%
\expandafter##1%
\else\space%
D\space##1%
\fi*%
\let(,\let),\let{,\let},\let ,\let^^M,\let_\@empty*% End def
\endgroup
%
\define@key[psset]{pst-plot}{nStep}{\edef\psk@nStep{#1}}
\define@key[psset]{pst-plot}{nStart}{\edef\psk@nStart{#1}}
\define@key[psset]{pst-plot}{nEnd}{\edef\psk@nEnd{#1}}
\define@key[psset]{pst-plot}{xStep}{\edef\psk@xStep{#1}}
\define@key[psset]{pst-plot}{yStep}{\edef\psk@yStep{#1}}
%
\define@key[psset]{pst-plot}{xStart}{\edef\psk@xStart{#1}}
\define@key[psset]{pst-plot}{xEnd}{\edef\psk@xEnd{#1}}
\define@key[psset]{pst-plot}{yStart}{\edef\psk@yStart{#1}}
\define@key[psset]{pst-plot}{yEnd}{\edef\psk@yEnd{#1}}
%
\define@key[psset]{pst-plot}{plotNo}{\edef\psk@plotNo{#1}}
\define@key[psset]{pst-plot}{plotNoMax}{\edef\psk@plotNoMax{#1}}
%
\define@key[psset]{pst-plot}{xAxisLabel}{\edef\psk@xAxisLabel{#1}}
\define@key[psset]{pst-plot}{yAxisLabel}{\edef\psk@yAxisLabel{#1}}
\define@key[psset]{pst-plot}{xAxisLabelPos}{\edef\psk@xAxisLabelPos{#1}}
\define@key[psset]{pst-plot}{yAxisLabelPos}{\edef\psk@yAxisLabelPos{#1}}
%
\newcount\linecnt
\begingroup
\catcode`\,=13
\catcode`\_=13
\gdef\savedata@#1[#2]{%
  \xdef\pst@tempg{#2_}%
  \endgroup
  \let#1\pst@tempg
  \global\let\pst@tempg\relax
  \ignorespaces}
\gdef\readdata@{%
  \read1 to \pst@tempa
  \ifnum\linecnt=\psk@nStep
    \global\linecnt=0
    \expandafter\readdata@@\pst@tempa_\@nil
  \fi
  \global\advance\linecnt by 1
  \ifeof1\else\expandafter\readdata@\fi}
\gdef\pst@@readfile#1#2\@nil{\addto@pscode{,#1#2}}%
\gdef\readdata@@#1#2\@nil{\xdef\pst@tempg{\pst@tempg,#1#2}}%
\endgroup
%
\def\readdata{\@ifnextchar[{\readdata@i}{\readdata@i[]}}
\def\readdata@i[#1]#2#3{{%
  \def\pst@tempa{#1}%
  \ifx\pst@tempa\@empty\else\psset{#1}\fi
  \openin1=#3
  \begingroup
  \edef\pst@tempg{}%
  \ifeof1
    \@pstrickserr{Data file `#3' not found.}\@ehpa
  \else
    \pst@datadelimiters
    \catcode`\[=1
    \catcode`\]=2
    \global\linecnt=\psk@nStep
    \readdata@%
  \fi
  \endgroup
  \global\let#2\pst@tempg
  \global\let\pst@tempg\relax
  \ignorespaces%
}\ignorespaces}
%
\def\pst@readfile#1{{\let\readdata@@\pst@@readfile\readdata\pst@tempg{#1}}}
\def\pst@altreadfile#1{%
  \openin1=#1
  \ifeof1
    \@pstrickserr{Data file `#1' not found.}\@ehpa
  \else
    \catcode`\{=10
    \catcode`\}=10
    \catcode`\(=10
    \catcode`\)=10
    \catcode`\,=10
    \catcode`\^^M=10
    \catcode`\[=1
    \catcode`\]=2
    \pst@@altreadfile
  \fi}
\def\pst@@altreadfile{%
  \read1 to \pst@tempg
  \expandafter\pst@@@altreadfile\pst@tempg\@empty\@nil
  \ifeof1\else\expandafter\pst@@@altreadfile\fi}
\def\pst@@@altreadfile#1#2\@nil{\addto@pscode{#1#2}}%

\def\savedata#1{\begingroup\pst@datadelimiters\savedata@{#1}}

\def\beginplot@line{\begin@OpenObj}
\def\endplot@line{\psline@ii}
\def\beginplot@polygon{\begin@ClosedObj}
\def\endplot@polygon{\pspolygon@ii}
\def\beginplot@curve{\begin@OpenObj}
\def\endplot@curve{\pscurve@ii}
\def\beginplot@ecurve{\begin@OpenObj}
\def\endplot@ecurve{\psecurve@ii}
\def\beginplot@ccurve{\begin@ClosedObj}
\def\endplot@ccurve{\psccurve@ii}
\def\beginplot@dots{\begin@SpecialObj}
\def\endplot@dots{\psdots@ii}
\def\beginplot@bezier{\begin@OpenObj}
\def\endplot@bezier{\psbezier@ii}
\def\beginplot@cbezier{\begin@ClosedObj}
\def\endplot@cbezier{\pscbezier@ii}

\define@key[psset]{pst-plot}{plotstyle}{%
  \@ifundefined{beginplot@#1}%
    {\@pstrickserr{Plot style `#1' not defined}\@eha}%
    {\edef\psplotstyle{#1}}}
\define@key[psset]{pst-plot}{plotpoints}{%
  \pst@cntg=#1\relax
  \ifnum\pst@cntg<2
    \@pstrickserr{plotpoints parameter must be at least 2}\@ehpa
  \else
    \advance\pst@cntg-1
    \edef\psk@plotpoints{\the\pst@cntg\space}%
  \fi}
%
% For quick plots, define:
%   \beginqp@<foo>   : What to do to first point (PS code only).
%   \doqp@<foo>      : What to do to subsequent points (PS code only).
%   \endqp@<foo>     : How to end plot.
%   \testqp@<foo>    : Set \@psttrue if OK to use quick plot.

\def\beginqp@line{\pst@oplineto}
\def\doqp@line{L }
\def\endqp@line{\end@OpenObj}%
\def\testqp@line{%
  \ifdim\pslinearc>\z@\else
    \ifshowpoints\else
      \ifx\psk@arrowA\@empty
        \ifx\psk@arrowB\@empty
          \@psttrue
        \fi
      \fi
    \fi
  \fi}

\def\beginqp@polygon{moveto }
\def\doqp@polygon{L }
\def\endqp@polygon{%
  \addto@pscode{closepath}%
  \end@ClosedObj}
\def\testqp@polygon{%
  \ifdim\pslinearc>\z@\else
    \ifshowpoints\else
      \@psttrue
    \fi
  \fi}

\def\beginqp@dots{%
  \psk@dotsize
  \@nameuse{psds@\psk@dotstyle}
% DG/SR modification begin - Dec. 12, 1999 - Patch 2
%  /TheDot { gsave \psk@dotangle \psk@dotscale Dot grestore } def
%  TheDot }
  Dot }
%\def\doqp@dots{TheDot }
\def\doqp@dots{Dot }
% DG/SR modification end
\def\endqp@dots{\end@SpecialObj}
\def\testqp@dots{\@psttrue}

\def\beginqp@bezier{/n 0 def \pst@oplineto}
\def\doqp@bezier{/n n 1 add def n 3 mod 0 eq { curveto } if }
\def\endqp@bezier{%
  \addto@pscode{n 3 mod { pop pop } repeat}
  \end@OpenObj}%
\def\testqp@bezier{%
  \ifshowpoints\else
    \ifx\psk@arrowA\@empty
      \ifx\psk@arrowB\@empty
        \@psttrue
      \fi
    \fi
  \fi}

\def\beginqp@cbezier{/n 0 def moveto }
\def\doqp@cbezier{\doqp@bezier}
\def\endqp@cbezier{%
  \addto@pscode{n 3 mod { pop pop } repeat closepath}
  \end@ClosedObj}%
\def\testqp@cbezier{\ifshowpoints\else\@psttrue\fi}
%
%
\def\dataplot{\def\pst@par{}\pst@object{dataplot}}
\def\dataplot@i#1{%
  \pst@killglue
  \begingroup
    \use@par
    \@pstfalse
    \@nameuse{testqp@\psplotstyle}%
    \if@pst
      \dataplot@ii{\addto@pscode{#1}}%
    \else
      \listplot@ii{\addto@pscode{#1}}%
    \fi
  \endgroup
  \ignorespaces}
\def\dataplot@ii#1{%
  \@nameuse{beginplot@\psplotstyle}%
    \addto@pscode{%
      /Dx { \pst@number\psxunit mul /D { Dy } def } def
      /Dy { \pst@number\psyunit mul Do /D { Dx } def } def
      /D { /D { Dx } def } def
      /Do {
        \@nameuse{beginqp@\psplotstyle}%
        /Do { \@nameuse{doqp@\psplotstyle}} def
      } def}%
    #1%
    \addto@pscode{D}%
  \@nameuse{endqp@\psplotstyle}}

\def\fileplot{\def\pst@par{}\pst@object{fileplot}}
\def\fileplot@i#1{%
  \pst@killglue
  \begingroup
    \use@par
    \@pstfalse
    \@nameuse{testqp@\psplotstyle}%
    \if@pst
      \dataplot@ii{\pst@readfile{#1}}%
    \else
      \listplot@ii{\pst@altreadfile{#1}}%
    \fi
  \endgroup
  \ignorespaces}
%
\def\pstScalePoints(#1,#2)#3#4{%
%  xScale | yScale | xOperator | yOperator  
% the operators can be any Postscript code
  \pst@def{ScalePoints}<%
    /y ED /x ED
    counttomark dup dup cvi eq not { exch pop } if
    /m exch def /n m 2 div cvi def
    n {
      #4\space y mul #2\space mul m 1 roll
      #3\space x mul #1\space mul m 1 roll
      /m m 2 sub
      def } repeat>
}
%\pstScalePoints(1,1){}{}% the default -> no special operators
%
\pst@def{ScalePoints}<%
  /y ED /x ED
  counttomark dup dup cvi eq not { exch pop } if
  /m exch def /n m 2 div cvi def
  n { y mul m 1 roll x mul m 1 roll /m m 2 sub def } repeat>

\pst@def{PreparePoints}<{%
  counttomark /m exch def
  /n m \psk@plotNoMax\space 1 add div cvi def
  \psk@plotNoMax\space 1 gt {% multiple data files?
    n {
      \psk@plotNoMax\space \psk@plotNo\space 1 sub neg roll % x yNo y y y ...
      \psk@plotNoMax\space 1 sub { pop } repeat % x yNo
      /m m \psk@plotNoMax\space 1 sub sub def
      m 2 roll
    } repeat
  } if % no multiple data files
%	counttomark /m exch def
%	/n m 2 div cvi def
  /xMax -99999 def /yMax -99999 def
  /xP 0 def /yP 0 def
  m copy
  n {
    /y exch def /x exch def
    xMax x lt { /xMax x def } if
    yMax y lt {/yMax y def } if
    xP x gt { /xP x def } if
    yP y gt { /yP y def } if
  } repeat
%	m 2 roll
  \psk@xStep\space 0 gt \psk@yStep\space 0 gt or (\psk@xStart) length 0 gt or
  (\psk@yStart) length 0 gt or (\psk@xEnd) length 0 gt or (\psk@yEnd) length 0 gt or {
%
    (\psk@xStart) length 0 gt {\psk@xStart\space }{ xP } ifelse /xStart exch def
    (\psk@yStart) length 0 gt {\psk@yStart\space }{ yP } ifelse /yStart exch def
    (\psk@xEnd) length 0 gt { \psk@xEnd\space }{ xMax } ifelse /xEnd exch def
    (\psk@yEnd) length 0 gt { \psk@yEnd\space }{ yMax } ifelse /yEnd exch def
    n {
      m -2 roll
      2 copy /yVal exch def /xVal exch def
      xVal xP ge
      yVal yP ge and
      xVal xEnd le and
      yVal yEnd le and
      xVal xStart ge and
      yVal yStart ge and {
        /xP xP \psk@xStep\space add def
        /yP yP \psk@yStep\space add def
      }{%
        pop pop
        /m m 2 sub def
      } ifelse
    } repeat
  }{%
    /ncount 0 def
    (\psk@nEnd) length 0 gt { \psk@nEnd\space }{ m } ifelse 
    /nEnd exch def
    n {
      m -2 roll
      \psk@nStep\space 1 gt {
        ncount \psk@nStart\space sub \psk@nStep\space mod 0 eq }{ true } ifelse
        ncount nEnd le and
        ncount \psk@nStart\space ge and not {
          pop pop
          /m m 2 sub def
        } if
        /ncount ncount 1 add def
      } repeat
  } ifelse
}>
%

\def\listplot{\def\pst@par{}\pst@object{listplot}}
\def\listplot@i#1{\listplot@ii{\addto@pscode{#1}}}
\def\listplot@ii#1{%
  \@nameuse{beginplot@\psplotstyle}%
  \addto@pscode{/D {} def mark}%
  #1%
  \addto@pscode{%
    \tx@PreparePoints
    \pst@number\psxunit
    \pst@number\psyunit
    \tx@ScalePoints%
  }%
  \@nameuse{endplot@\psplotstyle}%
}
%
%
% \psplot
%
% D.G. addition - Jun.  9, 1998 - Polar plots using the \psplot macro
% Code added according the way suggested by Ulrich Dirr <udirr@das-team.com>
%
% For polar plots
\newif\ifpolarplot
\define@key[psset]{pst-plot}{polarplot}{\@nameuse{polarplot#1}}
%
\define@key[psset]{pst-plot}{method}{\edef\psk@method{#1}}%
\define@key[psset]{pst-plot}{whichabs}{\edef\psk@whichabs{#1}}%
\define@key[psset]{pst-plot}{whichord}{\edef\psk@whichord{#1}}%
\define@key[psset]{pst-plot}{plotfuncx}{\edef\psk@plotfuncx{#1}}%
\define@key[psset]{pst-plot}{plotfuncy}{\edef\psk@plotfuncy{#1}}%
\define@key[psset]{pst-plot}{expression}{\edef\psk@expression{#1}}%
\define@key[psset]{pst-plot}{VarStep}{\edef\psk@VarStep{#1}}%
%
\newif\ifPst@buildvector%
\define@key[psset]{pst-plot}{buildvector}[true]{\@nameuse{Pst@buildvector#1}}%
%\newif\ifPst@algebraic%
%\define@key[psset]{pst-plot}{algebraic}[false]{\@nameuse{Pst@algebraic#1}}%
%
\def\@rkiv{rk4}%
\def\@adams{adams}%
\def\@undef{undef}%
\def\@default{default}%
%
% none= 0 (postfix) , TeX=1 (conversion by TeX), PS=-1 (conversion by PostScript)
\define@key[psset]{pst-plot}{infix}{\pst@expandafter\pstinfix@i{#1}\@nil}
\def\pstinfix@i#1#2\@nil{%
  \ifx#1n\let\psk@infix\z@\else
    \ifx#1T\let\psk@infix\@ne\else
      \ifx#1P\let\psk@infix\m@ne\else
        \@pstrickserr{Bad infix modus: `#1#2'}\@ehpa
  \fi\fi\fi}
%
\def\psplotinit#1{\xdef\psplot@init{#1 }}
\def\psplot@init{}
%
\def\psplot{\def\pst@par{}\pst@object{psplot}}
\def\psplot@i#1#2#3{%
  \pst@killglue
  \begingroup
    \use@par
    \@nameuse{beginplot@\psplotstyle}%
% D.G. modification begin - Jun.  9, 1998
    \ifpolarplot
      \addto@pscode{%
        \psplot@init
        /x #1 def
        /x1 #2 def
        /dx x1 x sub \psk@plotpoints div def
        /F@pstplot \ifnum\psk@infix<\z@\space Pst@algebraic (#3) tx@EqdfDict begin AlgParser end cvx
            \else { #3 } \fi  def 
        /xy {% Adapted from \parametricplot@i
          F@pstplot dup x cos mul exch x sin mul
%          #3 dup x cos mul exch x sin mul
          \pst@number\psxunit mul exch
          \pst@number\psyunit mul exch
        } def}%
    \else
% D.G. modification end
    \addto@pscode{%
      \psplot@init
      /x #1 def
      /x1 #2 def
      /dx x1 x sub \psk@plotpoints div def
      /F@pstplot \ifnum\psk@infix<\z@\space (#3) tx@EqdfDict begin AlgParser end cvx
                 \else { #3 } \fi  def 
      /xy {
        x \pst@number\psxunit mul F@pstplot \pst@number\psyunit mul
%       \ifPst@algebraic F@pstplot \else #3 \fi \pst@number\psyunit mul
      } def}%
    \fi
    \gdef\psplot@init{}%
    \@pstfalse
    \@nameuse{testqp@\psplotstyle}%
    \if@pst
      \psplot@ii
    \else
      \psplot@iii
    \fi
  \endgroup
  \ignorespaces}
\def\psplot@ii{%
  \ifx\psk@VarStep\@default
    \addto@pscode{%
      xy \@nameuse{beginqp@\psplotstyle}
      \psk@plotpoints 1 sub {
        /x x dx add def
        xy \@nameuse{doqp@\psplotstyle}
      } repeat
      /x x1 def
      xy \@nameuse{doqp@\psplotstyle}}%
  \else
    \addto@pscode{%
      xy \@nameuse{beginqp@\psplotstyle}
      /Olddt dx def /OldOlddt dx def /Oldy F@pstplot def /y 0 def
      /Ndt dx def /Ny 0 def
      { /Ndt Ndt 1.5 mul def %%new step is 50% larger
        x1 x sub dup Ndt lt { /Ndt exch def } { pop } ifelse 
        { /x x /Ny /x x Ndt add def tx@EqdfDict begin F@pstplot end def def
          /x x /y /x x Ndt 2 div add def tx@EqdfDict begin F@pstplot end def def
              /Oldy tx@EqdfDict begin F@pstplot end def
         Ny y sub Ndt div y Oldy sub Ndt div dup 0 eq { pop pop exit } if div abs 
         dup 1 gt { 1 exch div } if \psk@VarStep\space gt%%if greater than rate OK
         { exit } { /Ndt Ndt 2 div def } ifelse
          Ndt dx lt { exit } if
       } loop
        /x x Ndt add def xy \@nameuse{doqp@\psplotstyle}
        x x1 eq { exit } if
        /Olddt Ndt def
        /OldOlddt Olddt def
       /y Ny def} loop}%
  \fi
  \@nameuse{endqp@\psplotstyle}}
\def\psplot@iii{%
  \ifx\psk@VarStep\@default
    \addto@pscode{%
      mark
      /n 2 def
      \psk@plotpoints {
        xy
        n 2 roll
        /n n 2 add def
        /x x dx add def
      } repeat
      /x x1 def
      xy
      n 2 roll}%
  \else\pst@killglue
    \addto@pscode{%
      /Olddt dx def /OldOlddt dx def /Oldy F@pstplot def /y 0 def
      /Ndt dx def /Ny 0 def /n 2 def
      mark xy
      { /Ndt Ndt 1.5 mul def %%new step is 50% larger
        x1 x sub dup Ndt lt { /Ndt exch def } { pop } ifelse 
        { /x x /Ny /x x Ndt add def tx@EqdfDict begin F@pstplot end def def
          /x x /y /x x Ndt 2 div add def tx@EqdfDict begin F@pstplot end def def
              /Oldy tx@EqdfDict begin F@pstplot end def
         Ny y sub Ndt div y Oldy sub Ndt div dup 0 eq { pop pop exit } if div abs 
         dup 1 gt { 1 exch div } if \psk@VarStep\space gt%%if greater than rate OK
         { exit } { /Ndt Ndt 2 div def } ifelse
          Ndt dx lt { exit } if
       } loop
        /x x Ndt add def xy /n n 2 add def n 2 roll
        x x1 eq { exit } if
        /Olddt Ndt def
        /OldOlddt Olddt def
       /y Ny def} loop}%
  \fi
  \@nameuse{endplot@\psplotstyle}}
\def\parametricplot@i#1#2#3{%
  \pst@killglue%
  \begingroup
    \use@par
    \@nameuse{beginplot@\psplotstyle}%
    \addto@pscode{%
      \psplot@init
      /t #1 def
      /t1 #2 def
      /dt t1 t sub \psk@plotpoints div def
      \ifnum\psk@infix<\z@\space /F@pstplot (#3) tx@EqdfDict begin AlgParser end cvx def \fi
      /xy {
        \ifnum\psk@infix<\z@\space  F@pstplot \else #3 \fi
        \pst@number\psyunit mul exch
        \pst@number\psxunit mul exch
      } def}%
    \gdef\psplot@init{}%
    \@pstfalse
    \@nameuse{testqp@\psplotstyle}%
    \if@pst
      \parametricplot@ii
    \else
      \parametricplot@iii
    \fi
  \endgroup
  \ignorespaces}
%
\def\parametricplot@ii{%
    \addto@pscode{%
      xy \@nameuse{beginqp@\psplotstyle}
      \psk@plotpoints 1 sub {
        /t t dt add def
        xy \@nameuse{doqp@\psplotstyle}
      } repeat
      /t t1 def
      xy \@nameuse{doqp@\psplotstyle}}%
  \@nameuse{endqp@\psplotstyle}}
\def\parametricplot@iii{%
    \addto@pscode{%
      mark
      /n 2 def
      \psk@plotpoints {
        xy
        n 2 roll
        /n n 2 add def
        /t t dt add def
      } repeat
      /t t1 def
      xy
      n 2 roll}%
  \@nameuse{endplot@\psplotstyle}}

% These axes macros are complicated. Be careful.

% \pst@ticks{angle}{dx}{n}{int}
% int=1 if ticks appear on top of axes, 0 otherwise.
\def\pst@ticks#1#2#3#4{%
  \begin@SpecialObj
    \addto@pscode{%
      #1 rotate
      /n #3 def
      /dx #2 def
      n 0 lt { /dx dx neg def /n n neg def } if
      /y2 \psk@ticksize CLW 2 div add def
      /y1 y2 neg def
      \ifnum\psk@tickstyle=1
        \ifdim#4<\z@ /y2 \else /y1 \fi 0 def
      \else
        \ifnum\psk@tickstyle=-1
          \ifdim#4<\z@ /y1 \else /y2 \fi 0 def
        \fi
      \fi
      /x dx def
      n { x y1 moveto x y2 lineto stroke /x x dx add def } repeat}%
  \end@SpecialObj}
%
\define@key[psset]{pst-plot}{ticks}{\pst@expandafter\ticks@i{#1}\@nil\psk@ticks}
\def\ticks@i#1#2\@nil#3{%
  \ifx#1a\let#3\z@\else
    \ifx#1x\let#3\@ne\else
      \ifx#1y\let#3\tw@\else
        \ifx#1n\let#3\thr@@\else
          \@pstrickserr{Bad argument: `#1#2'}\@ehpa
  \fi\fi\fi\fi}
%
\define@key[psset]{pst-plot}{labels}{\pst@expandafter\ticks@i{#1}\@nil\psk@labels}
%
\define@key[psset]{pst-plot}{Ox}{\def\psk@Ox{#1}}
\define@key[psset]{pst-plot}{Dx}{\def\psk@Dx{#1}}
\define@key[psset]{pst-plot}{dx}{%
  \pssetxlength\pst@dimg{#1}%
  \edef\psk@dx{\number\pst@dimg}}
%
\define@key[psset]{pst-plot}{Oy}{\def\psk@Oy{#1}}
\define@key[psset]{pst-plot}{Dy}{\def\psk@Dy{#1}}
\define@key[psset]{pst-plot}{dy}{%
  \pssetylength\pst@dimg{#1}%
  \edef\psk@dy{\number\pst@dimg}}
%
\newif\ifshoworigin
\define@key[psset]{pst-plot}{showorigin}{\@nameuse{showorigin#1}}
%%-------------- hv patch 3 2004-05-23 --------------------------
% to get axesstyle=dashed run
\def\pst@linetype{0}
%%---------------hv end ----------------------------------------
%
%% #1 integer
%% #2 decimals
%% #3 dot
\def\@stripDecimals#1.#2.#3\@nil{%
  \def\dummy{#1}%
  \ifx\dummy\@empty\the\@zero\else#1\fi% the integer part
}
%
\def\pst@@@vlabel#1{%
  \edef\@xyDecimals{\psk@yDecimals}%
  \psk@yLabel%
  \ifodd\psk@ticks  % ticks=all||y (0,2)
  \else
    \ifx\psk@ylogBase\@empty
      \ifPst@plot@comma\psvlabel{\expandafter\@LabelComma#1..\@nil\psk@ylabelFactor}%
      \else\psvlabel{#1}%
      \fi%
    \else%
      {\psk@yLabel\psvlabel{\psk@ylogBase\textsuperscript{\expandafter\@stripDecimals#1..\@nil }}}
    \fi%
  \fi%
}
\newcount\@digitcounter\@digitcounter=0\relax
\def\@inc@digitcounter{\global\advance\@digitcounter by 1\relax}
\def\@get@digitcounter{\the\@digitcounter\relax}
\def\@Reset@digitcounter{\global\@digitcounter=0\relax}
\def\@zeroFill{%
  \ifnum \@xyDecimals>\@get@digitcounter
    \bgroup
      0\@inc@digitcounter\@zeroFill
    \egroup%
  \fi%
}
% #1 the value, maybe empty
%
\def\@process@digits#1#2;{%
	\ifx *#1\@zeroFill\else#1\@inc@digitcounter\@process@digits#2;\fi%
}
\def\@writeDecimals#1{%
  \ifx\@xyDecimals\@empty% take value as is
    \def\@tempa{#1}% write only if not empty
    \ifx\@tempa\@empty% write nothing
    \else
      \ifPst@plot@comma,\else.\fi%
      #1%
    \fi%
  \else% write only \xy@decimals
    \ifnum\@xyDecimals>\@zero
      \ifPst@plot@comma,\else.\fi
        \@Reset@digitcounter
        \@process@digits#1*;
      \fi%
  \fi%
}
%% #1 integer
%% #2 decimals
%% #3 dot
\def\@LabelComma#1.#2.#3\@nil{%
  \def\dummy{#1}%
  \ifx\dummy\@empty\the\@zero\else#1\fi% the integer part
  \def\dummy{#2}%
  \ifx\dummy\@empty\@writeDecimals{}\else\@writeDecimals{#2}\fi%
}
%
\def\psxs@axes{{%
  \ifPst@plot@xAxis\psxs@@axes\pst@dima\pst@dimb\pst@dimc\pst@dimd{}\fi
  \ifPst@plot@yAxis\psxs@@axes\pst@dima\pst@dimb\pst@dimc\pst@dimd{exch}\fi
}}
\def\psaxes{\pst@object{psaxes}}
\def\psaxes@i{\pst@getarrows\psaxes@ii}
\def\psaxes@ii(#1){\@ifnextchar({\psaxes@iii(#1)}{\psaxes@iv(0,0)(0,0)(#1)}}
\def\psaxes@iii(#1)(#2){\@ifnextchar({\psaxes@iv(#1)(#2)}{\psaxes@iv(#1)(#1)(#2)}}
%
\def\psaxes@iv(#1,#2)(#3,#4)(#5,#6){%
  \pst@killglue
  \begingroup
  \setbox\pst@hbox=\hbox\bgroup%
  \use@par%
  \pssetxlength\pst@dimg{#1}% o-x
  \pssetylength\pst@dimh{#2}% o-y
  \pssetxlength\pst@dima{#3}% bl-x
  \pssetylength\pst@dimb{#4}% bl-y
  \pssetxlength\pst@dimc{#5}% ur-x
  \pssetylength\pst@dimd{#6}% ur-y
    % If minimum values are negative in log mode, we modify Ox
    % (respectively Oy) if this was not done by the user
    % X axis labels (\psk@log = 0 or 1)
  \ifx\psk@xlogBase\@empty\else%
    \ifx\psk@Ox\@empty\else% we do nothing ... hv
      \ifdim\psk@Ox pt=\z@
        \ifdim#3pt<\z@
          \pssetxlength\pst@dimg{#3}% o-x
          \psset{Ox=#3}%
	\fi
      \fi%
    \fi%
  \fi%
% Y axis labels (\psk@log = 0 or 2)
  \ifx\psk@ylogBase\@empty\else%
    \ifx\psk@Ox\@empty\else
      \ifdim#4pt<\z@
        \ifdim\psk@Oy pt=\z@
          \pssetylength\pst@dimh{#4}% o-y
          \psset{Oy=#4}%
	\fi
      \fi
    \fi
  \fi
% Whole thing will be translated to origin:
  \advance\pst@dima-\pst@dimg % Dist. from bl-x to o-x
  \advance\pst@dimb-\pst@dimh % Dist. from bl-y to o-y
  \advance\pst@dimc-\pst@dimg % Dist. from ur-x to o-x
  \advance\pst@dimd-\pst@dimh % Dist. from ur-y to o-y
% Make lines/arrows or frame:
  \@nameuse{psxs@\psk@axesstyle}%  \psxs@axes or \psxs@frame
  \ifPst@plot@yAxis  
  \begingroup
    \ifdim\pst@dima=\z@\else\showoriginfalse\fi
    \ifnum\psk@dy=\z@
      \pst@dimg=\psk@Dy\psyunit
      \edef\psk@dy{\number\pst@dimg}%
    \fi%
    \ifPst@plot@xAxis\else\showorigintrue\fi%
    \pst@vlabels{\pst@dimd}{\psk@arrowB}{#3}{#5}%
    \ifPst@plot@xAxis\showoriginfalse\fi%
    \pst@vlabels{\pst@dimb}{\psk@arrowA}{#3}{#5}%
  \endgroup%
  \fi
  \ifPst@plot@xAxis  
  \begingroup%
    \ifdim\pst@dimb=\z@\else\showoriginfalse\fi%
    \ifnum\psk@dx=\z@%
      \pst@dimg=\psk@Dx\psxunit%
      \edef\psk@dx{\number\pst@dimg}%
    \fi%
    \ifPst@plot@yAxis\else\showorigintrue\fi%
    \pst@hlabels{\pst@dimc}{\psk@arrowB}{#4}{#6}%
    \ifPst@plot@yAxis\showoriginfalse\fi%
    \pst@hlabels{\pst@dima}{\psk@arrowA}{#4}{#6}%
  \endgroup%
  \fi
% Now close "\pst@hbox" (which is 0-dimensional), and put it at the origin.
  \egroup%
  \pssetxlength\pst@dimg{#1}%
  \pssetylength\pst@dimh{#2}%
  \leavevmode\psput@cartesian\pst@hbox%
  \ignorespaces%
  \endgroup
}
%
\newdimen\psk@subticksize\psk@subticksize=\z@
\newdimen\pst@xticksizeA
\newdimen\pst@xticksizeB
\newdimen\pst@xticksizeC
\newdimen\pst@yticksizeA
\newdimen\pst@yticksizeB
\newdimen\pst@yticksizeC
%
\define@key[psset]{pst-plot}{ticksize}{\psset{xticksize=#1,yticksize=#1}}
\define@key[psset]{pst-plot}{xticksize}{%
  \pst@expandafter\pst@getydimdim{#1} {\pst@missing} {} {}\@nil%
  \pst@xticksizeA=\the\pst@dimm%
  \pst@xticksizeB=\the\pst@dimn%
  \edef\psk@xticksize{\pst@number\pst@xticksizeA\space \pst@number\pst@xticksizeB}%
  \ifdim\pst@xticksizeA=0pt\pst@xticksizeC=\z@\else
    \pst@absdim{\pst@xticksizeA}\pst@dimm%
    \pst@absdim{\pst@xticksizeB}\pst@dimn%
    \pst@maxdim{\pst@dimm}{\pst@dimn}\pst@dimo%
    \pst@xticksizeC=\pst@dimo%
  \fi%
}
\define@key[psset]{pst-plot}{yticksize}{%  overwrites the definition in pstricks
  \pst@expandafter\pst@getxdimdim{#1} {\pst@missing} {} {}\@nil%
  \pst@yticksizeA=\the\pst@dimm%
  \pst@yticksizeB=\the\pst@dimn%
  \edef\psk@yticksize{\pst@number\pst@yticksizeA \pst@number\pst@yticksizeB}%
  \ifdim\pst@yticksizeA=0pt\pst@yticksizeC=\z@\else
    \pst@absdim{\pst@yticksizeA}\pst@dimm%
    \pst@absdim{\pst@yticksizeB}\pst@dimn%
    \pst@maxdim{\pst@dimm}{\pst@dimn}\pst@dimo%
    \pst@yticksizeC=\pst@dimo%
  \fi%
}
%
% full= 0, top=1, bottom=-1
\define@key[psset]{pst-plot}{tickstyle}{%
  \pst@expandafter\tickstyle@i{#1}\@nil%
  \ifnum\psk@tickstyle>\z@%
    \psset{xticksize=0 \pst@xticksizeB}%
    \psset{yticksize=0 \pst@yticksizeB}%
  \else\ifnum\psk@tickstyle<\z@%
%    \expandafter\psset@xticksize\expandafter{\pst@xticksizeA\space 0}%
%    \expandafter\psset@yticksize\expandafter{\pst@yticksizeA\space 0}%
    \psset{xticksize=\pst@xticksizeA\space 0}%
    \psset{yticksize=\pst@yticksizeA\space 0}%
  \fi\fi%
}
\def\tickstyle@i#1#2\@nil{%
  \ifx#1f\let\psk@tickstyle\z@\else
    \ifx#1t\let\psk@tickstyle\@ne\else
      \ifx#1b\let\psk@tickstyle\m@ne\else
        \@pstrickserr{Bad tick style: `#1#2'}\@ehpa
  \fi\fi\fi}
%
\define@key[psset]{pst-plot}{subticks}{\edef\psk@xsubticks{#1}\edef\psk@ysubticks{#1}}
\define@key[psset]{pst-plot}{xsubticks}{\edef\psk@xsubticks{#1}}
\define@key[psset]{pst-plot}{ysubticks}{\edef\psk@ysubticks{#1}}
%
\define@key[psset]{pst-plot}{subticksize}{\edef\psk@xsubticksize{#1}\edef\psk@ysubticksize{#1}}
\define@key[psset]{pst-plot}{xsubticksize}{\edef\psk@xsubticksize{#1}}
\define@key[psset]{pst-plot}{ysubticksize}{\edef\psk@ysubticksize{#1}}
%
\define@key[psset]{pst-plot}{tickwidth}{%
  \pst@getlength{#1}\psk@xtickwidth
  \pst@getlength{#1}\psk@ytickwidth}
\define@key[psset]{pst-plot}{xtickwidth}{\pst@getlength{#1}\psk@xtickwidth}
\define@key[psset]{pst-plot}{ytickwidth}{\pst@getlength{#1}\psk@ytickwidth}
\define@key[psset]{pst-plot}{subtickwidth}{%
  \pst@getlength{#1}\psk@xsubtickwidth
  \pst@getlength{#1}\psk@ysubtickwidth}
\define@key[psset]{pst-plot}{xsubtickwidth}{\pst@getlength{#1}\psk@xsubtickwidth}
\define@key[psset]{pst-plot}{ysubtickwidth}{\pst@getlength{#1}\psk@ysubtickwidth}
%
\define@key[psset]{pst-plot}{tickcolor}{\edef\psk@xtickcolor{#1}\edef\psk@ytickcolor{#1}}
\define@key[psset]{pst-plot}{xtickcolor}{\edef\psk@xtickcolor{#1}}
\define@key[psset]{pst-plot}{ytickcolor}{\edef\psk@ytickcolor{#1}}
\define@key[psset]{pst-plot}{subtickcolor}{\edef\psk@xsubtickcolor{#1}\edef\psk@ysubtickcolor{#1}}
\define@key[psset]{pst-plot}{xsubtickcolor}{\edef\psk@xsubtickcolor{#1}}
\define@key[psset]{pst-plot}{ysubtickcolor}{\edef\psk@ysubtickcolor{#1}}
%
\define@key[psset]{pst-plot}{xticklinestyle}{%
  \@ifundefined{psls@#1}%
    {\@pstrickserr{Line style `#1' not defined}\@eha}%
    {\edef\psxticklinestyle{#1}}%
}
\define@key[psset]{pst-plot}{xsubticklinestyle}{%
  \@ifundefined{psls@#1}%
    {\@pstrickserr{Line style `#1' not defined}\@eha}%
    {\edef\psxsubticklinestyle{#1}}%
}
\define@key[psset]{pst-plot}{yticklinestyle}{%
  \@ifundefined{psls@#1}%
    {\@pstrickserr{Line style `#1' not defined}\@eha}%
    {\edef\psyticklinestyle{#1}}%
}
\define@key[psset]{pst-plot}{ysubticklinestyle}{%
  \@ifundefined{psls@#1}%
    {\@pstrickserr{Line style `#1' not defined}\@eha}%
    {\edef\psysubticklinestyle{#1}}%
}
\define@key[psset]{pst-plot}{ticklinestyle}{\psset{xticklinestyle=#1,yticklinestyle=#1}}
\define@key[psset]{pst-plot}{subticklinestyle}{\psset{xsubticklinestyle=#1,ysubticklinestyle=#1}}
%
\newif\ifis@yAxis%
%
\def\psxs@@axes#1#2#3#4#5{% llx,lly,urx,ury,exch,arrowA,arrowB
  \begin@SpecialObj%
    \def\pst@tempa{#5}%
    \ifx\pst@tempa\@empty\relax
      \is@yAxisfalse
      \ifnum\psk@dx=\z@
        \pst@dimg=\psk@Dx\psxunit
        \edef\psk@dx{\number\pst@dimg}%
      \fi
    \else
      \is@yAxistrue
      \ifnum\psk@dy=\z@
        \pst@dimg=\psk@Dy\psyunit       
        \edef\psk@dy{\number\pst@dimg}%
      \fi
    \fi% 
    \let\pst@linetype\pst@arrowtype%
    \pst@addarrowdef%
    \addto@pscode{%
      \ifis@yAxis 0 \pst@number#4 \else \pst@number#3 0 \fi
      \ifis@yAxis 0 \pst@number#2 \else \pst@number#1 0 \fi
      ArrowA
      CP 4 2 roll
      ArrowB 
      /yEnd exch def /xEnd exch def
      xEnd yEnd L                                  % the line with arrows 
      /yStart exch def
      /xStart exch def
      stroke                                       % draw the main line
%\psk@ticks: all=0; x=1; y=2; none=3
      \number\psk@ticks\space dup 2 mod 0 eq \ifis@yAxis true \else false \fi and 
      exch 2 lt \ifis@yAxis false \else true \fi and or {
      /viceversa 
        \ifis@yAxis\pst@number#2 \pst@number#4 \else\pst@number#1 \pst@number#3 \fi
         gt { true }{ false } ifelse def           % other way round
      /epsilon 0.01 def                            % rounding errors
      /minTickline \ifis@yAxis \pst@number#1 \else \pst@number#2 \fi def
      /maxTickline \ifis@yAxis \pst@number#3 \else \pst@number#4 \fi def
      /dT \ifis@yAxis\psk@dy\else\psk@dx\fi\space 65536 div viceversa {neg} if def 
      /subTNo \ifis@yAxis\psk@ysubticks\else\psk@xsubticks\fi \space def
      subTNo 0 gt { /dsubT dT subTNo div def}{ /dsubT 0 def } ifelse  % deltaSubTick
      \ifis@yAxis \psk@yticksize \else \psk@xticksize \fi
      /tickend exch def /tickstart exch def
      /Twidth \ifis@yAxis \psk@ytickwidth \else \psk@xtickwidth \fi\space def
      /subTwidth \ifis@yAxis \psk@ysubtickwidth \else \psk@xsubtickwidth \fi\space def
      /STsize \ifis@yAxis \psk@ysubticksize \else \psk@xsubticksize \fi\space def
      /TColor {
        \ifis@yAxis\pst@usecolor\psk@ytickcolor
        \else\pst@usecolor\psk@xtickcolor\fi\space } def
      /subTColor {
        \ifis@yAxis\pst@usecolor\psk@ysubtickcolor
        \else\pst@usecolor\psk@xsubtickcolor\fi\space } def
      /MinValue { \ifis@yAxis yStart \else xStart \fi
        \ifx\psk@arrowA\@empty\else 
          \psk@arrowsize\space CLW mul add \psk@arrowlength\space mul 
           viceversa { sub epsilon add } { add epsilon sub } ifelse \fi } def
      /MaxValue { \ifis@yAxis yEnd \else xEnd \fi
        \ifx\psk@arrowB\@empty\else
          \psk@arrowsize\space CLW mul add \psk@arrowlength\space mul 
           viceversa { add epsilon sub } { sub epsilon add } ifelse \fi } def
      /logLines {
        \ifnum\psk@logLines=\z@ true \else         % all axes
          \ifnum\psk@logLines<\tw@                 % x axis
            \ifis@yAxis false \else true \fi       % do we have x or y axis
          \else
            \ifnum\psk@logLines<\thr@@             % y axis
              \ifis@yAxis true \else false \fi     % do we have x or y axis
            \else 
              false                                % no one
            \fi
          \fi
        \fi
      } def
      /LSstroke {                                  % set linestyle and stroke
        \ifis@yAxis\@nameuse{psls@\psyticklinestyle}
        \else\@nameuse{psls@\psxticklinestyle}\fi } def
      /subLSstroke {                               % set sublinestyle and stroke
        \ifis@yAxis\@nameuse{psls@\psysubticklinestyle}
        \else\@nameuse{psls@\psxsubticklinestyle}\fi } def
%\iffalse
% start ticks ---------------------------------------------------------
      0 dT MaxValue {                              % the positive part of the axes
        /cntTick exch def                          % the index
        logLines {                                 % log lines?
          gsave
          1 10 subTNo div 9.99 {                   % do not write a line for 1
            /dx exch def                           % save index
            /x cntTick dT dx log mul add def       %
            x abs MaxValue abs le {                % out of range?
              x minTickline #5\space moveto
              x maxTickline #5\space lineto
            } if
          } for
          subTwidth SLW subTColor                  % set line width and subtick color
          subLSstroke
          grestore                                 % restore main tick status
          stroke
          /dsubT 0 def                             % no other subticks
        } if
        dsubT abs 0 gt {                           % du we have subticks?
          gsave                                    % save graphic state
% start subticks ----------------------------------------------------------
            /cntsubTick cntTick dsubT add def
            subTNo 1 sub {
            cntsubTick abs MaxValue abs le {       % out of range?
              cntsubTick tickstart STsize mul #5\space moveto
              cntsubTick tickend STsize mul #5\space lineto
            }{ exit }  ifelse
            /cntsubTick cntsubTick dsubT add def
          } repeat 
          subTwidth SLW subTColor                  % set line width and subtick color
          subLSstroke
% end subticks ----------------------------------------------------------
          grestore                                 % restote tick status
        } if
        gsave
        cntTick tickstart #5\space moveto          % line begin main Tick
        cntTick tickend #5\space lineto            % lineto tick end
        Twidth SLW TColor                            % set line width and tick color
        LSstroke
        grestore
      } for
% end ticks ----------------------------------------------------------
%\fi
%\iffalse
% ================================================ % the other side 
      /dT dT neg def                               % the other side of the axis
      /dsubT dsubT neg def
% start ticks ----------------------------------------------------------
      0 dT MinValue epsilon viceversa { add }{ sub } ifelse {
        /cntTick exch def
        logLines {                                 % log lines?
          gsave
          1 10 subTNo div 9.99 {                   % do not write a line for 1
            /dx exch def                           % save index
            /x cntTick dT dx log mul add def       %
            x abs MinValue abs le {                % out of range?
              x minTickline #5\space moveto
              x maxTickline #5\space lineto
            } if
          } for
          /dsubT 0 def 
          subTwidth SLW subTColor                  % set line width and subtick color
          subLSstroke
          grestore
        }
        dsubT abs 0 gt {                           % do we have subticks?
          gsave                                    % save main state
% start subticks ----------------------------------------------------------
          /cntsubTick cntTick dsubT add def
          subTNo 1 sub {
            cntsubTick abs MinValue abs le {       % out of range?
              cntsubTick tickstart STsize mul #5\space moveto
              cntsubTick tickend STsize mul #5\space lineto
            }{ exit } ifelse
            /cntsubTick cntsubTick dsubT add def
          } repeat % for
% end subticks ----------------------------------------------------------
          subTwidth SLW subTColor                  % set line width and subtick color
          subLSstroke
          grestore                                 % restore main state
        } if
        gsave
        cntTick tickstart #5\space moveto         % line begin main Tick
        cntTick tickend #5\space lineto           % lineto tick end
        Twidth SLW TColor                         % set line width and tick color
        LSstroke
        grestore
      } for
% end ticks ----------------------------------------------------------
%\fi    
    } if 
   }%  
  \end@SpecialObj%
}
%
\def\psxs@frame{%
  \begin@SpecialObj%
    \addto@pscode{%
      0 0 moveto \pst@number\pst@dimc 0 L
      0 \pst@number\pst@dimd 2 copy rlineto L closepath}%
    \pst@stroke%
    \psk@fillstyle%
  \end@SpecialObj%
  \let\psk@arrowA\@empty%
  \let\psk@arrowB\@empty%
  \pst@xticksizeC=\z@\pst@yticksizeC=\z@%
  \psxs@@axes\pst@dima\pst@dimb\pst@dimc\pst@dimd{}%
  \psxs@@axes\pst@dima\pst@dimb\pst@dimc\pst@dimd{exch}%
}
%
\define@key[psset]{pst-plot}{axesstyle}{%
  \@ifundefined{psxs@#1}%
    {\@pstrickserr{Axes style `#1' not defined}\@eha}%
    {\edef\psk@axesstyle{#1}}}

\def\psxs@none{\let\psk@arrowA\@empty\let\psk@arrowB\@empty}

% The origin is never the only label.
% #1:Max/Min in pt from origin; #2:arrow; #3:min; #4:max
\def\pst@hlabels#1#2#3#4{%
  \ifdim#1=\z@\else%                   % start from 0 ?
    \ifx#2\empty\else\advance#1\ifdim#1>\z@-\fi7\pslinewidth\fi%
    \pst@cnta=#1\relax%                % Distance (in sp) to end.
    \divide\pst@cnta\psk@dx\relax%     % Number of ticks/labels
    \ifnum\pst@cnta=\z@\else%
      \pst@dimb=\psk@dx sp%            % Space between ticks.
%        \ifPst@plot@yAxis\else\showorigintrue\fi%
        \ifnum\psk@labels<\tw@ \ifPst@plot@xAxis\pst@@hlabels\fi\fi%
        \showoriginfalse%
    \fi%
  \fi%
}
% Knows \pst@dimb and \pst@cnta
\def\pst@@hlabels{%
  \pst@dimm=\pst@xticksizeC\advance\pst@dimm by \pslabelsep%
  \vbox to\z@{%
    \ifdim\pst@dimd>\z@\vskip\pst@dimm\else\vss\fi%
    \ifnum\pst@cnta<\z@\pst@dimb=-\pst@dimb\fi%
    \hbox to\z@{%
      \ifshoworigin\hbox to \z@{\hss\pst@@@hlabel{\psk@Ox}\hss}\fi%
      \mmultido{\n=\psk@Ox+\psk@Dx}{\pst@cnta}{%
        \hskip\pst@dimb\hbox to \z@{\hss\pst@@@hlabel{\n}\hss}%
      }\hss%
    }%
    \ifdim\pst@dimd>\z@\vss\else\vskip\pst@dimm\fi%
}}%
%
\def\pshlabel#1{$#1$}
%
% #1:Max/Min in pt from origin; #2:arrow; #3:min; #4:max
% \psk@labels 0:all; 1:x; 2:y; 3:none
\def\pst@vlabels#1#2#3#4{%
  \ifdim#1=\z@\else%
    \ifx#2\empty\else\advance#1\ifdim#1>\z@-\fi7\pslinewidth\fi%
    \pst@cnta=#1\relax          %      % Distance (in sp) to end.
    \divide\pst@cnta\psk@dy\relax  %   % Number of ticks/labels
    \ifnum\pst@cnta=\z@\else%
      \pst@dima=\psk@dy sp%            % Space between ticks.
%      \ifPst@plot@xAxis\else\showorigintrue\fi%
      \ifodd\number\psk@labels\else\ifPst@plot@yAxis\pst@@vlabels\fi\fi%
      \showoriginfalse%
    \fi%
  \fi%
}

% \pst@dima: the width between two labels in pt
% \pst@dimc: the coordinate of the origin in pt
% \pst@cnta: the lowest label
\def\pst@@vlabels{%
  \pst@dimm=\pst@yticksizeC%
  \advance\pst@dimm by \pslabelsep%
  \vbox to\z@{%
    \ifnum\pst@cnta>\z@\pst@dima=-\pst@dima\fi%
    \offinterlineskip%
    \ifshoworigin
      \vbox to \z@{\vss\hbox to\z@{%
        \ifdim\pst@dimc>\z@\hss\else\hskip\pst@dimm\fi
        \pst@@@vlabel{\psk@Oy}%
        \ifdim\pst@dimc>\z@\hskip\pst@dimm\else\hss\fi}\vss}%
    \fi%
    \mmultido{\n=\psk@Oy+\psk@Dy}{\pst@cnta}{%
      \vbox to\pst@dima{\vss}
      \vbox to \z@{%
        \vss\hbox to\z@{%
          \ifdim\pst@dimc>\z@%
            \hss\pst@@@vlabel{\n}\hskip\pst@dimm%
          \else%
            \hskip\pst@dimm\pst@@@vlabel{\n}\hss%
          \fi
        }\vss%
    }}%
    \vss%
}}
%
\def\psvlabel#1{$#1$}
%
\define@key[psset]{pst-plot}{barwidth}{\pst@getlength{#1}\psk@barwidth}
%
\def\psbar@ii{\addto@pscode{false \tx@NArray \psbar@iii}}% hv
\def\psbar@iii{%
  newpath
  n { 
    /Yval exch def /Xval exch def 
    Xval \number\psk@barwidth 0.5 mul sub 0 moveto 
    0 Yval rlineto \number\psk@barwidth 0 rlineto 
    0 Yval neg rlineto \number\psk@barwidth neg 0 rlineto
  } repeat
}%
\def\beginplot@bar{\begin@SpecialObj}
\def\endplot@bar{%
  \psbar@ii\psk@fillstyle\ifpsshadow\pst@closedshadow\fi%
  \pst@stroke
  \end@SpecialObj%
}
%
\define@key[psset]{pst-plot}{xAxisLabel}{\edef\psk@xAxisLabel{#1}}
\define@key[psset]{pst-plot}{yAxisLabel}{\edef\psk@yAxisLabel{#1}}
\define@key[psset]{pst-plot}{xAxisLabelPos}{\edef\psk@xAxisLabelPos{#1}}
\define@key[psset]{pst-plot}{yAxisLabelPos}{\edef\psk@yAxisLabelPos{#1}}
%
\newdimen\psk@llx
\newdimen\psk@lly
\newdimen\psk@urx
\newdimen\psk@ury
\define@key[psset]{pst-plot}{llx}{\pssetxlength\psk@llx{#1}}
\define@key[psset]{pst-plot}{lly}{\pssetylength\psk@lly{#1}}
\define@key[psset]{pst-plot}{urx}{\pssetxlength\psk@urx{#1}}
\define@key[psset]{pst-plot}{ury}{\pssetylength\psk@ury{#1}}
%
\newif\ifPst@plot@box
\define@key[psset]{pst-plot}{box}[true]{\@nameuse{Pst@plot@box#1}}
%
\newdimen\pst@xunit
\newdimen\pst@yunit
%
%\def\psgraph*{\psgraph{*}}
\def\psgraph{\begingroup\pst@ifstar\psgraph@i}
\def\psgraph@i{%
  \@ifnextchar[{\psgraph@ii}{\psgraph@ii[]}}
\def\psgraph@ii[#1](#2,#3)(#4,#5)#6#7{%
%\def\psgraph{\@ifnextchar[{\psgraph@i}{\psgraph@i[]}}
%\def\psgraph@i[#1](#2,#3)(#4,#5)#6#7{%
% options | minX | minY | maxX | maxY | Length x-axis | length y-axis%  
  \pst@dimo=#4\p@\advance\pst@dimo by -#2\p@ % delta x
  \pst@divide{#6}{\pst@dimo}\pst@tempa
  \pst@xunit=\pst@tempa\p@%
%  
  \pst@dimo=#5\p@\advance\pst@dimo by -#3\p@ % delta y
  \pst@divide{#7}{\pst@dimo}\pst@tempa%
  \pst@yunit=\pst@tempa\p@%
  %
  \pst@dima=#2\pst@xunit \advance\pst@dima by \psk@llx%
  \pst@dimb=#3\pst@yunit \advance\pst@dimb by \psk@lly%
  \pst@dimc=#4\pst@xunit \advance\pst@dimc by \psk@urx%
  \pst@dimd=#5\pst@yunit \advance\pst@dimd by \psk@ury%
  \if@star\pspicture*(\pst@dima,\pst@dimb)(\pst@dimc,\pst@dimd)\else
    \pspicture(\pst@dima,\pst@dimb)(\pst@dimc,\pst@dimd)\fi
  \psset{xunit=\pst@xunit,yunit=\pst@yunit}%
    \psaxes[#1](0,0)(#2,#3)(#4,#5)%
    \ifx\psk@xAxisLabel\@empty\else%
      \ifx\psk@xAxisLabelPos\@empty%
        \uput[0](#4,0){\psk@xAxisLabel}%
        \uput[90](0,#5){\psk@yAxisLabel}%
      \else%
        \rput(\psk@xAxisLabelPos){\psk@xAxisLabel}%
        \rput{90}(\psk@yAxisLabelPos){\psk@yAxisLabel}%
      \fi%
    \fi %
}
\def\endpsgraph{%
  \pst@killglue
  \endpspicture%
  \endgroup%
}
\@namedef{psgraph*}{\psgraph*}
\@namedef{endpsgraph*}{\endpsgraph}
%
\def\resetPSTPlotOptions{%
  \@zero=0%
  \pstScalePoints(1,1){}{}% reset
  \psset[pst-plot]{%
       infix=none,% TeX | PS
       xyAxes=true,
       xyDecimals={},%
       xyLabel={},
       xylogBase={},
       trigLabels=false,
       logLines=none,
       xlabelFactor=\relax,ylabelFactor=\relax,
       nStep=1,nStart=0,nEnd={},%
       xStep=0,yStep=0,xStart={},xEnd={},yStart={},yEnd={},comma=false,%
       plotNo=1,plotNoMax=1,
       xAxisLabel=x,yAxisLabel=y,
       yAxisLabelPos=\@empty,xAxisLabelPos=\@empty,
       plotstyle=line,
       plotpoints=50,
       polarplot=false,
       method=default, whichabs=default, whichord=default,%
       plotfuncx=default, plotfuncy=default, buildvector=false, 
%       algebraic=false, 
       VarStep=default,
%       dimen=middle,% remark of ML
       ticks=all,
       labels=all,
       Ox=0,Dx=1,dx=0,Oy=0,Dy=1,dy=0,
       showorigin=true,
       xticksize=-4pt 4pt,yticksize=-4pt 4pt,
       tickstyle=full,
       subticksize=0.75,subticks=1,tickcolor=black,ticklinestyle=solid,%
       subticklinestyle=solid,%
       subtickcolor=gray,%
       tickwidth=0.5\pslinewidth,%
       subtickwidth=0.25\pslinewidth,
       axesstyle=axes,
       barwidth=0.25cm,
       xAxisLabel=x,yAxisLabel=y,
       yAxisLabelPos=\@empty,xAxisLabelPos=\@empty,
       llx=\z@, lly=\z@, urx=\z@, ury=\z@}% prevents rounding errors 
}
\resetPSTPlotOptions%
%
\catcode`\@=\PstAtCode\relax
\endinput
%%
%% END pst-plot.tex
