%% BEGIN pst-node.tex
%%
%% Nodes with PSTricks
%% See the betadoc documentation for usage. 
%% This uses the header file `pst-node.pro'.
%%
%%
%% COPYRIGHT 1993, 1994, 1999 by Timothy Van Zandt, tvz@nwu.edu.
%% COPYRIGHT 2004,2005 by Herbert Voss <voss _at_ pstricks.de>.
%%
%% This program can be redistributed and/or modified under the terms
%% of the LaTeX Project Public License Distributed from CTAN
%% archives in directory macros/latex/base/lppl.txt.
%%
\csname PSTnodesLoaded\endcsname
\let\PSTnodesLoaded\endinput
\ifx\PSTricksLoaded\endinput\else\input pstricks2 \fi
\ifx\PSTXKeyLoaded\endinput\else\input pst-xkey \fi
%
\def\fileversion{2.001}
\def\filedate{2004/12/23}
\message{ v\fileversion, \filedate}
\message{`pst-node2' v\fileversion, \filedate\space (tvz,hv)}
%
\edef\PstAtCode{\the\catcode`\@} \catcode`\@=11\relax
\pst@addfams{pst-node}
\SpecialCoor
%
\pstheader{pst-node2.pro}
%
\def\pst@nodedict{tx@NodeDict begin }
\def\pst@zapspace#1 #2{%
  #1%
  \ifx#2\@empty\else\expandafter\pst@zapspace\fi%
  #2}
\def\pst@getnode#1#2{%
  \pst@expandafter\pst@@getnode{#1},,\@nil#2%
}
\def\pst@@getnode#1,#2,#3\@nil#4{%
  \ifx\@empty#3\@empty
    \edef#4{/N@\pst@zapspace#1 \@empty\space}%
  \else
    \pst@cntg=#1\relax
    \pst@cnth=#2\relax
    \edef#4{/N@M-\ifnum\psmatrixcnt=\z@ 1\else\the\psmatrixcnt\fi
      -\the\pst@cntg-\the\pst@cnth\space}%
  \fi%
}
\def\tx@NewNode{NewNode }
\def\pst@newnode#1#2#3#4{%
% DG/SR modification begin - Nov.  9, 2000 - Patch 11
  \pst@killglue%
% DG/SR modification end
  \leavevmode%
  \pst@getnode{#1}\pst@thenode%
  \pst@Verb{%
    \pst@nodedict
    {#3}%
    \ifx\psk@name\relax false \else \psk@name true \fi
    \pst@thenode
    #2
    {#4}
    \tx@NewNode
    end%
  }%
  \global\let\psk@name\relax%
  \pstree@nodehook%
  \global\let\pstree@nodehook\relax%
}
\let\pstree@nodehook\relax
\newif\ifnodealign
\define@key[psset]{pst-node}{nodealign}{\@nameuse{nodealign#1}}
\psset[pst-node]{nodealign=false}
\def\pst@nodealign{%
  \pst@dimg=\ht\pst@hbox%
  \advance\pst@dimg-\dp\pst@hbox%
  \divide\pst@dimg2%
  \lower\pst@dimg}
\def\tx@InitPnode{InitPnode }
\def\pnode{\@ifnextchar({\pnode@}{\pnode@(0,0)}}
\def\pnode@(#1)#2{%
  \pst@@getcoor{#1}%
  \pst@newnode{#2}{10}{\pst@coor}{\tx@InitPnode}%
  \ignorespaces%
}
\def\tx@InitCnode{InitCnode }
\def\cnode{\pst@object{cnode}}
\def\cnode@i{\@ifnextchar({\cnode@ii}{\cnode@ii(0,0)}}
\def\cnode@ii(#1)#2#3{%
\leavevmode
\hbox{%
\use@par
\pst@@getcoor{#1}%
\pssetlength\pst@dimc{#2}%
\pst@dimg=\psk@dimen\pslinewidth
\advance\pst@dimc-\pst@dimg
\advance\pst@dimc.5\pslinewidth
\ifnodealign
\kern\pst@dimc
\vrule width\z@ height \pst@dimc depth \pst@dimc
\fi
\pscircle@do(#1){#2}%
\pst@newnode{#3}{11}{\pst@coor \pst@number\pst@dimc}{\tx@InitCnode}%
% DG/SR modification begin - Jul. 30, 1997 - Patch 2
%\ifnodealign \kern\pst@dimc\egroup \fi}%
\ifnodealign\kern\pst@dimc\fi}%
% DG/SR modification end
\ignorespaces}
\def\Cnode{\pst@object{Cnode}}
\def\Cnode@i{\@ifnextchar({\Cnode@ii}{\Cnode@ii(0,0)}}
\def\Cnode@ii(#1)#2{\cnode@ii(#1){\psk@radius}{#2}}%
\def\cnodeput{\pst@object{cnodeput}}
\def\cnodeput@i{\@ifnextchar({\cnodeput@iii}{\cnodeput@ii}}
\def\cnodeput@ii#1{%
\addto@par{rot={#1}}%
\@ifnextchar({\cnodeput@iii}{\cnodeput@iii(\z@,\z@)}}
\def\cnodeput@iii(#1)#2{%
\pst@killglue
\@fixedradiusfalse
\def\pst@nodehook{\cnodeput@iv{#2}}%
\pst@makebox{\cput@v{#1}}}
\def\cnodeput@iv#1{%
\pst@newnode{#1}{11}{\pscirclebox@iv \pst@number\pslinewidth add}%
{\tx@InitCnode}%
\global\let\pst@nodehook\relax}
\def\Cnodeput{\pst@object{Cnodeput}}
\def\Cnodeput@i{\@ifnextchar({\Cnodeput@iii}{\Cnodeput@ii}}
\def\Cnodeput@ii#1{%
\addto@par{rot={#1}}%
\@ifnextchar({\Cnodeput@iii}{\Cnodeput@iii(\z@,\z@)}}
\def\Cnodeput@iii(#1)#2{%
\pst@killglue
\@fixedradiustrue
\def\pst@nodehook{\Cnodeput@iv{#2}}%
\pst@makebox{\cput@iv{#1}}}
\def\Cnodeput@iv#1{%
\pst@newnode{#1}{11}{%
\pst@number{\wd\pst@hbox} 2 div \pst@number\pst@dima % x y
\pst@number\pst@dimb \pst@number\pslinewidth \psk@dimen .5 sub mul sub }% r
{\tx@InitCnode}%
\global\let\pst@nodehook\relax}
\def\circlenode{\pst@object{circlenode}}
\def\circlenode@i#1{\pst@makebox{\circlenode@ii{#1}}}
\def\circlenode@ii#1{%
\begingroup
\pst@useboxpar
\setbox\pst@hbox=\hbox{%
\cnodeput@iv{#1}%
\pscirclebox@iii
\box\pst@hbox}%
\ifnodealign \psboxseptrue \fi
\ifpsboxsep \pscirclebox@sep \fi
\leavevmode
\ifnodealign\pst@nodealign\fi
\box\pst@hbox
\endgroup}
\def\Circlenode{\pst@object{Circlenode}}
\def\Circlenode@i#1{\pst@makebox{\Circlenode@ii{#1}}}
\def\Circlenode@ii#1{%
\begingroup
\pst@useboxpar
\pst@dima=\ht\pst@hbox
\advance\pst@dima\dp\pst@hbox
\divide\pst@dima\tw@
\pssetlength\pst@dimb\psk@radius
\setbox\pst@hbox=\hbox{%
\Cnodeput@iv{#1}%
\pscircle(.5\wd\pst@hbox,\pst@dima){\pst@dimb}%
\box\pst@hbox}%
\ifnodealign \psboxseptrue \fi
\ifpsboxsep \psCirclebox@sep \fi
\leavevmode
\ifnodealign\pst@nodealign\fi
\box\pst@hbox
\endgroup}
\def\tx@GetRnodePos{GetRnodePos }
\def\tx@InitRnode{InitRnode }
\def\rnode{\@ifnextchar[{\rnode@i}{\def\pst@par{}\rnode@ii}}
\def\rnode@i[#1]{\def\pst@par{ref=#1}\rnode@ii}
\def\rnode@ii#1{\pst@makebox{\rnode@iii\rnode@iv{#1}}}
\def\rnode@iii#1#2{%
% DG modification begin - Jan. 1997
  \leavevmode
% DG modification end
  \begingroup
% DG/SR modification begin - Apr. 28, 1998 - Patch 6
    \pst@useboxpar
% DG/SR modification end
    #1%
    \if@star\pst@starbox\fi
    \ifnodealign\lower\pst@dimb\fi
    \hbox{%
      \pst@newnode{#2}{16}{%
        \pst@number{\ht\pst@hbox}%
        \pst@number{\dp\pst@hbox}%
        \pst@number{\wd\pst@hbox}%
        \pst@number\pst@dima
        \pst@number\pst@dimb%
      }{\tx@InitRnode}%
      \box\pst@hbox%
    }%
  \endgroup}
\def\rnode@iv{%
  \pst@dima=\psk@xref\wd\pst@hbox
  \ifx\psk@yref\relax
    \pst@dimb=\z@
  \else
    \pst@dimb=\ht\pst@hbox
    \advance\pst@dimb\dp\pst@hbox
    \pst@dimb=\psk@yref\pst@dimb
    \advance\pst@dimb-\dp\pst@hbox
  \fi%
}
\define@key[psset]{pst-node}{href}{\pst@checknum{#1}\psk@href}
\define@key[psset]{pst-node}{vref}{\def\psk@vref{#1}}
\psset[pst-node]{href=0,vref=.7ex}
%
\def\Rnode{\pst@object{Rnode}}
\def\Rnode@i#1{\pst@makebox{\rnode@iii\Rnode@ii{#1}}}
\def\Rnode@ii{%
% DG modification begin - Jan. 1997
%   - \begingroup removed  as it seems to doesn't work any more
%   - \Rnode doesn't process the optional parameter changes
%\begingroup
    \use@par
% DG modification end
    \pst@dima=\psk@href\wd\pst@hbox
    \advance\pst@dima\wd\pst@hbox
    \divide\pst@dima 2
    \pssetlength\pst@dimb{\psk@vref}%
}
\def\tx@DiaNodePos{DiaNodePos }
  \def\dianode{\pst@object{dianode}}%
  \def\dianode@i#1{\pst@makebox{\dianode@ii{#1}}}%
  \def\dianode@ii#1{%
    \begingroup
    \pst@useboxpar
    \psdiabox@iii
    \setbox\pst@hbox=\hbox{%
	\pst@newnode{#1}{14}{}{%
	    /X \pst@number\pst@dima def
	    /Y \pst@number\pst@dimb def
	    /w \pst@number\pst@dimc 2 mul def
	    /h \pst@number\pst@dimd 2 mul def
	    /NodePos { \tx@DiaNodePos } def%
	}%
	\box\pst@hbox%
    }%
    \ifnodealign\psboxseptrue\fi
    \ifpsboxsep\psdiabox@sep\fi
% DG/SR modification begin - Sep. 2, 1997 - Patch 3
    \leavevmode
% DG/SR modification end
    \ifnodealign\lower\pst@dimb\fi
    \box\pst@hbox
    \endgroup%
}
\def\tx@TriNodePos{TriNodePos }
\def\tx@InitTriNode{InitTriNode }
\def\trinode{\pst@object{trinode}}
\def\trinode@i#1{\pst@makebox{\trinode@ii{#1}}}
\def\trinode@ii#1{%
\begingroup
\pst@useboxpar
\pstribox@iii
\setbox\pst@hbox=\hbox{%
\pst@newnode{#1}{14}{}{%
\pst@number\pst@dimc
\pst@number\pst@dimd
\ifodd\psk@trimode
exch
\pst@number\pst@dima
\else
\pst@number\pst@dimb
\fi
\psk@trimode
\pst@number{\wd\pst@hbox}%
\pst@number{\ht\pst@hbox}%
\pst@number{\dp\pst@hbox}%
\tx@InitTriNode}%
\box\pst@hbox}%
\ifnodealign\psboxseptrue\fi
\ifpsboxsep\pstribox@sep\fi
% DG/SR modification begin - Sep. 2, 1997 - Patch 3
\leavevmode
% DG/SR modification end
\ifnodealign\lower\pst@tempa\fi
\box\pst@hbox
\endgroup}
\def\tx@OvalNodePos{OvalNodePos }
\def\ovalnode{\pst@object{ovalnode}}
\def\ovalnode@i#1{\pst@makebox{\ovalnode@ii{#1}}}
\def\ovalnode@ii#1{%
    \begingroup
	\pst@useboxpar
	\psovalbox@iii
	\setbox\pst@hbox=\hbox{%
	    \pst@newnode{#1}{14}{}{%
		/X \pst@number\pst@dima def
		/Y \pst@number\pst@dimb def
		/w \pst@number\pst@dimc def
		/h \pst@number\pst@dimd def
		/NodePos { \tx@OvalNodePos } def
	    }%
	    \unhbox\pst@hbox%
	}%
	\ifnodealign\psboxseptrue\fi
	\ifpsboxsep\psovalbox@sep\fi
% DG/SR modification begin - Sep. 2, 1997 - Patch 3
	\leavevmode
% DG/SR modification end
	\ifnodealign\lower\pst@dimb\fi
	\box\pst@hbox
    \endgroup%
}
\def\dotnode{\pst@object{dotnode}}
\def\dotnode@i{\@ifnextchar({\dotnode@ii}{\dotnode@ii(\z@,\z@)}}
\def\dotnode@ii(#1)#2{%
  \leavevmode%
  \hbox{%
    \use@par%
    \pst@@getcoor{#1}%
    \pst@getdotsize%
    \pstree@nodehook%
    \ifnodealign%
      \pst@dima=\pst@dimg%
      \kern\pst@dima%
      \vrule width\z@ height \pst@dimh depth \pst@dimh%
    \fi%
    \pst@newnode{#2}{14}{}{%
      \pst@coor
      /Y ED /X ED
      /w \pst@number\pst@dimg def
      /h \pst@number\pst@dimh def
      /NodePos { \tx@OvalNodePos } def}%
    \psdot@ii(#1)%
    \ifnodealign\kern\pst@dima\fi}%
  \ignorespaces}
\define@key[psset]{pst-node}{framesize}{\pst@expandafter\pstframesize@ii{#1} \@nil}
\def\pstframesize@ii#1 #2\@nil{%
  \pssetlength\pst@dimg{#1}%
  \divide\pst@dimg2
  \edef\psk@framewidth{\pst@number\pst@dimg}%
  \ifx\@empty#2\@empty
    \let\psk@frameheight\psk@framewidth
  \else
    \pssetlength\pst@dimg{#2}%
    \divide\pst@dimg2
    \edef\psk@frameheight{\pst@number\pst@dimg}%
  \fi%
}
\psset[pst-node]{framesize=10pt}
%
\def\fnode{\pst@object{fnode}}
\def\fnode@i{\@ifnextchar({\fnode@ii}{\fnode@ii(\z@,\z@)}}
\def\fnode@ii(#1)#2{%
  \leavevmode%
  \pst@killglue%
  \hbox{%
    \use@par%
    \begin@ClosedObj%
    \ifnodealign%
      \kern\psk@framewidth\p@%
      \vrule width\z@ height \psk@frameheight\p@ depth \psk@frameheight\p@%
      \edef\pst@coor{0 0 }%
    \else%
      \pst@@getcoor{#1}%
    \fi%
    \pst@newnode{#2}{14}{}{%
      \pst@coor
      /Y ED /X ED
      /d \psk@dimen .5 sub CLW mul neg def
      /r \psk@framewidth d add def
      /l r neg def
      /u \psk@frameheight d add def
      /d u neg def
      /NodePos { \tx@GetRnodePos } def}%
    \addto@pscode{%
      /x2 \psk@framewidth CLW \psk@dimen mul sub def
      /y2 \psk@frameheight CLW \psk@dimen mul sub def
      \pst@coor 2 copy
      y2 sub /y1 ED
      x2 sub /x1 ED
      y2 add /y2 ED
      x2 add /x2 ED
      \psk@cornersize
      1 index 0 eq { pop pop \tx@Rect } { \tx@OvalFrame } ifelse}%
    \def\pst@linetype{2}%
    \showpointsfalse%
    \end@ClosedObj%
    \ifnodealign\kern\psk@framewidth\p@\fi}%
  \ignorespaces}
\define@key[psset]{pst-node}{nodesepA}{%
  \pst@getlength{#1}\psk@nodesepA
  \def\psk@nodeseptypeA{0 }}
\define@key[psset]{pst-node}{nodesepB}{%
  \pst@getlength{#1}\psk@nodesepB%
  \def\psk@nodeseptypeB{0 }}
\define@key[psset]{pst-node}{nodesep}{%
  \pst@getlength{#1}\psk@nodesepA%
  \let\psk@nodesepB\psk@nodesepA%
  \def\psk@nodeseptypeA{0 }%
  \def\psk@nodeseptypeB{0 }}
\psset[pst-node]{nodesep=0pt}
%
\define@key[psset]{pst-node}{XnodesepA}{%
    \pst@getlength{#1}\psk@nodesepA%
    \def\psk@nodeseptypeA{1 }%
}
\define@key[psset]{pst-node}{XnodesepB}{%
    \pst@getlength{#1}\psk@nodesepB%
    \def\psk@nodeseptypeB{1 }%
}
\define@key[psset]{pst-node}{Xnodesep}{%
    \pst@getlength{#1}\psk@nodesepA%
    \let\psk@nodesepB\psk@nodesepA%
    \def\psk@nodeseptypeA{1 }%
    \def\psk@nodeseptypeB{1 }%
}
\define@key[psset]{pst-node}{YnodesepA}{%
    \pst@getlength{#1}\psk@nodesepA%
    \def\psk@nodeseptypeA{2 }%
}
\define@key[psset]{pst-node}{YnodesepB}{%
    \pst@getlength{#1}\psk@nodesepB%
    \def\psk@nodeseptypeB{2 }%
}
\define@key[psset]{pst-node}{Ynodesep}{%
    \pst@getlength{#1}\psk@nodesepA%
    \let\psk@nodesepB\psk@nodesepA%
    \def\psk@nodeseptypeA{2 }%
    \def\psk@nodeseptypeB{2 }%
}
\define@key[psset]{pst-node}{armA}{%
  \pst@getlength{#1}\psk@armA
  \def\psk@armtypeA{0 }}
\define@key[psset]{pst-node}{armB}{%
  \pst@getlength{#1}\psk@armB
  \def\psk@armtypeB{0 }}
\define@key[psset]{pst-node}{arm}{%
  \pst@getlength{#1}\psk@armA%
  \let\psk@armB\psk@armA%
  \def\psk@armtypeA{0 }%
  \def\psk@armtypeB{0 }}
\psset[pst-node]{arm=10pt}
%
\define@key[psset]{pst-node}{XarmA}{%
  \pst@getlength{#1}\psk@armA%
  \def\psk@armtypeA{1 }}
\define@key[psset]{pst-node}{XarmB}{%
  \pst@getlength{#1}\psk@armB%
  \def\psk@armtypeB{1 }}
\define@key[psset]{pst-node}{Xarm}{%
  \pst@getlength{#1}\psk@armA%
  \let\psk@armB\psk@armA%
  \def\psk@armtypeA{1 }%
  \def\psk@armtypeB{1 }}
\define@key[psset]{pst-node}{YarmA}{%
  \pst@getlength{#1}\psk@armA%
  \def\psk@armtypeA{2 }}
\define@key[psset]{pst-node}{YarmB}{%
  \pst@getlength{#1}\psk@armB%
  \def\psk@armtypeB{2 }}
\define@key[psset]{pst-node}{Yarm}{%
  \pst@getlength{#1}\psk@armA%
  \let\psk@armB\psk@armA%
  \def\psk@armtypeA{2 }%
  \def\psk@armtypeB{2 }}
\define@key[psset]{pst-node}{offsetA}{\pst@getlength{#1}\psk@offsetA}
\define@key[psset]{pst-node}{offsetB}{\pst@getlength{#1}\psk@offsetB}
\define@key[psset]{pst-node}{offset}{\pst@getlength{#1}\psk@offsetA\let\psk@offsetB\psk@offsetA}
\define@key[psset]{pst-node}{angleA}{\pst@getangle{#1}\psk@angleA}
\define@key[psset]{pst-node}{angleB}{\pst@getangle{#1}\psk@angleB}%
\define@key[psset]{pst-node}{angle}{%
  \pst@getangle{#1}\psk@angleA
  \let\psk@angleB\psk@angleA%
}
\define@key[psset]{pst-node}{arcangleA}{\pst@getangle{#1}\psk@arcangleA}
\define@key[psset]{pst-node}{arcangleB}{\pst@getangle{#1}\psk@arcangleB}%
\define@key[psset]{pst-node}{arcangle}{%
  \pst@getangle{#1}\psk@arcangleA
  \let\psk@arcangleB\psk@arcangleA}
\define@key[psset]{pst-node}{ncurvA}{\pst@checknum{#1}\psk@ncurvA}
\define@key[psset]{pst-node}{ncurvB}{\pst@checknum{#1}\psk@ncurvB}%
\define@key[psset]{pst-node}{ncurv}{\pst@checknum{#1}\psk@ncurvA\let\psk@ncurvB\psk@ncurvA}
\psset[pst-node]{ncurv=.67,arcangle=8,angle=0,offset=0pt}
%
\define@key[psset]{pst-node}{lineAngle}{%
  \pst@getlength{#1}\psk@armB
  \def\psk@armtypeB{0 }
  \def\psk@lineAngle{#1}}%
\psset[pst-node]{lineAngle=0}%
%
%
\def\tx@GetCenter{GetCenter }
\def\tx@XYPos{XYPos }
\def\tx@GetEdge{GetEdge }
\def\tx@AddOffset{AddOffset }
\def\tx@GetEdgeA{GetEdgeA }
\def\tx@GetEdgeB{GetEdgeB }
\def\tx@GetArmA{GetArmA }
\def\tx@GetArmB{GetArmB }
\def\check@arrow#1#2{%
  \check@@arrow#2-\@nil
  \if@pst
    \addto@par{arrows=#2}%
    \def\next{#1}%
  \else\def\next{#1{#2}}\fi
  \next%
}
\def\check@@arrow#1-#2\@nil{%
  \ifx\@nil#2\@nil\@pstfalse\else\@psttrue\fi}
\def\tx@InitNC{InitNC }
\def\nc@object#1#2#3#4#5{%
  \csname begin@#1Obj\endcsname%
  \showpointsfalse%
  \pst@getnode{#2}\pst@tempa%
  \pst@getnode{#3}\pst@tempb%
  \gdef\npos@default{#4 }%
  \addto@pscode{%
    /IfArrowInside { false } def
    /NCLW CLW def
    \pst@nodedict
    \psk@offsetA
    \psk@offsetB neg
    \psk@nodesepA
    \psk@nodesepB
    \psk@nodeseptypeA
    \psk@nodeseptypeB
    \pst@tempa
    \pst@tempb
    \tx@InitNC { #5 } if
    end}%
  \def\use@pscode{%
    \pst@Verb{gsave \tx@STV newpath \pst@code\space grestore}%
    \gdef\pst@code{}}%
  \csname end@#1Obj\endcsname%
  \pst@shortput}
\def\npos@default{.5 }
\def\pc@object#1{%
  \@ifnextchar({\pc@@object#1}{\pst@getarrows{\pc@@object#1}}}
\def\pc@@object#1(#2)(#3){%
  \pnode(#2){@@A}\pnode(#3){@@B}%
  #1{@@A}{@@B}%
}
\def\tx@LPutLine{LPutLine }
\def\tx@LPutLines{LPutLines }
\def\tx@BezierMidpoint{BezierMidpoint }
\def\tx@HPosBegin{HPosBegin }
\def\tx@HPosEnd{HPosEnd }
\def\tx@HPutLine{HPutLine }
\def\tx@HPutLines{HPutLines }
\def\tx@VPosBegin{VPosBegin }
\def\tx@VPosEnd{VPosEnd }
\def\tx@VPutLine{VPutLine }
\def\tx@VPutLines{VPutLines }
\def\tx@HPutCurve{HPutCurve }
\def\tx@NCCoor{NCCoor }
\def\tx@NCLine{NCLine }
%
\def\ncline{\pst@object{ncline}}
\def\ncline@i{\check@arrow{\ncline@ii}}
\def\ncline@ii#1#2{\nc@object{Open}{#1}{#2}{.5}{ % hv
  /IfArrowInside { (\psk@ArrowInside) length 0 gt { true }{ false } ifelse } bind def
  /ArrowInsidePos \psk@ArrowInsidePos\space def
  /ArrowInsideOffset \psk@ArrowInsideOffset\space def
  /ArrowInsideNo \psk@ArrowInsideNo\space cvi def
  \tx@NCLine /LPutPos { xB yB xA yA \tx@LPutLine } def
}}
\def\ncline@iii#1#2{}
\def\pcline{\pst@object{pcline}}
\def\pcline@i{\pc@object\ncline@ii}
%
\def\tx@NCLines{NCLines }
\def\nclines{\pst@object{nclines}}
\def\nclines@i{\check@arrow\nclines@ii}
\def\nclines@ii#1#2{%
  \begingroup%
  \use@par%
  \def\pst@aftercoors{\nclines@iii{#1}{#2}}%
  \def\pst@coors{}%
  \pst@@getcoors%
}
\def\nclines@iii#1#2{%
  \nc@object{Open}{#1}{#2}{.5}{%
  tx@Dict begin \psline@iii pop end
  mark \pst@coors \tx@NCLines}%
  \endgroup%
  \ignorespaces%
}
%
\def\tx@NCCurve{NCCurve }
\def\nccurve{\pst@object{nccurve}}
\def\nccurve@i{\check@arrow{\nccurve@ii}}
\def\nccurve@ii#1#2{\nc@object{Open}{#1}{#2}{.5}{%
  /IfArrowInside { (\psk@ArrowInside) length 0 gt { true }{ false } ifelse } bind def
  /ArrowInsidePos \psk@ArrowInsidePos\space def
  /ArrowInsideOffset \psk@ArrowInsideOffset\space def
  /ArrowInsideNo \psk@ArrowInsideNo\space cvi def
  /AngleA \psk@angleA\space def /AngleB \psk@angleB\space def
  \psk@ncurvB\space \psk@ncurvA\space
  \tx@NCCurve}}
\def\pccurve{\pst@object{pccurve}}
\def\pccurve@i{\pc@object\nccurve@ii}
%
\def\ncarc{\pst@object{ncarc}}
\def\ncarc@i{\check@arrow{\ncarc@ii}}
\def\ncarc@ii#1#2{\nc@object{Open}{#1}{#2}{.5}{%
  yB yA sub xB xA sub \tx@Atan dup
  \psk@arcangleA\space add /AngleA ED
  \psk@arcangleB\space sub 180 add /AngleB ED
  \psk@ncurvB\space \psk@ncurvA\space
  \tx@NCCurve}}
\def\pcarc{\pst@object{pcarc}}
\def\pcarc@i{\pc@object\ncarc@ii}
\def\tx@NCAngles{NCAngles }
\def\ncangles{\pst@object{ncangles}}
\def\ncangles@i{\check@arrow{\ncangles@ii}}
\def\ncangles@ii#1#2{%
\nc@object{Open}{#1}{#2}{1.5}{\ncangles@iii \tx@NCAngles}}
\def\ncangles@iii{%
  tx@Dict begin \psline@iii pop end
  /AngleA \psk@angleA def
  /AngleB \psk@angleB def
  /ArmA \psk@armA def
  /ArmB \psk@armB def
  /ArmTypeA \psk@armtypeA def
  /ArmTypeB \psk@armtypeB def }
\def\pcangles{\pst@object{pcangles}}
\def\pcangles@i{\pc@object\ncangles@ii}
\def\tx@NCAngle{NCAngle }
\def\ncangle{\pst@object{ncangle}}
\def\ncangle@i{\check@arrow{\ncangle@ii}}
\def\ncangle@ii#1#2{%
\nc@object{Open}{#1}{#2}{1.5}{\ncangles@iii \tx@NCAngle}}
\def\pcangle{\pst@object{pcangle}}
\def\pcangle@i{\pc@object\ncangle@ii}
\def\tx@NCBar{NCBar }
\def\ncbar{\pst@object{ncbar}}
\def\ncbar@i{\check@arrow{\ncbar@ii}}
\def\ncbar@ii#1#2{\nc@object{Open}{#1}{#2}{1.5}{%
  \ncangles@iii /AngleB \psk@angleA def \tx@NCBar}}
\def\pcbar{\pst@object{pcbar}}
\def\pcbar@i{\pc@object\ncbar@ii}
%
\def\tx@NCBarr{NCBarr }
\def\ncbarr{\pst@object{ncbarr}}
\def\ncbarr@i{\check@arrow{\ncbarr@ii}}
\def\ncbarr@ii#1#2{\nc@object{Open}{#1}{#2}{2.5}{%
  \ncangles@iii
  /tpos \psk@tpos\space def 
  /AngleB AngleA 180 add def 
  \tx@NCBarr}}
\def\pcbarr{\pst@object{pcbarr}}
\def\pcbarr@i{\pc@object\ncbarr@ii}
%
\def\tx@NCDiag{NCDiag }
\def\ncdiag{\pst@object{ncdiag}}
\def\ncdiag@i{\check@arrow{\ncdiag@ii}}
\def\ncdiag@ii#1#2{%
  \nc@object{Open}{#1}{#2}{1.5}{\ncangles@iii /lineAngle \psk@lineAngle\space def \tx@NCDiag}}
\def\pcdiag{\pst@object{pcdiag}}
\def\pcdiag@i{\pc@object\ncdiag@ii}
\def\tx@NCDiagg{NCDiagg }
\def\ncdiagg{\pst@object{ncdiagg}}
\def\ncdiagg@i{\check@arrow{\ncdiagg@ii}}
\def\ncdiagg@ii#1#2{%
  \nc@object{Open}{#1}{#2}{.5}{\ncangles@iii /lineAngle \psk@lineAngle\space def \tx@NCDiagg}}
\def\pcdiagg{\pst@object{pcdiagg}}
\def\pcdiagg@i{\pc@object\ncdiagg@ii}
\def\tx@NCLoop{NCLoop }
\define@key[psset]{pst-node}{loopsize}{\pst@getlength{#1}\psk@loopsize}
\psset[pst-node]{loopsize=1cm}
%
\def\ncloop{\pst@object{ncloop}}
\def\ncloop@i{\check@arrow{\ncloop@ii}}
\def\ncloop@ii#1#2{%
\nc@object{Open}{#1}{#2}{2.5}%
{\ncangles@iii /loopsize \psk@loopsize def \tx@NCLoop}}
\def\pcloop{\pst@object{pcloop}}
\def\pcloop@i{\pc@object\ncloop@ii}
\def\tx@NCCircle{NCCircle }
\def\nccircle{\pst@object{nccircle}}
\def\nccircle@i{\check@arrow{\nccircle@ii}}
\def\nccircle@ii#1#2{%
\pssetlength\pst@dima{#2}%
\nc@object{Open}{#1}{#1}{.5}{%
  /AngleA \psk@angleA def
  /r \pst@number\pst@dima def
  \tx@NCCircle \psarc@v end}}
\def\tx@NCBox{NCBox }
\def\ncbox{\pst@object{ncbox}}
\def\ncbox@i{\check@arrow{\ncbox@ii}}
\def\ncbox@ii#1#2{%
  \def\pst@linetype{2}%
  \nc@object{Closed}{#1}{#2}{.5}{%
    tx@Dict begin \psline@iii pop end
    \psk@boxheight \psk@boxdepth
    \tx@NCBox}}
\def\pcbox{\pst@object{pcbox}}
\def\pcbox@i{\pc@object\ncbox@ii}
\def\tx@NCArcBox{NCArcBox }
\define@key[psset]{pst-node}{boxheight}{\pst@getlength{#1}\psk@boxheight}
\define@key[psset]{pst-node}{boxdepth}{\pst@getlength{#1}\psk@boxdepth}
\define@key[psset]{pst-node}{boxsize}{%
  \psset[pst-node]{boxheight=#1}%
  \let\psk@boxdepth\psk@boxheight}
\psset[pst-node]{boxsize=.4cm}
%
\def\ncarcbox{\pst@object{ncarcbox}}
\def\ncarcbox@i{\check@arrow{\ncarcbox@ii}}
\def\ncarcbox@ii#1#2{%
\def\pst@linetype{1}%
\nc@object{Closed}{#1}{#2}{.5}{%
\psk@arcangleA \psk@boxheight \psk@boxdepth \pst@number\pslinearc
\tx@NCArcBox}}
\def\pcarcbox{\pst@object{pcarcbox}}
\def\pcarcbox@i{\pc@object\ncarcbox@ii}
\def\tx@Tfan{Tfan }
% Changed according pst-beta.bug December 3, 1993
% nrot=:<angle> does not work when : is active.
\begingroup
\catcode`\:=13
\gdef\pst@activerot{\def:{\string:}}
\endgroup
\define@key[psset]{pst-node}{nrot}{%
  \begingroup%
  \pst@activerot%
  \pst@expandafter{\@ifnextchar:{\pstnrot@ii}{\pstrot@ii}}{#1}\@nil%
  \global\let\pst@tempg\psk@rot%
  \endgroup%
  \let\psk@nrot\pst@tempg}
\def\pstnrot@ii:#1\@nil{%
  \pstrot@ii#1\@nil%
  \edef\psk@rot{NAngle \ifx\psk@rot\@empty\else\psk@rot add \fi}}
%
\def\tx@LPutCoor{LPutCoor }
\def\tx@LPut{LPut }
\define@key[psset]{pst-node}{npos}{%
  \def\pst@tempa{#1}%
  \ifx\pst@tempa\@empty
    \def\psk@npos{\npos@default}%
  \else%
    \pst@checknum{#1}\psk@npos%
  \fi}
\psset{nrot=0,npos={}}
%
\def\ncput{\pst@object{ncput}}
\def\ncput@i{\pst@killglue\pst@makebox{\ncput@ii}}
\def\ncput@ii{%
    \begingroup
	\use@par
	\if@star\pst@starbox\fi
	\pst@makesmall\pst@hbox
	\pst@rotate\psk@nrot\pst@hbox
	\ncput@iii
    \endgroup
    \pst@shortput%
}
\def\ncput@iii{%
    \leavevmode
    \hbox{%
	\pst@Verb{%
	    \pst@nodedict
	    /t \psk@npos def
	    \tx@LPut
	    end
	    \tx@PutBegin
	}%
	\box\pst@hbox
	\pst@Verb{\tx@PutEnd}%
    }%
}
\def\naput{\pst@object{naput}}
\def\naput@i{\pst@killglue\pst@makebox{\naput@ii{NAngle 90 add}}}
\def\naput@ii#1{%
\begingroup
\use@par
\if@star\pst@starbox\fi
\def\psk@refangle{#1 }%
\let\psk@rot\psk@nrot
\uput@vii
{exch pop add a \tx@PtoC h1 add exch w1 add exch }%
{tx@Dict /NCLW known { NCLW add } if }%
\ncput@iii
\endgroup
\pst@shortput}
\def\nbput{\pst@object{nbput}}
\def\nbput@i{\pst@killglue\pst@makebox{\naput@ii{NAngle 90 sub}}}
\define@key[psset]{pst-node}{tpos}{%
  \pst@checknum{#1}\psk@tpos%
  \ifdim\psk@tpos \p@<\z@%
    \def\psk@tpos{.5}%
% DG/SR modification begin - Sep. 23, 1998 - Patch 7
%\@pstrickserr{Bad `tpos' value: `#1'. Must be 0<tpos<1}\@epha
    \@pstrickserr{Bad `tpos' value: `#1'. Must be 0<tpos<1}\@ehpa%
% DG/SR modification end
  \else%
    \ifdim\psk@tpos \p@>\p@%
      \def\psk@tpos{.5}%
% DG/SR modification begin - Sep. 23, 1998 - Patch 7
%\@pstrickserr{Bad `tpos' value: `#1'. Must be 0<tpos<1}\@epha
      \@pstrickserr{Bad `tpos' value: `#1'. Must be 0<tpos<1}\@ehpa%
% DG/SR modification end
    \fi%
  \fi}
\psset[pst-node]{tpos=0.5}
%
\def\tvput{\pst@object{tvput}}
\def\tvput@i{\pst@makebox{\psput@tput{H}{1}}}
\def\tlput{\pst@object{tlput}}
\def\tlput@i{\pst@makebox{\psput@tput{H}{true}}}
\def\trput{\pst@object{trput}}
\def\trput@i{\pst@makebox{\psput@tput{H}{false}}}
\def\thput{\pst@object{thput}}
\def\thput@i{\pst@makebox{\psput@tput{V}{1}}}
\def\taput{\pst@object{taput}}
\def\taput@i{\pst@makebox{\psput@tput{V}{true}}}
\def\tbput{\pst@object{tbput}}
\def\tbput@i{\pst@makebox{\psput@tput{V}{false}}}
\def\tx@HPutAdjust{HPutAdjust }
\def\tx@VPutAdjust{VPutAdjust }
\def\psput@tput#1#2{%
    \begingroup
	\use@par
	\pst@tputmakesmall
	\leavevmode
	\hbox{%
	    \pst@Verb{%
		\pst@nodedict
		/t \psk@tpos \pst@tposflip def
		tx@NodeDict /HPutPos known
		    { #1PutPos }
		    { CP /Y ED /X ED /NAngle 0 def /NCLW 0 def } ifelse
		/Sin NAngle sin def
		/Cos NAngle cos def
		/s \pst@number\pslabelsep NCLW add def
		/l \pst@number\pst@dima def
		/r \pst@number\pst@dimb def
		/h \pst@number\pst@dimc def
		/d \pst@number\pst@dimd def
% DG/SR modification begin - Sep. 26, 1997 - Patch 4
		%\ifnum1=0#2\else
		\ifnum1=0#2 \else
% DG/SR modification end
		    /flag #2 def
		    \csname tx@#1PutAdjust\endcsname
		\fi
		\tx@LPutCoor
		end
		\tx@PutBegin
	    }%
	    \box\pst@hbox
	    \pst@Verb{\tx@PutEnd}%
	}%
    \endgroup
    \pst@shortput%
}
\def\pst@tposflip{}
\def\pst@tputmakesmall{%
\pst@dima=\wd\pst@hbox
\divide\pst@dima 2
\pst@dimg=\psk@href\pst@dimg
\pst@dimb\pst@dima
\advance\pst@dima\pst@dimg % leftsize
\advance\pst@dimb-\pst@dimg % rightsize
\pst@dimd=\psk@vref\relax
\pst@dimc=\ht\pst@hbox
\advance\pst@dimc-\pst@dimd % height
\advance\pst@dimd\dp\pst@hbox % depth
\setbox\pst@hbox=\hbox to\z@{%
\kern-\pst@dima\vbox to\z@{\vss\box\pst@hbox\vskip-\pst@dimd}\hss}}
\def\MakeShortNab#1#2{%
  \def\pst@shortput@nab{%
    \def\pst@tempg{\next}%
    \ifx#1\next
      \let\pst@tempg\naput
    \else
      \ifx#2\next
        \let\pst@tempg\nbput
      \else
        \ifx\@sptoken\next
          \let\pst@tempg\pst@shortput
        \fi
      \fi
    \fi
    \pst@tempg}}
\MakeShortNab{^}{_}
\def\MakeShortTablr#1#2#3#4{%
  \def\pst@shortput@tablr{%
    \def\pst@tempg{\next}%
    \ifx#1\next
      \let\pst@tempg\taput
    \else
      \ifx#2\next
        \let\pst@tempg\tbput
      \else
        \ifx#3\next
          \let\pst@tempg\tlput
        \else
          \ifx#4\next
            \let\pst@tempg\trput
          \else
            \ifx\@sptoken\next
              \let\pst@tempg\pst@shortput
            \fi
          \fi
        \fi
      \fi
    \fi
    \pst@tempg}}
\MakeShortTablr{^}{_}{<}{>}
\def\MakeShortTab#1#2{%
  \def\pst@shortput@tab{%
    \def\pst@tempg{\next}%
    \ifx#1\next%
      \def\pst@tempg{%
        \@nameuse{%
          t\ifodd\psk@treemode\ifpstreeflip b\else a\fi%
          \else\ifpstreeflip r\else l\fi\fi put}}%
    \else%
      \ifx#2\next%
        \def\pst@tempg{%
          \@nameuse{%
            t\ifodd\psk@treemode\ifpstreeflip a\else b\fi%
            \else\ifpstreeflip l\else r\fi\fi put}}%
      \else\ifx\@sptoken\next\let\pst@tempg\pst@shortput\fi%
      \fi%
    \fi%
    \pst@tempg}}
\MakeShortTab{^}{_}
\define@key[psset]{pst-node}{shortput}{%
  \def\pst@tempg{#1}%
  \ifx\pst@tempg\@none\let\pst@shortput\ignorespaces%
  \else%
    \@ifundefined{pst@shortput@#1}%
      {\@pstrickserr{Bad short put: `#1'}\@ehpa}%
      {\edef\pst@shortput{\noexpand\afterassignment\expandafter\noexpand%
         \csname pst@shortput@#1\endcsname\noexpand\let\noexpand\next}}%
  \fi%
}
\psset[pst-node]{shortput=none}
%
\def\lput{\def\pst@par{}\pst@ifstar{\@ifnextchar[{\lput@i}{\lput@ii}}}
\def\lput@i[#1]{\addto@par{ref=#1}\lput@ii}
\def\lput@ii{\@ifnextchar({\lput@iv}{\lput@iii}}
\def\lput@iii#1{\addto@par{nrot=#1}\@ifnextchar({\lput@iv}{\ncput@i}}
\def\lput@iv(#1){\addto@par{npos=#1}\ncput@i}
\def\mput{\def\pst@par{}\pst@ifstar{\@ifnextchar[{\mput@i}{\ncput@i}}}
\def\mput@i[#1]{\addto@par{ref=#1}\ncput@i}
\def\Lput{\def\pst@par{}\pst@ifstar{\@ifnextchar[{\Lput@ii}{\Lput@i}}}
\def\Lput@i#1{\addto@par{labelsep=#1}\Lput@ii}
\def\Lput@ii[#1]{\addto@par{ref={#1}}\@ifnextchar({\Lput@iv}{\Lput@iii}}
\def\Lput@iii#1{\addto@par{nrot={#1}}\@ifnextchar({\Lput@iv}{\Lput@v}}
\def\Lput@iv(#1){\addto@par{npos=#1}\Lput@v}
\def\Lput@v{\pst@killglue\pst@makebox{\Lput@vi}}
\def\Lput@vi{%
    \begingroup%
	\use@par%
	\if@star\pst@starbox\fi%
	\Rput@vi%
	\pst@makesmall\pst@hbox%
	\pst@rotate\psk@nrot\pst@hbox%
	\ncput@iii%
    \endgroup%
    \pst@shortput%
}
\def\Mput{\def\pst@par{}\pst@ifstar{\@ifnextchar[{\Mput@ii}{\Mput@i}}}
\def\Mput@i#1{\addto@par{labelsep=#1}\Mput@ii}
\def\Mput@ii[#1]{\addto@par{ref={#1}}\Lput@v}
\def\aput@#1{\def\pst@par{}\pst@ifstar{\@ifnextchar[{\aput@i#1}{\aput@ii#1}}}
\def\aput@i#1[#2]{\addto@par{labelsep=#2}\aput@ii#1}
\def\aput@ii#1{\@ifnextchar({\aput@iv#1}{\aput@iii#1}}
\def\aput@iii#1#2{\addto@par{nrot=#2}\@ifnextchar({\aput@iv#1}{#1}}
\def\aput@iv#1(#2){\addto@par{npos=#2}#1}
\def\aput{\aput@\naput@i}
\def\bput{\aput@\nbput@i}
\def\Aput{\def\pst@par{}\pst@ifstar{\@ifnextchar[{\Aput@i}{\naput@i}}}
\def\Aput@i[#1]{\addto@par{labelsep=#1}\naput@i}
\def\Bput{\def\pst@par{}\pst@ifstar{\@ifnextchar[{\Bput@i}{\nbput@i}}}
\def\Bput@i[#1]{\addto@par{labelsep=#1}\nbput@i}
\def\node@coor#1;#2\@nil{%
\pst@getnode{#1}\pst@tempg
\edef\pst@coor{%
\pst@nodedict
tx@NodeDict \pst@tempg known
{ \pst@tempg load \tx@GetCenter }
{ 0 0 }
ifelse
end }}
\def\Node@coor[#1]#2;#3\@nil{%
  \begingroup
  \psset{#1}%
  \@ifnextchar\bgroup{\Node@@@coor}{\Node@@coor}#2\@nil
  \endgroup
  \let\pst@coor\pst@tempg}
\def\Node@@coor#1\@nil{%
  \pst@getnode{#1}\pst@tempg
  \xdef\pst@tempg{%
    \pst@nodedict
    tx@NodeDict \pst@tempg known
    { \psk@nodesepA \psk@angleA
      \pst@tempg load \psk@nodeseptypeA \tx@GetEdge
      \psk@offsetA \psk@angleA \tx@AddOffset
      \pst@tempg load \tx@GetCenter
      3 -1 roll add 3 1 roll add exch }
    { CP }
    ifelse
    end 
  }%
}%
\def\Node@@@coor#1{%
    \pst@@getcoor{#1}%
\def\psk@angleA{%
\pst@tempg load \tx@GetCenter \pst@coor
3 -1 roll sub 3 1 roll sub neg \tx@Atan}%
\Node@@coor}
\def\nput{\pst@object{nput}}
\def\nput@i#1#2{\pst@killglue\pst@makebox{\nput@ii{#1}{#2}}}
\def\nput@ii#1#2{%
    \begingroup
	\use@par
	\psset{refangle=#1}%
	\let\psk@angleA\psk@refangle
	\edef\psk@nodesepA{\pst@number\pslabelsep}%
	\def\psk@nodeseptypeA{0 }%
	\pslabelsep\z@
	\uput@vi
	\Node@@coor#2\@nil
	\let\pst@coor\pst@tempg
	\leavevmode
	\psput@special\pst@hbox
    \endgroup
    \ignorespaces%
}
%
\newcount\psrow
\newcount\pscol
\newcount\psmatrixcnt
\newskip\psrowsep
\newskip\pscolsep
\define@key[psset]{pst-node}{colsep}{\pssetlength\pscolsep{#1}}
\define@key[psset]{pst-node}{rowsep}{\pssetlength\psrowsep{#1}}
\psset[pst-node]{colsep=1.5cm,rowsep=1.5cm}
\newif\ifpsmatrix
% DG/SR modification begin - Nov. 27, 1998 - Patch 8
%\let\mscount\@multicnt
\ifx\mscount\@undefined\let\mscount\@multicnt\fi
% DG/SR modification end
\def\psmatrix{%
    \begingroup
    {\ifnum0=`}\fi % Don't want to expand any &.
    \@ifnextchar[{\psmatrix@i}{\ifnum0=`{\fi}{}\psmatrix@ii}%
}
\def\psmatrix@i[#1]{%
    \ifnum0=`{\fi}{}%
    \psset{#1}%
    \psmatrix@ii%
}
\def\psmatrix@ii{%
    \KillGlue
    \edef\psm@beginmath{%
	\ifmmode$\m@th\ifinner\textstyle\else\displaystyle\fi\fi}%
    \edef\psm@endmath{\ifmmode$\fi}%
    \let\\\psm@cr
    \advance\psmatrixcnt 1
    \def\psm@thenode{M-\the\psmatrixcnt-\the\psrow-\the\pscol}%
    \tabskip\z@
    \psrow1
    \pscol\z@
    \psset[pst-node]{shortput=tablr}%
    \leavevmode
    \vbox\bgroup\halign\bgroup&%
    \begingroup
	\global\advance\pscol 1
	\csname psrowhook\romannumeral\psrow\endcsname
	\csname pscolhook\romannumeral\pscol\endcsname
	\psm@beginnode##\psm@endnode\endgroup
	\cr%
}
\def\endpsmatrix{%
    \crcr\egroup\unskip\egroup
    \endgroup%
}
\def\psm@cr{{\ifnum0=`}\fi\@ifnextchar[{\psm@@cr}{\psm@@@cr{}}}
\def\psm@@cr[#1]{\psm@@@cr{\vskip#1\relax}}
\def\psm@@@cr#1{%
    \ifnum0=`{\fi}{}\cr
    \noalign{%
	\global\advance\psrow 1
	\global\pscol\z@
	\vskip\psrowsep
	#1%
    }%
}
\def\psm@beginnode{%
    \@ifnextchar\psm@endnode
	{\let\psm@endnode@i\relax\setbox\pst@hbox=\hbox{}}%
	{\pst@object{psm@beginnode}}%
}
\def\psm@beginnode@i{%
    \setbox\pst@hbox=\hbox\bgroup
    \psm@beginmath
    \begingroup
    \ignorespaces%
}
\def\psm@endnode@i{%
    \unskip
    \endgroup
    \psm@endmath
    \egroup
    \use@par
    \@psttrue%
}
\def\psm@endnode{%
    \@pstfalse
    \psm@endnode@i
    \ifnum\pscol>1 \hskip\pscolsep \fi
    \psk@mnodesize
    \hfil
    \nodealigntrue
    \if@pst
	\csname mnode@\psk@mnode\endcsname
    \else
	\csname mnode@\psk@emnode\endcsname
    \fi
    \psk@mcol
    \psk@@mnodesize%
}
% DG/SR modification begin - Sep. 3, 1999 - Patch 10 - From Michael Sharpe
%\def\psspan#1{\mscount#1\relax\loop\ifnum\mscount>\@ne \sp@n\repeat}
\def\psspan#1{\global\mscount#1\relax\pstloop\ifnum\mscount>\@ne\sp@n\repeat}
\def\pstloop#1\repeat{\gdef\pstiterate{#1\relax\expandafter\pstiterate\fi}%
  \pstiterate%
  \let\pstiterate\relax}
% DG/SR modification end
\define@key[psset]{pst-node}{name}{\pst@getnode{#1}\psk@name}
\let\psk@name\relax
\define@key[psset]{pst-node}{mcol}{%
    \ifx r#1\relax%
	\let\psk@mcol\relax%
    \else%
	\ifx l#1\relax%
	    \let\psk@mcol\hfill%
	\else%
	    \let\psk@mcol\hfil%
	\fi%
    \fi%
}
\define@key[psset]{pst-node}{mnodesize}{%
    \pssetlength\pst@dimg{#1}%
    \ifdim\pst@dimg<\z@
	\let\psk@mnodesize\relax
	\let\psk@@mnodesize\relax
    \else
	\edef\psk@mnodesize{\noexpand\hbox to\number\pst@dimg sp\noexpand\bgroup}%
	\let\psk@@mnodesize\egroup
    \fi%
}
\psset[pst-node]{mnodesize=-1pt,mcol=c}
\def\mnode@R{\rnode@iii\Rnode@ii{\psm@thenode}}
\def\mnode@r{\rnode@iii\rnode@iv{\psm@thenode}}
\def\mnode@oval{\ovalnode@ii{\psm@thenode}}
\def\mnode@tri{\trinode@ii{\psm@thenode}}
\def\mnode@dia{\dianode@ii{\psm@thenode}}
\def\mnode@C{{\nodealigntrue\cnode@ii(\z@,\z@){\psk@radius}{\psm@thenode}}}
\def\mnode@f{{\nodealigntrue\fnode@ii(\z@,\z@){\psm@thenode}}}
\def\mnode@circle{\circlenode@ii{\psm@thenode}}
\def\mnode@p{\pnode(\z@,\z@){\psm@thenode}}
% DG/SR modification begin - Jul. 22, 1997 - Patch 1
\def\mnode@dot{\dotnode@ii(\z@,\z@){\psm@thenode}}
% DG/SR modification end
\def\mnode@none{\box\pst@hbox}
\define@key[psset]{pst-node}{mnode}{%
  \@ifundefined{mnode@#1}%
    {\@pstrickserr{\string\psmatrix\space node `#1' not defined.}\@ehpa}%
    {\edef\psk@mnode{#1}}}
\define@key[psset]{pst-node}{emnode}{%
  \@ifundefined{mnode@#1}%
    {\@pstrickserr{\string\psmatrix\space node `#1' not defined.}\@ehpa}%
    {\edef\psk@emnode{#1}}}
\psset[pst-node]{mnode=R,emnode=none}
%
%%%% FROM pst-coil.tex
\def\nccoil{\pst@object{nccoil}}
\def\nccoil@i{\check@arrow{\nccoil@ii}}
\def\nccoil@ii#1#2{\nc@object{Open}{#1}{#2}{.5}{%
  \tx@NCCoor
  tx@Dict begin
% DG/SR modification begin - Mar. 19, 1998 - Patch 5
  4 2 roll
% DG/SR modification end
  \psk@coilwidth \pscoilheight
  \psk@coilarmA \psk@coilarmB
  \psk@coilaspect \psk@coilinc
  \pst@coildict \tx@Coil end
  end}}
\def\nczigzag{\pst@object{nczigzag}}
\def\nczigzag@i{\check@arrow{\nczigzag@ii}}
\def\nczigzag@ii#1#2{%
  \nc@object{Open}{#1}{#2}{.5}{%
    \tx@NCCoor
    tx@Dict begin
% DG/SR modification begin - Mar. 19, 1998 - Patch 5
    4 2 roll
% DG/SR modification end
    \pscoilheight
    \psk@coilwidth
    \psk@coilarmA
    \psk@coilarmB
    \pst@coildict \tx@ZigZag end
    \psline@iii
    \tx@Line
    end %
  }%
}
%
%%--------- the new 2005 stuff ---------%%
%
%   #1-------#4----------------#2
% where #1#4= #3 * #1#2
%
\def\setLNode(#1)(#2)#3#4{%
  \pst@getcoor{#1}\pst@tempa%
  \pst@getcoor{#2}\pst@tempb%
  \pnode(!%
    \pst@tempa /YA exch \pst@number\psyunit div def
    /XA exch \pst@number\psxunit div def
    \pst@tempb /YB exch \pst@number\psyunit div def
    /XB exch \pst@number\psxunit div def
    /dx XB XA sub def
    /dy YB YA sub def
    XA dx #3\space mul add YA dy #3\space mul add){#4}
}
%
% build the linear combination #2*#1+#4*#3=#5
\def\setLCNode(#1)#2(#3)#4#5{%
  \pst@getcoor{#1}\pst@tempa%
  \pst@getcoor{#3}\pst@tempb%
  \pnode(!%
    \pst@tempa /YA exch \pst@number\psyunit div def
    /XA exch \pst@number\psxunit div def
    \pst@tempb /YB exch \pst@number\psyunit div def
    /XB exch \pst@number\psxunit div def
    XA #2\space mul XB #4\space mul add
    YA #2\space mul YB #4\space mul add){#5}
}
%
\newif\ifPst@trueAngle
\define@key[psset]{pst-node}{trueAngle}[true]{\@nameuse{Pst@trueAngle#1}}
\psset[pst-node]{trueAngle=false}
%
\def\psRelLine{\pst@object{psRelLine}}
\def\psRelLine@i{\@ifnextchar({\psRelLine@iii}{\psRelLine@ii}}
\def\psRelLine@ii#1{%
  \addto@par{arrows=#1}%
  \psRelLine@iii%
}
\def\psRelLine@iii(#1)(#2)#3#4{{
  \pst@killglue
  \use@par
  \pst@getcoor{#1}\pst@tempa
  \pst@getcoor{#2}\pst@tempb
  \pnode(!
    \pst@tempa /YA exch \pst@number\psyunit div def
    /XA exch \pst@number\psxunit div def
    \pst@tempb /YB exch \pst@number\psyunit div def
    /XB exch \pst@number\psxunit div def
    /AlphaStrich \psk@angleA\space def
    /unit \pst@number\psyunit \pst@number\psxunit div def % yunit/xunit
%            
    /dx XB XA sub  def
    /dy YB YA sub \ifPst@trueAngle\space unit mul \fi\space def
    /laenge dy dup mul dx dup mul add sqrt #3 mul def
    /Alpha dy dx atan def 
    /beta Alpha AlphaStrich add def
    laenge beta cos mul XA add
    laenge beta sin mul \ifPst@trueAngle\space unit div \fi\space YA add ){#4}%
  \psline(#1)(#4)%
}\ignorespaces}
%
% #1 options
% draw a parallel line to #2 #3
%     #2---------#3
%         #4----------#5(new node)
% #5 length of the line
% #6 node name
\def\psParallelLine{\pst@object{psParallelLine}}
\def\psParallelLine@i{\@ifnextchar({\psParallelLine@iii}{\psParallelLine@ii}}
\def\psParallelLine@ii#1{%
  \addto@par{arrows=#1}%
  \psParallelLine@iii%
}
\def\psParallelLine@iii(#1)(#2)(#3)#4#5{{
  \pst@killglue
  \use@par
  \pst@getcoor{#1}\pst@tempa
  \pst@getcoor{#2}\pst@tempb
  \pst@getcoor{#3}\pst@tempc
%  \pst@getlength{#4}\pst@dima
  \pnode(!%
     \pst@tempa /YA exch \pst@number\psyunit div def
     /XA exch \pst@number\psxunit div def
     \pst@tempb /YB exch \pst@number\psyunit div def
     /XB exch \pst@number\psxunit div def
     \pst@tempc /YC exch \pst@number\psyunit div def
     /XC exch \pst@number\psxunit div def
%            
    /dx XB XA sub  def
    /dy YB YA sub  def
    /laenge dy dup mul dx dup mul add sqrt #4 mul def
    /Alpha dy dx atan def 
    laenge Alpha cos mul XC add
    laenge Alpha sin mul YC add ){#5}%
  \psline(#3)(#5)
}\ignorespaces}
%
\def\psIntersectionPoint(#1)(#2)(#3)(#4)#5{%
    \pst@getcoor{#1}\pst@tempa
    \pst@getcoor{#2}\pst@tempb
    \pst@getcoor{#3}\pst@tempc
    \pst@getcoor{#4}\pst@tempd
\pnode(!%
     \pst@tempa /YA exch \pst@number\psyunit div def
     /XA exch \pst@number\psxunit div def
     \pst@tempb /YB exch \pst@number\psyunit div def
     /XB exch \pst@number\psxunit div def
     \pst@tempc /YC exch \pst@number\psyunit div def
     /XC exch \pst@number\psxunit div def
     \pst@tempd /YD exch \pst@number\psyunit div def
     /XD exch \pst@number\psxunit div def
    /dY1 YB YA sub def
    /dX1 XB XA sub def
    /dY2 YD YC sub def
    /dX2 XD XC sub def
    dX1 abs 0.01 lt {
        /m2 dY2 dX2 div def
        XA dup XC sub m2 mul YC add
    }{
        dX2 abs 0.01 lt {
            /m1 dY1 dX1 div def
            XC dup XA sub m1 mul YA add
        }{%
            /m1 dY1 dX1 div def
            /m2 dY2 dX2 div def
            m1 XA mul m2 XC mul sub YA sub YC add m1 m2 sub div dup
            XA sub m1 mul YA add
        } ifelse
    } ifelse ){#5}%
}
%
%
\def\resetPSTNodeOptions{%
\psset[pst-node]{%  set only the options of pst-node
  nodealign=false,%
  href=0,%
  vref=.7ex,%
  framesize=10pt,%
  nodesep=0pt,%
  arm=10pt,%
  offset=0pt,%
  angle=0,%
  arcangle=8,%
  ncurv=.67,%
  loopsize=1cm,%
  boxsize=.4cm,%
%  nrot=0,%
%  npos=,%
  tpos=0.5,%
  shortput=none,%
  colsep=1.5cm,%
  rowsep=1.5cm,%
%  shortput=tablr,%%
  mcol=c,%
  mnode=R,%
  emnode=none,
  trueAngle=false}
}
\reserPSTNodeOptions
%
\catcode`\@=\PstAtCode\relax
\endinput
%%
%% END pst-node.tex
